% $date: 2015-04-10
% $timetable:
%   g9r2:
%     2015-04-10:
%       2:

\section*{Индукция}

% $authors:
% - Александр Шаповалов

% $matter[-no-ashap-link,no-header]:
% - .[no-ashap-link]
% - /_ashap-link

Математическая индукция помогает коротко записать строгое решения,
но~не~объясняет, как его придумать, и~в~чем его смысл.

\subsection*{Индуктивное построение}

Наиболее оправдано применение индукции при построении сложных конструкций,
когда очередной этаж строится на~основе уже построенных нижних этажей.
Такое построение может быть при необходимости преобразовано в~явный алгоритм.

\begin{problems}

\item
От~прямоугольника с~неравными сторонами отрезают квадрат со~стороной, равной
меньшей стороне прямоугольника.
Если оставшаяся часть не~квадрат, процесс повторяют.
Докажите, что для любого $n$ найдется прямоугольник, для которого процесс
закончится ровно после $n$-го отрезания, причем все отрезанные квадраты будут
разного размера.

\item
В~компании из~$n$ человек ($n \geq 4$) каждый узнал по~новости.
Созвонившись, двое рассказывают друг другу все известные им~новости.
Как за~$2 n - 4$ звонка все смогут узнать все новости?

\end{problems}

\subsection*{Рекурсия}

\emph{Редукция} сводит решение задачи к~более простой.
Пусть удается свести к~такой~же задаче с~меньшим значением полуинварианта.
Если полуинвариант не~может уменьшаться бесконечно, а~для его крайних значений
задача решена, то~это~--- \emph{рекурсия.}
Такую \emph{цепочку редукций} тоже оформляют как индукцию, объявляя
полуинвариант параметром индукции.

\begin{problems}

\item
В~городе 100 домов.
Какое наибольшее число замкнутых непересекающихся заборов можно построить так,
чтобы любые два забора ограничивали разные группы домов?

\end{problems}

Индукция незаменима при логической рекурсии
(<<иль думал, что я думала, что думал он~я сплю>>).

\begin{problems}

\item
10~бандитов ограбили банк на~миллион долларов и~уселись в~ряд за~стол делить
деньги.
Сначала первый предлагает, кому сколько: мне столько-то, второму столько-то
и~т.~д., и~все 10 голосуют.
Если <<за>> не~менее половины, то~предложение принимается, каждый получает
предложенную долю, и~все расходятся.
Если более половины голосуют <<против>>, первого убивают, и~тогда уже второй
бандит предлагает кому сколько на~тех~же условиях, и~т.~д.
Каждый бандит руководствуется в~первую очередь желанием выжить, во~вторую
(если жизнь вне опасности)~--- получить побольше денег, в~третью
(если на~жизнь и~сумму это не~влияет)~--- не~убивать без необходимости
(дело-то не~последнее!).
Как распределятся деньги, если все бандиты будут действовать и~рассуждать
абсолютно логически?

\end{problems}

