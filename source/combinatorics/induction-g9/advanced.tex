% $date: 2015-04-10
% $timetable:
%   g9r1:
%     2015-04-10: {}
%   g9r2:
%     2015-04-10: {}

% $matter[-contained,no-header]:
% - verbatim: \section*{Индукция}
% - .[contained]

\subsection*{Дополнительные задачи}

% $authors:
% - Александр Шаповалов

% $matter[-no-ashap-link,contained,no-header]:
% - .[no-ashap-link]
% - /_ashap-link

\begin{problems}

\item
В~классе каждый болтун дружит хотя~бы с~одним молчуном.
При этом болтун молчит, если в~кабинете находится нечетное число его
друзей-молчунов.
Докажите, что учитель может пригласить на~факультатив не~менее половины класса
так, чтобы все болтуны молчали.

\item
Есть гири с~номерами от~$1$ до~$n$, для каждого $k$ вес $k$-й гирьки целый
и~не~превосходит $k$, а~сумма всех весов четна.
Докажите, что все гири можно разбить на~две кучки равного веса.

\item
В~вершинах связного графа с~$n$~вершинами записано по~два положительных числа:
синее и~красное, причем сумма синих равна сумме красных.
За~один ход можно изменить два синих числа в~концах любого одного ребра так,
чтобы чтобы они остались положительными и~сумма сохранилась.
Докажите, что не~более чем за~$(n - 1)$ ход можно добиться, чтобы в~каждой
вершине синее число стало равно красному.

\end{problems}

