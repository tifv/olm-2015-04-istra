% $date: 2015-04-09
% $timetable:
%   g11r1:
%     2015-04-09:
%       1:

\section*{Алгоритмы и стратегии}

% $authors:
% - Олег Орлов

\begin{problems}

\item
На~плоскости нарисованы квадрат и~невидимая точка, не~лежащая на~границе
квадрата.
За~один ход Вася может провести прямую и~спросить, по~какую сторону лежит точка
(если точка лежит на~прямой, он~получает произвольный ответ).
За~какое наименьшее число вопросов он~сможет узнать, лежит~ли точка внутри
квадрата?

\item
Даны 8 гирек весом $1, 2, \ldots, 8$ грамм, но~неизвестно, какая из~них
сколько весит.
Барон Мюнхгаузен утверждает, что помнит, какая из~гирек сколько весит,
и~в~доказательство своей правоты готов провести одно взвешивание, в~результате
которого будет однозначно установлен вес хотя~бы одной из~гирь.
Не~обманывает~ли он?

\item
Двое игроков ставят крестики и~нолики на~бесконечной клетчатой бумаге, первый
крестик, второй~--- нолик, первый~--- крестик, второй~--- нолик и~т.~д.
Докажите, что первый может добиться, чтобы некоторые четыре крестика образовали
квадрат (со~сторонами, параллельными линиям клеток).

\item
Двое по~очереди закрашивают клетки доски $999 \times 999$ черным цветом
(в~начале доска полностью белая).
Нельзя красить клетку дважды, нельзя чтобы в~одной строке или столбце было
больше двух закрашенных клеток.
Проигрывает тот, кто не~может сделать ход.
Кто выигрывает при правильной игре?

\item
Император пригласил на~праздник $2015$ волшебников, некоторые из~которых
добрые, а~остальные злые.
Добрый волшебник всегда говорит правду, а~злой может говорить что угодно.
При этом волшебники знают, кто добрый и~кто злой, а~император нет.
На~празднике император задает каждому волшебнику (в~каком хочет порядке)
по~вопросу, на~которые можно ответить <<да>> или <<нет>>.
Опросив всех волшебников, император изгоняет одного.
Изгнанный волшебник выходит в~заколдованную дверь, и~император узнает, добрый
он~был или злой.
Затем император вновь задает каждому из~оставшихся волшебников по~вопросу,
вновь одного изгоняет, и~так далее, пока император не~решит остановиться
(он~может это сделать после любого вопроса).
Докажите, что император может изгнать всех злых волшебников, удалив при этом
не~более одного доброго.

\item
\subproblem
Двое показывают карточный фокус.
Первый снимает пять карт из~колоды, содержащей $52$ карты (предварительно
перетасованной кем-то из~зрителей), смотрит в~них и~после этого выкладывает
их~в~ряд слева направо, причем одну из~карт кладет рубашкой вверх,
а~остальные~--- картинкой вверх.
Второй участник фокуса отгадывает закрытую карту.
Докажите, что они могут так договориться, что второй всегда будет угадывать
карту.
\\
\subproblem
Второй фокус отличается от~первого тем, что первый участник выкладывает слева
направо четыре карты картинкой вверх, а~одну не~выкладывает.
Могут~ли в~этом случае участники фокуса так договориться, чтобы второй всегда
угадывал невыложенную карту?

\end{problems}

