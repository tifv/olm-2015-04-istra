% $date: 2015-04-07
% $timetable:
%   g11r2:
%     2015-04-07:
%       1:

\section*{Алгоритмы и стратегии}

% $authors:
% - Олег Орлов

\begin{problems}

\item
На~плоскости нарисованы квадрат и~невидимая точка, не~лежащая на~границе
квадрата.
За~один ход Вася может провести прямую и~спросить, по~какую сторону лежит точка
(если точка лежит на~прямой, он~получает произвольный ответ).
За~какое наименьшее число вопросов он~сможет узнать, лежит~ли точка внутри
квадрата?

\item
Даны 8 гирек весом $1, 2, \ldots, 8$ грамм, но~неизвестно, какая из~них
сколько весит.
Барон Мюнхгаузен утверждает, что помнит, какая из~гирек сколько весит,
и~в~доказательство своей правоты готов провести одно взвешивание, в~результате
которого будет однозначно установлен вес хотя~бы одной из~гирь.
Не~обманывает~ли он?

\item
Двое игроков ставят крестики и~нолики на~бесконечной клетчатой бумаге, первый
крестик, второй~--- нолик, первый~--- крестик, второй~--- нолик и~т.~д.
Докажите, что первый может добиться, чтобы некоторые четыре крестика образовали
квадрат (со~сторонами, параллельными линиям клеток).

\item
Среди пяти внешне одинаковых монет три настоящие и~две фальшивые, одинаковые
по~весу, но~неизвестно, тяжелее или легче настоящих.
Как за~наименьшее число взвешиваний найти хотя~бы одну настоящую монету?

\item
На~бесконечной клетчатой полоске шириной в~одну клетку расставлены две ловушки,
между которыми находится $n$ свободных клеток, в~одной из~которых сидит
своенравный кузнечик.
Кузя и~кузнечик играют в~игру.
Кузя называет натуральное число, а~кузнечик прыгает на~названное число клеток
куда ему заблагорассудится (т.~е. вправо или влево).
При каких $n$ вне зависимости от~начального положения кузнечика Кузе удастся
загнать кузнечика в~ловушку?

\item
Переаттестация Совета Мудрецов происходит так: король выстраивает их~в~колонну
по~одному и~надевает каждому колпак черного или белого цвета.
Все мудрецы видят цвета всех колпаков впереди стоящих мудрецов, а~цвет своего
и~всех стоящих сзади не~видят.
Раз в~минуту один из~мудрецов должен выкрикнуть один из~двух цветов (каждый
мудрец выкрикивает цвет один раз).
После окончания этого процесса король казнит каждого мудреца, выкрикнувшего
цвет, отличный от~цвета его колпака.
Накануне переаттестации все сто членов Совета Мудрецов договорились
и~придумали, как минимизировать число казненных.
Скольким из~них гарантированно удастся избежать казни?

\item
Двое по~очереди закрашивают клетки доски $999 \times 999$ черным цветом
(в~начале доска полностью белая).
Нельзя красить клетку дважды, нельзя чтобы в~одной строке или столбце было
больше двух закрашенных клеток.
Проигрывает тот, кто не~может сделать ход.
Кто выигрывает при правильной игре?

\end{problems}

