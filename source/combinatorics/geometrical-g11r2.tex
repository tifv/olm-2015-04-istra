% $date: 2015-04-03
% $timetable:
%   g11r2:
%     2015-04-03:
%       2:

\section*{Комбинаторная геометрия}

% $authors:
% - Олег Орлов

\begin{problems}

\item
Из~точки~$O$ выходит несколько лучей.
Угол между любыми двумя меньше $120^{\circ}$.
Докажите, что найдутся два луча такие, что все остальные содержатся в~угле
между ними.

\item
Длина наибольшей стороны треугольника равна~$1$.
Докажите, что три круга радиуса $\frac{1}{\sqrt{3}}$ с~центрами в~вершинах
покрывают треугольник целиком.

\item
На~плоскости дано $5$~точек, причем никакие три из~них не~лежат на~одной
прямой.
Докажите, что четыре из~этих точек расположены в~вершинах выпуклого
многоугольника.

\item
Могут~ли в~выпуклом пятиугольнике $ABCDE$ углы $ACD$, $BDE$, $CEA$, $DAB$,
$EBC$ быть тупыми?

\item
Внутри выпуклого $n$-угольника отмечена точка.
Докажите, что ее~проекция на~одну из~сторон попадет строго на~сторону,
а~не~на~продолжение.

%\item
%На~плоскости дано $n$ различных точек.
%Отметили середины всех отрезков их~соединяющих.
%Докажите, что отметили хотя~бы $(2 n - 3)$ различные точки.

\item
На~прямой расположено $2 n + 1$ отрезков.
Каждый из~них пересекает не~менее $n$ других.
Докажите, что один из~отрезков пересекает все остальные.

\item
Внутри выпуклого стоугольника отмечено $k$ точек ($2 \leq k \leq 50$).
Докажите, что можно выбрать $2 k$ вершин стоугольника так, чтобы $2 k$-угольник
с~вершинами в~них содержал все отмеченные точки (точки могут лежать внутри или
на~границе $2 k$-угольника).

\item
\sp
Вершины выпуклого пятиугольника расположены в~узлах целочисленной сетки.
Докажите, что внутри пятиугольника есть хотя~бы одна целая точка.
\\
\sp
Вершины выпуклого пятиугольника расположены в~узлах целочисленной решетки.
В~пятиугольнике провели все диагонали.
Докажите, что существует узел решетки внутри маленького пятиугольника.

\item
Конечное множество точек на~плоскости удовлетворяет следующему условию: для
любых двух точек множества на~прямой, их~соединяющей, найдется третья точка
из~множества.
Докажите, что все точки множества лежат на~одной прямой.

\end{problems}

