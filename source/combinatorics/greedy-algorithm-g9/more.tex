% $date: 2015-03-31
% $timetable:
%   g9r2:
%     2015-03-31: {}
%   g9r1:
%     2015-03-31: {}

% $matter[-contained,no-header]:
% - verbatim: \section*{Жадный алгоритм}
% - .[contained]

\subsection*{Домашнее задание}

% $authors:
% - Александр Шаповалов

% $matter[-no-ashap-link,contained,no-header]:
% - .[no-ashap-link]
% - /_ashap-link

\begin{problems}

\item
На блюде лежат 15 кусков сыра двух весов.
Сначала Вася может разрезать некоторые из этих кусков (но не все) каждый на две
части так, чтобы в результате снова получились куски двух весов.
Затем Петя берет себе один из кусков, потом Вася~--- один из оставшихся кусков,
затем снова Петя и~т.~д. пока не разберут весь сыр.
Каждый старается получить как можно больше.
Каков результат игры при наилучших действиях сторон?

% spell \text{м}

\item
Два мага сражаются друг с другом.
Вначале они оба парят над морем на высоте $100\,\text{м}$.
Маги по очереди применяют заклинания вида <<уменьшить высоту парения над морем
на $a\,\text{м}$ у себя и на $b\,\text{м}$ у соперника>>, где $a$, $b$~---
действительные числа, $0 < a < b$.
Набор заклинаний у магов конечен и одинаков, их можно использовать в любом
порядке и неоднократно.
Маг выигрывает дуэль, если после чьего-либо хода его высота над морем будет
положительна, а у соперника~--- нет.
Существует ли такой набор заклинаний, что второй маг может гарантированно
выиграть (как бы ни действовал первый)?

\item
На доске написаны числа $1, 2, 3, \ldots, 2012$.
Петя стирает их по одному.
Докажите, что он может это делать в таком порядке, чтобы сумма нестертых чисел
всегда была составным числом.

\item
На первой горизонтали шахматной доски стоят 8 одинаковых черных ферзей,
а на последней~--- 8 одинаковых белых ферзей.
За какое минимальное число ходов белые ферзи могут обменяться местами
с черными?
Ходят белые и черные по очереди, по одному ферзю за ход.

\item
На столе лежат в ряд 2014 монет, правая~--- рубль, остальные~--- монеты
в 1 копейку.
Петя и Вася по очереди берут монеты, начиная слева, по 1 или 2 монеты за ход.
Начинает Петя.
Кто из них в итоге может получить больше денег, как бы ни играл соперник?

\item
\subproblem
100 карточек в стопке пронумерованы числами от 1 до 100 сверху вниз.
Двое играющих по очереди снимают сверху по одной или нескольку карточек
и отдают противнику.
Выигрывает тот, у кого первого произведение всех чисел на карточках станет
кратно 1000000.
Каков будет результат игры при правильной игре сторон?
\\
\subproblem
Тот же вопрос при $N!$ карточек, выигрывает тот, у кого первого произведение
разделится на $N!$.

\end{problems}

