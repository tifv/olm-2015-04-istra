% $date: 2015-03-31
% $timetable:
%   g9r2:
%     2015-03-31:
%       3:
%   g9r1:
%     2015-03-31:
%       1:

\section*{Жадный алгоритм}

% $authors:
% - Александр Шаповалов

% $matter[-no-ashap-link,no-header]:
% - .[no-ashap-link]
% - /_ashap-link

\begin{flushright}
    Это жадность-то до добра не доводит?!!!
\end{flushright}

Алгоритм~--- это способ достижения цели через жестко определенную
последовательность шагов.
Когда в ответе надо предъявить алгоритм, естественно рассматривать его как
составную конструкцию.
Типичные примеры: выигрышная или ничейная стратегия в играх.
Кроме того, алгоритмы регулярно возникают в задачах на испытания.
Если цель~--- максимум какой-то величины, то ее часто достигают с помощью
<<жадного алгоритма>>, то есть, добиваясь максимально возможного приращения
на каждом шаге.

\begin{problems}

\item
На столе лежат карточки со 100 последовательными числами.
Двое игроков по очереди берут по карточке пока не разберут все.
Тот, у кого сумма меньше, выплачивает разность сумм противнику.
Каков результат игры при наилучших действиях сторон?

\item
На блюде лежат 10~кусков сыра разного веса.
Сначала Вася режет каждый из кусков на два.
Затем Петя и Вася разбирают эти 20~кусков, беря по очереди по одному, начинает
Петя.
Каждый старается получить как можно больше сыра по весу.
Каков результат игры при наилучших действиях сторон?

\end{problems}

Бывает полезно ввести вспомогательную величину для оптимизации. 

\begin{problems}

\item
За какое наименьшее число ходов конь может пройти из левого нижнего угла доски
$100 \times 100$ в правый верхний?

\end{problems}

Экономные действия позволяют строить примеры проще

\begin{problems}

\item
На плоскости нарисован черный равносторонний треугольник.
Имеется десять треугольных плиток того же размера и той же формы.
Нужно положить их на плоскости так, чтобы они не перекрывались и чтобы каждая
плитка покрывала хотя бы часть черного треугольника (хотя бы одну точку внутри
него).
Как это сделать?

\end{problems}

\subsection*{Отклонение от жадности}

Часто можно показать, что жадный алгоритм не достигает результата.
Доказав недостижимость, подумайте, нельзя ли из этого извлечь указания,
и достичь результата, следующего за жадным.

\begin{problems}

\item
$ABCD$~--- квадрат со стороной~8.
Разрешено делать шаги длины~1, не выходя за пределы квадрата.
За какое наименьшее число шагов можно пройти из $A$ в $C$?

\item
В банке работают 2002 сотрудника.
Все сотрудники пришли на юбилей, и их рассадили за один круглый стол.
Известно, что зарплаты сидящих рядом различаются на 2 или 3 доллара.
Какой наибольшей может быть разница двух зарплат сотрудников этого банка, если
известно, что все зарплаты сотрудников различны?

\item
\subproblem
На каждом из полей верхней и нижней горизонтали шахматной доски стоит по фишке:
внизу~--- белые, вверху~--- черные.
За один ход разрешается передвинуть любую фишку на соседнюю свободную клетку
по вертикали или горизонтали.
За какое наименьшее число ходов можно добиться того, чтобы все черные фишки
стояли внизу, а белые~--- вверху?
\\
\subproblem
То же для доски $9 \times 9$.

\item
На столе лежат в ряд 300 монет, правая~--- рубль, остальные~--- монеты
в 1~копейку.
Петя и Вася по очереди берут монеты, начиная слева, по 1 или 2 монеты за ход.
Начинает Петя.
Кто из них в итоге может получить больше денег, как бы ни играл соперник?

\end{problems}

