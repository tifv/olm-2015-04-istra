% $date: 2015-04-05
% $timetable:
%   g11r2:
%     2015-04-05:
%       1:

\section*{Добавка по комбинаторике}

% $authors:
% - Олег Орлов

\begin{problems}

\item
В~прямоугольной таблице некоторые клетки отмечены: в~них стоит звездочка.
Известно, что для любой отмеченной клетки число звездочек в~ее~столбце равно
числу звездочек в~ее~строке.
Докажите, что число строк таблицы, где есть хотя~бы одна звездочка, равно числу
столбцов таблицы, где есть хотя~бы одна звездочка.

\item
В~каждой клетке доски $2000 \times 2002$ находится по~лампочке.
В~начале, на~доске горит больше чем $1999 \cdot 2001$ лампочек.
За~одну операцию можно взять квадратик $2\times 2$, в~котором выключено ровно
3~лампочки и~выключить четвертую в~этом квадратике (конечно, только если такой
найдется).
Докажите, что такими операциями нельзя выключить все лампочки на~доске.

\item
В~клетчатом прямоугольнике $49 \times 69$ отмечены все $50 \cdot 70$ вершин
клеток.
Двое играют в~следующую игру: каждым своим ходом каждый игрок соединяет две
точки отрезком, при этом одна точка не~может являться концом двух проведенных
отрезков.
Отрезки могут содержать общие точки.
Отрезки проводятся до~тех пор, пока точки не~кончатся.
Если после этого первый может выбрать на~всех проведенных отрезках направления
так, что сумма всех полученных векторов равна нулевому вектору,
то~он~выигрывает, иначе выигрывает второй.
Кто выигрывает при правильной игре?

\end{problems}

