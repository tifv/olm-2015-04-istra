% $date: 2015-04-04
% $timetable:
%   g9r2:
%     2015-04-04: {}

% $matter[-contained,no-header]:
% - verbatim: \section*{Дискретная непрерывность и разрывы}
% - .[contained]

\subsection*{Домашнее задание}

% $authors:
%  - Александр Шаповалов

% $matter[-no-ashap-link,contained,no-header]:
% - .[no-ashap-link]
% - /_ashap-link

\begin{problems}

% spell Ш.

\item
В~стране Ш. человек считается \emph{богатым,} если его зарплата больше
зарплаты премьер-министра.
В~этой стране богатые мужчины предпочитают жениться на~бедных женщинах.
Все зарплаты в~стране различные.
Докажите, что можно премьер-министру установить такую зарплату, чтобы
количество богатых мужчин было в~точности равно количеству бедных женщин.

\item
В~ряд стоят несколько солдат.
Рост соседей отличается не~более чем на~$2{,}4\,\text{см}$.
\\
\subproblem
В~строю есть солдат ростом $152\,\text{см}$, и~солдат ростом $198\,\text{см}$.
Докажите, что есть солдат, чей рост отличается от~$170\,\text{см}$ не~более, чем
на~$1{,}2\,\text{см}$.
\\
\subproblem
Докажите, что если солдаты встанут по~росту, то~по-прежнему рост соседей будет
отличаться не~более, чем на~$2{,}4\,\text{см}$.

\item
На~бесконечной шахматной доске стоят две белые ладьи и~невидимый черный король.
Известно, что король может дойти до~одной ладьи (известно, какой) за~100 ходов.
Ходят по~очереди.
Докажите, что ладьи смогут поставить шах.

\item
В~одном из~100 окопов, расположенных в~ряд, спрятался робот-пехотинец.
Автоматическая пушка может одним выстрелом накрыть любой окоп.
В~каждом промежутке между выстрелами робот (если уцелел) обязательно перебегает
в~соседний окоп (быть может, только что обстрелянный).
\\
\subproblem
Известно, что вначале слева от~робота~--- нечетное число пустых окопов.
Сможет~ли пушка наверняка накрыть робота?
\\
\subproblem
Вначале робот может быть в~любом из~окопов.
Сможет~ли пушка наверняка накрыть робота?

\end{problems}

