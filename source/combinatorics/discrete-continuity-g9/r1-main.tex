% $date: 2015-04-04
% $timetable:
%   g9r1:
%     2015-04-04:
%       3:

\section*{Дискретная непрерывность и разрывы}

% $authors:
%  - Александр Шаповалов

% $matter[-no-ashap-link,no-header]:
% - .[no-ashap-link]
% - /_ashap-link

Если какая-то целочисленная величина в~процессе меняется на~каждом шаге
не~больше чем на~1 (в~ту или другую сторону), то~она обязательно проходит через
все промежуточные значения между начальным и~конечным.
Такая величина называется дискретной, а~прием~—
дискретной непрерывностью\ldots

\begin{problems}

\item
\subproblem
В~баскетбольном матче первым забил «Зенит», а~выиграл «Спартак».
Команды в~сумме набрали более 100 очков.
Верно~ли, что в~какой то~момент счет был ничейный?
(Попадание в~корзину «с~игры» приносит команде 2 или 3 очка)
\\
\subproblem
В~футбольном матче первым забил «Зенит», а~выиграл «Спартак».
Докажите, что в~какой то~момент счет был ничейный.

\item
В~ряд выложены 200 шаров, из~них 100 черных и~100 красных, причем первый
и~последний шары~— черные.
Докажите, что можно убрать с~правого края несколько шаров подряд так, чтобы
красных и~черных шаров осталось поровну.

\end{problems}

Полезно раскрасить объекты в~два цвета так, чтобы граница или разрыв отделяли
цвета.

\begin{problems}

\item
В~последовательности целых чисел каждое число, начиная со~второго, на~1 больше
предыдущего или в~3 раза меньше предыдущего.
Первое число равно 1, последнее равно 100.
Докажите, что среди чисел есть и~77.

% spell \text{г}

\item
В~ряд лежат 100 яблок, соседние отличаются не~более чем на~$10\,\text{г}$.
Докажите, что если выложить яблоки в~ряд по~возрастанию веса, то~и~тогда
соседние будут отличаться не~более чем на~$10\,\text{г}$.

\end{problems}

Если процесса нет, организуй сам.
Подбери начало и~конец процесса так, чтобы они были по~разные стороны
от~нужного значения.

\begin{problems}

\item
В~ряд сидит 15~мальчиков и~15~девочек.
\\
\subproblem
Докажите, что можно выбрать 10~школьников подряд, чтобы среди них мальчиков
и~девочек было поровну.
\\
\subproblem
Всегда~ли из~них можно выбрать 20~школьников подряд, среди которых мальчиков
и~девочек поровну?

\item
\subproblem
По~кругу сидят 30~школьников, среди них мальчиков и~девочек поровну.
Докажите, что можно выбрать 20~школьников подряд, чтобы среди них мальчиков
и~девочек было поровну.
\\
\subproblem
По~кругу сидят 30~школьников, среди них мальчиков и~девочек поровну.
Докажите, что можно выбрать 18~школьников подряд, чтобы среди них мальчиков
и~девочек было поровну.

\end{problems}

Через промежуточное значение можно не~только пройти, но~и~перепрыгнуть.

\begin{problems}

\item
\subproblem
В~строке четырехзначных чисел первое число 2015, последнее 2051.
Соседние числа отличаются на~1 или на~100.
Верно~ли, что хотя~бы одно число делится на~101?
\\
\subproblem
Дополнительно известно, что ни~одно число не~делится на~100.
Докажите, что хотя~бы одно число делится на~101.

\end{problems}

В~некоторых процессах полезно начало и~конец поменять местами.
Тут помогает расположение на~окружности.

\begin{problems}

\item
На~окружности отмечены 77 точек, среди них нет диаметрально противоположных.
Докажите, что можно провести диаметр через одну из~точек так, что по~обе
стороны диаметра точек окажется поровну.

\item
На~клетчатой доске $100 \times 100$ стоит 1000 шашек.
\\
\subproblem
Докажите, что где~бы они не~стояли, доску можно разрезать по~границам клеток
на~две части, в~одной из~которых будет 300 шашек.
\\
\subproblem
Докажите, что где~бы они не~стояли, доску можно разрезать по~границам клеток
на~две равные части с~равным количеством шашек.

\end{problems}

Из~соображений непрерывности чаще всего доказывают существование или
невозможность некого объекта или конструкции, не~предъявляя его явно.
Однако дискретная непрерывность бывает полезна и~в~задачах на~конструкцию.
Например, построив алгоритм, докажем через непрерывность, что он~сработает.

\begin{problems}

\item
На~бесконечной шахматной доске стоят две белые ладьи и~невидимый черный король.
Известно, что король может дойти до~одной ладьи (известно, какой) за~100 ходов.
Ходят по~очереди.
Докажите, что ладьи смогут поставить шах.

\end{problems}

