% $date: 2015-04-04
% $timetable:
%   g9r1:
%     2015-04-04: {}

% $matter[-contained,no-header]:
% - verbatim: \section*{Дискретная непрерывность и разрывы}
% - .[contained]

\subsection*{Домашнее задание}

% $authors:
%  - Александр Шаповалов

% $matter[-no-ashap-link,contained,no-header]:
% - .[no-ashap-link]
% - /_ashap-link

\begin{problems}

\item
В~ряд стоят несколько солдат.
Рост соседей отличается не~более чем на~$2{,}4\,\text{см}$.
\\
\subproblem
В~строю есть солдат ростом $152\,\text{см}$, и~солдат ростом $198\,\text{см}$.
Докажите, что есть солдат, чей рост отличается от~$170\,\text{см}$ не~более, чем
на~$1{,}2\,\text{см}$.
\\
\subproblem
Докажите, что если солдаты встанут по~росту, то~по-прежнему рост соседей будет
отличаться не~более, чем на~$2{,}4\,\text{см}$.

\item
За~круглым столом равномерно посажены 100 дедов, причем у~любых двух соседей
количество волос в~бородах отличается не~больше, чем на~100.
Докажите, что найдется пара дедов, сидящих напротив друг друга, у~которых
количество волос в~бородах также отличается не~больше, чем на~100.

\item
В~круге проведены несколько хорд так, что любые две из~них пересекаются внутри
круга.
Докажите, что можно пересечь все хорды одним диаметром.

\item
В~одном из~100 окопов, расположенных в~ряд, спрятался робот-пехотинец.
Автоматическая пушка может одним выстрелом накрыть любой окоп.
В~каждом промежутке между выстрелами робот (если уцелел) обязательно перебегает
в~соседний окоп (быть может, только что обстрелянный).
\\
\subproblem
Известно, что вначале слева от~робота~--- нечетное число пустых окопов.
Сможет~ли пушка наверняка накрыть робота?
\\
\subproblem
Вначале робот может быть в~любом из~окопов.
Сможет~ли пушка наверняка накрыть робота?

\end{problems}

