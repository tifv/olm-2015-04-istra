% $date: 2015-04-04
% $timetable:
%   g9r1:
%     2015-04-04: {}

% $matter[-contained,no-header]:
% - verbatim: \section*{Дискретная непрерывность и разрывы}
% - verbatim: \setproblem{4}
% - .[contained]

\subsection*{Дополнительные задачи}

% $authors:
%  - Александр Шаповалов

% $matter[-no-ashap-link,contained,no-header]:
% - .[no-ashap-link]
% - /_ashap-link

\begin{problems}

\item
На~бесконечной шахматной доске стоят две белые ладьи и~невидимый черный король.
Про положение короля ничего не~известно.
Ходят по~очереди.
Докажите, что ладьи смогут поставить шах.

\item
Дракон заточил рыцаря в~темницу и~выдал ему 100 различных монет, половина
из~которых~— фальшивые (но~какие именно~— знает только дракон).
Каждый день рыцарь раскладывает монеты на~две кучки (не~обязательно равные).
Если в~какой-то день в~этих кучках окажется поровну настоящих монет, либо
поровну фальшивых, то~дракон отпустит рыцаря.
Сможет~ли рыцарь гарантированно освободиться не~позже, чем на~двадцать пятый
день?

\item
Вершины пятидесятиугольника делят окружность на~50~дуг, длины которых равны
числам $1, 2, 3, \ldots, 50$, взятым в~каком-то порядке.
Каждая пара «противоположных» дуг (соответствующих противоположным сторонам
$50$-угольника) отличается по~длине на~25.
Докажите, что у~пятидесятиугольника найдется пара параллельных сторон.

\item
Дано натуральное число.
Разрешается расставить между цифрами числа плюсы произвольным образом
и~вычислить сумму (например, из~числа $123456789$ можно получить
$12345 + 6 + 789 = 13140$).
С~полученным числом снова разрешается выполнить подобную операцию, и~так далее.
Докажите, что из~любого числа можно получить однозначное, выполнив не~более
10 таких операций.

\end{problems}

