% $date: 2015-04-07
% $timetable:
%   g9r1:
%     2015-04-07: {}
%   g9r2:
%     2015-04-07: {}

% $matter[-contained,no-header]:
% - verbatim: \section*{Неоднозначные данные}
% - .[contained]

\subsection*{Домашнее задание}

% $authors:
% - Александр Шаповалов

% $matter[-no-ashap-link,contained,no-header]:
% - .[no-ashap-link]
% - /_ashap-link

\begin{problems}

\item\emph{Письменная задача.}
В~колоде 52 карты (4 масти, 13 достоинств).
Про любую пару карт одной масти или одного достоинства известно, сколько карт
между ними лежит.
Всегда ли по этой информации можно узнать пару крайних карт колоды?

\item
\sp
В клетки доски $8 \times 8$ записали числа $1, 2, \ldots, 64$ в неизвестном
порядке.
Разрешается узнать сумму чисел в любой паре клеток с общей стороной.
Всегда ли можно узнать расположение всех чисел?
\\
\sp
То же для доски $9 \times 9$ и чисел от 1 до 81.

\item
У $N$ ключей от $N$ гостиничных номеров потерялись бирки.
Известно, что каждый ключ открывает ровно один из номеров.
Какое наименьшее число пробных открываний дверей надо сделать, чтобы наверняка
определить, от какого номера каждый ключ?

\end{problems}

