% $date: 2015-04-07
% $timetable:
%   g9r2:
%     2015-04-07:
%       2:

\section*{Неоднозначные данные}

% $authors:
% - Александр Шаповалов

% $matter[-no-ashap-link,no-header]:
% - .[no-ashap-link]
% - /_ashap-link

\begin{flushright}
<<А это вам знать пока рано>>,~--- сказала Баба-Яга своим 33 ученикам
и скомандовала:
<<Закройте глаза!>>.
Правый глаз закрыли все мальчики и треть девочек.
Левый глаз закрыли все девочки и треть мальчиков.
Сколько учеников все-таки увидели то, что знать пока рано?
\end{flushright}

\subsection*{Неразличимые примеры}

Чтобы доказать, что информации недостаточно для получения однозначного ответа,
можно построить два примера, которые удовлетворяют всем условиям, но дают
разные ответы.

\begin{problems}

\item
В ряд выписаны 100 чисел, первое равно~3, а сумма любых трех подряд равна 100.
Можно ли наверняка узнать, чему равно
\\
\subproblem 100-е число?
\quad
\subproblem 50-е число?

\item
\subproblem
Незнайка утверждает, что он может узнать с помощью чашечных весов без гирь,
есть ли среди любых трех камней такой, вес которого равен $1/3$ общего веса.
Не хвастает ли он?
\\
\subproblem
А узнать, есть ли среди десяти камней камень веса $1/10$ от общего веса?

\item
У Кащея есть куб, в каждой вершине которого вставлено по алмазу.
Известны веса этих алмазов: 1 карат, 2 карата, \ldots, 8 карат.
Кащей предлагает Ивану Царевичу такую игру: он сообщает Ивану сумму весов
алмазов на каждом ребре.
Если после этого Иван правильно назовет, куда какой по весу алмаз вставлен,
то получит этот куб вместе с алмазами, а если хотя бы в одном месте ошибется,
то распрощается с головой.
Стоит ли Ивану соглашаться играть?

\end{problems}

\subsection*{Примеры <<задним числом>>}
Неразличимые примеры и контрпримеры могут строится после того, как испытания
уже проведены и ответы даны, с использованием уже полученной информации.
Этот метод часто применяется, чтобы опровергнуть предположение о наличие
<<гарантированного>> алгоритма.

\begin{problems}

\item
На плоскости расположен квадрат, и невидимыми чернилами нанесена точка~$P$.
Человек в специальных очках видит точку.
Если провести прямую, то он отвечает на вопрос, по какую сторону от нее
лежит~$P$ (если $P$ лежит на прямой, то он говорит, что $P$ лежит на прямой).
Нужно определить, лежит ли точка~$P$ внутри квадрата.
Можно ли это наверняка узнать
\\
\subproblem за два вопроса?
\qquad
\subproblem за три вопроса?

\item
Путешественник посетил деревню, каждый житель которой либо всегда говорит
правду, либо всегда лжет.
Все жители деревни встали в круг лицом к центру, и каждый сказал
путешественнику про соседа справа, правдив ли тот.
На основании этих сообщений путешественник смог однозначно определить, какую
долю от всех жителей составляют лжецы.
Определите и вы, чему она равна.

\item
Суду предъявлен набор из 100 одинаковых с виду монет.
Суд знает, что все настоящие монеты весят одинаково, фальшивые~--- тоже
одинаково, но легче настоящих.
Адвокат знает, какие монеты на самом деле фальшивые.
Задача адвоката: показать суду, сколько есть фальшивых монет, не разгласив
ни про какую монету, фальшивая она или настоящая.
(Адвокат должен делать взвешивания на чашечных весах без гирь.
Число взвешиваний не ограничено.
Запрещены взвешивания и группы взвешиваний, из которых логически выводится, что
конкретная монета фальшивая или настоящая.)
\\
\subproblem
Суд уже установил, что фальшивых монет 3 или 4.
Как адвокату показать, что их ровно 4?
\\
\subproblem
Суд уже установил, что фальшивых монет 0 или 4.
Как адвокату показать, что их ровно 4?

\end{problems}

