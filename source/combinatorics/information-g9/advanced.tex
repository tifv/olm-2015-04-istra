% $date: 2015-04-07
% $timetable:
%   g9r1:
%     2015-04-07: {}

% $matter[-contained,no-header]:
% - verbatim: \section*{Неоднозначные данные}
% - verbatim: \setproblem{3}
% - .[contained]

\subsection*{Дополнительные задачи}

% $authors:
% - Александр Шаповалов

% $matter[-no-ashap-link,contained,no-header]:
% - .[no-ashap-link]
% - /_ashap-link

\begin{problems}

\item
Суду предъявлен набор из 100 одинаковых с виду монет.
Суд знает, что все настоящие монеты весят одинаково, фальшивые~--- тоже
одинаково, но легче настоящих.
Адвокат знает, какие монеты на самом деле фальшивые.
Задача адвоката: показать суду, сколько есть фальшивых монет, не разгласив
ни про какую монету, фальшивая она или настоящая.
(Адвокат должен делать взвешивания на чашечных весах без гирь.
Число взвешиваний не ограничено.
Запрещены взвешивания и группы взвешиваний, из которых логически выводится, что
конкретная монета фальшивая или настоящая.)
\\
\subproblem
Суд уже установил, что фальшивых монет 2 или 3.
Как адвокату показать, что их ровно 3?
\\
\subproblem
Суд уже установил, что фальшивых монет 2, 3 или 4.
Как адвокату показать, что их ровно 3?
\\
\subproblem
Суду о числе фальшивых монет ничего не известно.
Как адвокату показать, что их ровно 10?

\item
Из колоды вынули 7~карт, показали всем, перетасовали, и раздали Грише и Леше
по 3 карты, а оставшуюся карту
\\
\subproblem спрятали;
\quad
\subproblem отдали Коле.
\\
Игроки могут по очереди сообщать вслух открытым текстом любую информацию
о своих картах.
Могут ли сообщить друг другу свои карты так, чтобы при этом Коля не смог
вычислить местонахождение ни одной из карт (кроме той, что у него)?

\end{problems}

