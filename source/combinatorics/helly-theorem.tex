% $timetable:
%   g11r1: {}

\section*{Вокруг теоремы Хелли}

% $authors:
% - Алексей Канель

\begin{problems}

\item\claim{Теорема Хелли}
\sp
На~плоскости расположены выпуклые фигуры, любые~3 имеют общую точку.
Докажите, что все они имеют общую точку.
\\
\sp
В~$n$-мерном пространстве расположены выпуклые тела, любые $n + 1$ имеют общую
точку.
Докажите, что все они имеют общую точку.

\item
Стороны нескольких прямоугольников параллельны осям координат.
Любые два из~ них имеют общую точку.
Докажите, что все прямоугольники имеют общую точку.

\item\emph{(Задача на~исследование)}
На~столе лежат круглые салфетки.
Любые две пересекаются.
Докажите, что их~можно прибить 100 гвоздями.
Можно~ли уменьшить число 100?
А~если эти салфетки суть единичные круги?
Если они есть выпуклые фигуры, отличающиеся параллельным переносом?

\item\claim{Теорема Красносельского}
Дан (возможно, невыпуклый) многоугольник~$M$.
Известно, что для любых трех его сторон можно указать точку внутри $M$
из~которой они видны полностью.
Тогда существует точка внутри $M$ из~которой видны полностью все стороны $M$.

\item
На~плоскости дано несколько точек.
Известно, что любые~3 можно покрыть кругом единичного радиуса.
Тогда все точки можно покрыть кругом единичного радиуса.

\item\claim{Теорема Юнга}
На~плоскости дано несколько точек, расстояние между любыми двумя
не~превосходит~1.
Тогда все точки можно покрыть кругом радиуса $1 / \sqrt{3}$.

\item\claim{Теорема Юнга для трехмерья}
В~пространстве дано несколько точек, расстояние между любыми двумя
не~превосходит~1.
Тогда все можно покрыть шаром радиуса $\sqrt{6} / 4$.

\item
Любая фигура диаметра~1 заключается в~круг радиуса $1 / \sqrt{3}$.

\item\claim{Теорема Бляшке}
Любая выпуклая фигура ширины~1 содержит круг радиуса $1 / (2 \sqrt{3})$.

\item\claim{Теорема Бляшке для трехмерья}
Любое выпуклое тело ширины~1 содержит шар радиуса $1 / (2 \sqrt{3})$.

\item
На~плоскости отмечено $n$~точек общего положения.
Докажите, что существует точка~$O$ такая, что по~каждую сторону любой прямой,
проходящей через $O$ находится не~менее $n / 3$ отмеченных точек.

\item
На~плоскости дана ограниченная кривая~$K$ длины~$L$ без прямолинейных участков.
Докажите, что существует точка~$O$ такая, что любая прямая, проходящая
через $O$, рассекает $K$ на~две части, длина каждой не~менее $L / 3$.

\item
На~плоскости дана ограниченная фигура~$\Phi$ площади~$S$.
Докажите, что существует точка~$O$ такая, что любая прямая, проходящая
через $O$, рассекает $\Phi$ на~две части, площадь каждой не~менее $S / 3$.

\item
Докажите, что внутри каждой ограниченной выпуклой фигуры~$\Phi$ найдется
точка~$O$ такая, что всякая хорда $AB$ фигуры~$\Phi$, проходящая через $O$,
разбивается этой точкой на~отрезки $AO$ и~$BO$, длина каждого из~которых
не~менее $\lvert AB \rvert / 3$.

\item
Сформулируйте и~докажите аналоги предыдущих 4~задач для $n$-мерного
пространства.
Получите неулучшаемые оценки.

\end{problems}

