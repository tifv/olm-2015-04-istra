% $date: 2015-04-03
% $timetable:
%   g9r1:
%     2015-04-03:
%       3:

\section*{Конечное и бесконечное}

% $authors:
% - Александр Шаповалов

% $matter[-no-ashap-link,no-header]:
% - .[no-ashap-link]
% - /_ashap-link

\subsection*{Дирихле на бесконечности}

Если бесконечную кучу разделить на~конечное число частей, хотя~бы одна
из~частей должна быть бесконечной.

\begin{problems}

\item
Докажите, что есть бесконечно много простых чисел, дающих при делении на~2015
одинаковый остаток.

\item
Докажите, что среди цифр в~десятичной записи $\sqrt{2}$ есть две цифры, каждая
из~которых встречается бесконечно много раз.

\item
Круг разделен на~2015 секторов, и~в~каждом написано целое число.
В~один из~секторов ставится фишка.
Каждым ходом прочитывается число в~секторе, где стоит фишка
(пусть прочитано $k$), фишка сдвигается на~$\lvert k \rvert$ секторов по~часовой стрелке,
и~там, куда она придет, число увеличивается на~1.
Докажите, что со~временем все числа станут больше миллиона.

\end{problems}

\subsection*{Капля камень точит}

\claim{Принцип Архимеда}
Складывая само с~собой любое положительное число (даже очень малое), можно
превзойти любое число (даже очень большое).

\begin{problems}

\item
\subproblem
Может~ли сумма бесконечного числа положительных слагаемых быть конечным числом?
\\
\subproblem
Докажите, что среди сумм вида $1 + 1 / 2 + 1 / 3 + \ldots + 1 / n$ есть сколь
угодно большие.

\item
Два зеркала бесконечной длины образуют угол.
Луч света падает на~один из~них.
Докажите, что луч света отразится от~зеркал конечное число раз (даже если угол
очень маленький).

\item
Трактор равномерно растягивает резиновый жгут длиной $10\,\text{м}$
со~скоростью $1\,\text{м/мин}$.
По~жгуту с~одного конца ползет улитка со~скоростью $1\,\text{cм/мин}$.
Может~ли она со~временем доползти до~другого конца жгута?

\end{problems}

\subsection*{Сколько бы ни было — хочется еще}

Следует различать «сколь угодно большое» и~«бесконечное».

\begin{problems}

\item
\subproblem
Докажите, что есть сколь угодно длинная группа подряд идущих составных чисел.
\\
\subproblem
Найдется~ли бесконечная группа подряд идущих составных чисел?

\item
\subproblem
Продлим шахматную доску вправо и~влево на~миллион клеток.
Король стоит на~средней клетке нижней горизонтали.
Может~ли он~обойти всю доску, побывав на~каждой клетке ровно один раз?
\\
\subproblem
Тот~же вопрос, если доску продлили вправо и~влево до~бесконечности?

\end{problems}

\subsection*{Ковшом моря не вычерпаешь}

Если покрыто целое, то~покрыта его любая часть.

\begin{problems}

\item
Можно~ли покрыть прямую конечным числом кругов?

\item
На~плоскости отметили миллион точек, и~через каждые две провели прямую.
Докажите, что можно провести еще одну прямую так, чтобы угол между ней
и~любой ранее проведенной выражался нецелым числом градусов.

\item
Полоса~— это часть плоскости между двумя параллельными прямыми.
Можно~ли покрыть плоскость конечным числом полос?

\item
Можно~ли покрыть плоскость конечным числом внутренностей парабол?

\end{problems}

