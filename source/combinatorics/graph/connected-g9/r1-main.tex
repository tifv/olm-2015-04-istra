% $date: 2015-04-06
% $timetable:
%   g9r1:
%     2015-04-06:
%       1:

\section*{Свяжитесь с графом}

% $authors:
% - Александр Шаповалов

% $matter[-no-ashap-link,no-header]:
% - .[no-ashap-link]
% - /_ashap-link

% spell \text{В} \text{Р} \text{С}

Считаем ребра, вершины и компоненты без циклов.
Обозначим в графе \text{В}~--- число вершин, \text{Р}~--- число ребер,
\text{С}~--- число компонент связности.

\setcounter{jeolmsubproblem}{0}
\claim{Факт 1}
\\
\subproblem
В дереве (то есть связном графе без циклов) $\text{В} = \text{Р} + 1$.
\\
\subproblem
В графе без циклов $\text{В} = \text{Р} + \text{С}$.

\setcounter{jeolmsubproblem}{0}
\claim{Факт 2}
\\
\subproblem
В связном графе $\text{Р} \geq \text{В} - 1$.
\\
\subproblem
В любом графе $\text{Р} \geq \text{В} - \text{С}$.

\begin{problems}

\item
Какое наибольшее число ребер можно перекусить в проволочном каркасе додекаэдра
так, чтобы каркас не развалился на части?

\item
Пусть дан связный граф с $n$~вершинами и $k$~ребрами, причем $k > n - 1$.
Докажите, что можно удалить ребро так, чтобы граф остался связным.

\item
В многоугольнике проведены все диагонали из одной вершины.
Можно ли стороны и проведённые диагонали раскрасить в желтый и красный цвета
так, чтобы жук мог проползти из любой вершины в любую другую по желтым
отрезкам, а клоп~--- по красным?

\item
Из спичек сложена шахматная доска.
Жук через спичку не ползает.
Убрав часть спичек внутри доски, получаем \emph{лабиринт.}
Назовем его \emph{связным,} если жук может проползти между любыми двумя клетками.
Каких лабиринтов можно получить больше: связных или не связных?

\end{problems}

Увидеть граф за условием задачи помогают \emph{выделенные} пары объектов,
в частности, соседние объекты или клетки с общей границей.
Выписывая для таких графов уравнения и неравенства для
\text{В}, \text{Р}, \text{С}, можно получать нетривиальные оценки.

\begin{problems}

\item
Есть $m$ болельщиков: некоторые из них (возможно, все или никто) болеют за
<<Спартак>>, а остальные~--- за <<Динамо>>.
Разрешается спросить у любых двоих, болеют ли они за разные команды, и они
честно ответят <<да>> или <<нет>>.
Требуется посадить болельщиков в два автобуса так, чтобы в каждом были
болельщики только одной команды.
За какое минимальное количество вопросов это наверняка можно сделать?

\item
Тетрадный лист раскрасили в 23 цвета по клеткам (при этом все цвета
присутствуют).
Пара цветов называется \emph{хорошей,} если найдутся две соседние клетки,
закрашенные этими цветами.
Каково минимальное число хороших пар?

\item
В классе 30 человек.
За месяц было 29 дежурств, в каждом дежурила пара учеников.
Докажите, что можно так выставить всем ученикам класса по одной оценке
по 5-балльной шкале, что будет выставлена хотя бы одна пятерка, и в каждой паре
дежуривших сумма оценок будет равна 8.

\item
На клетчатой бумаге нарисован многоугольник площадью в $n$~клеток.
Его контур идет по линиям сетки.
Каков наибольший периметр многоугольника?
(Сторона клетки равна 1).

\item
Дан клетчатый прямоугольник $m \times n$.
Каждую его клетку разрезали по одной из диагоналей.
На какое наименьшее число частей мог распасться прямоугольник?

\item
Есть 100 камней разного веса.
За одно взвешивание можно про любые два камня узнать, который из них тяжелее.
За какое наименьшее число взвешиваний можно наверняка определить самый тяжелый
камень?

\end{problems}

