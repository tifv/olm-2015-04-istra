% $date: 2015-04-06
% $timetable:
%   g9r1:
%     2015-04-06: {}

% $matter[-contained,no-header]:
% - verbatim: \section*{Свяжитесь с графом}
% - .[contained]

\subsection*{Домашнее задание}

% $authors:
% - Александр Шаповалов

% $matter[-no-ashap-link,contained,no-header]:
% - .[no-ashap-link]
% - /_ashap-link

\begin{problems}

\item
Куб со стороной $n \geq 3$ разбит перегородками на единичные кубики.
Какое минимальное число перегородок между кубиками нужно удалить, чтобы
из каждого кубика можно было добраться до границы куба?

\item
На клетчатой бумаге по границам клеток обведен тысячеугольник.
Из какого наименьшего числа клеток он может состоять?

\item
Какое наибольшее число клеток доски $9 \times 9$ можно разрезать по обеим
диагоналям, чтобы при этом доска не распалась на несколько частей?

\item
Есть 101 банка консервов весами
$1001\,\text{г}$, $1002\,\text{г}$, \ldots, $1101\,\text{г}$.
Этикетки с весами потерялись, но завхозу кажется, что он помнит, какая банка
сколько весит.
Он хочет убедиться в этом за наименьшее число взвешиваний.
\\
\sp
У завхоза есть двое чашечных весов: одни точные, другие~--- грубые.
За одно взвешивание можно сравнить две банки.
Точные весы всегда показывают, какая банка тяжелее, а грубые~--- только если
разница больше $1{,}1\,\text{г}$ (а иначе показывают равновесие).
Завхоз может использовать только одни весы.
Какие ему следует выбрать?
\\
\sp
У завхоза есть только грубые весы.
Какое наименьшее число взвешиваний ему понадобится?

\end{problems}

