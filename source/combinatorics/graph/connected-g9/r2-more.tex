% $date: 2015-04-06
% $timetable:
%   g9r2:
%     2015-04-06: {}

\section*{Свяжитесь с графом: на дом}

% $authors:
% - Александр Шаповалов

% $matter[-no-ashap-link,no-header]:
% - .[no-ashap-link]
% - /_ashap-link

\begin{problems}

\itemy{СГ1}
Куб со стороной $n \geq 3$ разбит перегородками на единичные кубики.
Какое минимальное число перегородок между кубиками нужно удалить, чтобы
из каждого кубика можно было добраться до границы куба?

\itemy{СГ2}
На клетчатой бумаге по границам клеток обведен тысячеугольник.
Из какого наименьшего числа клеток он может состоять?

\itemy{СГ3}
Какое наибольшее число клеток доски $9 \times 9$ можно разрезать по обеим
диагоналям, чтобы при этом доска не распалась на несколько частей?

\itemy{СГ4}
Есть 100 камней разного веса.
За одно взвешивание можно про любые два камня узнать, который из них тяжелее.
За какое наименьшее число взвешиваний можно наверняка определить самый тяжёлый
камень?

\end{problems}

