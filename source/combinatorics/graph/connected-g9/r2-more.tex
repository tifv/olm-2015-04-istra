% $date: 2015-04-06
% $timetable:
%   g9r2:
%     2015-04-06: {}

% $matter[-contained,no-header]:
% - verbatim: \section*{Свяжитесь с графом}
% - .[contained]

\subsection*{Домашнее задание}

% $authors:
% - Александр Шаповалов

% $matter[-no-ashap-link,contained,no-header]:
% - .[no-ashap-link]
% - /_ashap-link

\begin{problems}

\item
Куб со стороной $n \geq 3$ разбит перегородками на единичные кубики.
Какое минимальное число перегородок между кубиками нужно удалить, чтобы
из каждого кубика можно было добраться до границы куба?

\item
На клетчатой бумаге по границам клеток обведен тысячеугольник.
Из какого наименьшего числа клеток он может состоять?

\item
Какое наибольшее число клеток доски $9 \times 9$ можно разрезать по обеим
диагоналям, чтобы при этом доска не распалась на несколько частей?

\item
Есть 100 камней разного веса.
За одно взвешивание можно про любые два камня узнать, который из них тяжелее.
За какое наименьшее число взвешиваний можно наверняка определить самый тяжёлый
камень?

\end{problems}

