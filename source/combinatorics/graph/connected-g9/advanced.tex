% $date: 2015-04-06
% $timetable:
%   g9r1:
%     2015-04-06: {}

% $matter[-contained,no-header]:
% - verbatim: \section*{Свяжитесь с графом}
% - verbatim: \setproblem{4}
% - .[contained]

\subsection*{Дополнительные задачи}

% $authors:
% - Александр Шаповалов

% $matter[-no-ashap-link,contained,no-header]:
% - .[no-ashap-link]
% - /_ashap-link

\begin{problems}

\item
Из спичек выложена доска $8 \times 8$ так, что каждую клетку ограничивают
четыре спички.
Какое наименьшее число спичек можно убрать, чтобы после этого не осталось
ни одного контура прямоугольника?

\item
Есть $2 n$ болельщиков: половина из них болеют за <<Спартак>>, а остальные~---
за <<Динамо>>.
Разрешается спросить у любых двоих, болеют ли они за разные команды, и они
честно ответят <<да>> или <<нет>>.
Требуется посадить болельщиков в два автобуса так, чтобы в каждом были
болельщики только одной команды.
За какое минимальное количество вопросов это наверняка можно сделать?

\item
Дана доска $m \times n$, разбитая на единичные клетки.
Сначала в $(m - 1) \cdot (n - 1) + 1$ клеток ставится по фишке.
Назовем \emph{квартетом} четверку клеток
\\
\subproblem в квадратике $2 \times 2$;
\qquad
\subproblem в вершинах прямоугольника со сторонами, параллельными краям доски.
\\
Если в квартете есть ровно одна фишка, ее разрешается снять.
Докажите, что разрешенными операциями нельзя снять все фишки.

\item
Хозяйка испекла для гостей пирог.
За столом может оказаться либо $p$ человек, либо $q$
($p$ и $q$ взаимно просты).
На какое минимальное количество кусков (не обязательно равных) нужно заранее
разрезать пирог, чтобы в любом случае его можно было раздать поровну?

\end{problems}

