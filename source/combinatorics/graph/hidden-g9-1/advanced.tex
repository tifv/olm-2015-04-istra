% $date: 2015-04-05
% $timetable:
%   g9r1:
%     2015-04-05: {}

\section*{Увидеть граф --- 1: для крутых}

% $authors:
% - Александр Шаповалов

% $matter[-no-ashap-link,no-header]:
% - .[no-ashap-link]
% - /_ashap-link

% $build$matter[print]: [[.], [.]]
% $build$style[print]:
% - .[tiled4,-print]
%% - verbatim: \parskip=0.8ex
%% - scale_font: [9.8, 11.76]

\begin{problems}

\itemy{УГ5}
Поверхность куба $11 \times 11 \times 11$ разбита на~клетки $1 \times 1$.
Муравей бегает по~диагоналям клеток, нигде не~поворачивая назад.
Он~не~может бывать внутри одной клетки более одного раза, но~может несколько
раз проходить одну вершину.
Может~ли он~посетить центры всех клеток?

\itemy{УГ6}
Имеется несколько юношей, каждый из~которых знаком с~некоторыми девушками.
Две свахи знают, кто с~кем знаком.
Одна сваха заявляет:
<<Я могу одновременно женить всех брюнетов так, чтобы каждый из~них женился
на~знакомой ему девушке!>>.
Вторая сваха говорит:
<<А~я могу устроить судьбу всех блондинок: каждая выйдет замуж за~знакомого
юношу!>>.
Этот диалог услышал любитель математики, который сказал:
<<В~таком случае можно сделать и~то, и~другое!>>.
Прав~ли он?

\itemy{УГ7}
Гриша записал в~клетки шахматной доски числа $1, 2, \ldots, 64$ в~неизвестном
порядке.
Он~сообщил Леше сумму чисел в~каждом прямоугольнике из~двух клеток и~добавил,
что 1 и~64 лежат на~одной диагонали.
Докажите, что по~этой информации Леша может точно определить, где какое число
записано.

\itemy{УГ8}
Есть много одинаковых бумажных квадратов, в~которых проведены обе диагонали.
Квадратами оклеили куб $100 \times 100 \times 100$ без щелей и~наложений
(при этом, возможно, перегибая квадраты через ребра и~располагая косо).
Жучка посадили в~одну из~вершин куба и~разрешили ползать только по~диагоналям
квадратов.
Докажите, что он~может добраться в~какую-то другую вершину куба.

\end{problems}

