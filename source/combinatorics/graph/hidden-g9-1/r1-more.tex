% $date: 2015-04-05
% $timetable:
%   g9r1:
%     2015-04-05: {}

% $matter[-contained,no-header]:
% - verbatim: \section*{Увидеть граф --- 1}
% - .[contained]

\subsection*{Домашнее задание}

% $authors:
% - Александр Шаповалов

% $matter[-no-ashap-link,contained,no-header]:
% - .[no-ashap-link]
% - /_ashap-link

\begin{problems}

\item
В~клетки доски $8 \times 8$ записали числа $1, 2, \ldots, 64$ в~неизвестном
порядке.
Разрешается узнать сумму чисел в~любой паре клеток, связанных ходом коня.
Всегда~ли можно узнать расположение всех чисел?

\item
Можно~ли расставить в~вершинах пятиугольной призмы натуральные числа так, чтобы
в~каждой паре чисел, связанных ребром, одно из~них делилось на~другое,
а~во~всех других парах такого не~было?

\item
10 кружковцев образовали дежурную команду для решения домашних задач.
В~команде всегда не~менее 3~человек.
Каждый вечер в~команду добавляется один человек либо из~нее исключается один
человек.
Можно~ли будет перебрать все допустимые составы команды ровно по~одному разу?

\item
На~свободные поля шахматной доски по~одному выставляются короли.
Первый выставляется произвольно, каждый следующий должен бить нечетное число
ранее выставленных.
Какое наибольшее число королей можно выставить?

\end{problems}

