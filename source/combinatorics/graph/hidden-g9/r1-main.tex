% $date: 2015-04-05
% $timetable:
%   g9r1:
%     2015-04-05:
%       1:

\section*{Увидеть граф}

% $authors:
% - Александр Шаповалов

% $matter[-no-ashap-link,no-header]:
% - .[no-ashap-link]
% - /_ashap-link

\subsection*{Позиции и ходы}

\begin{problems}

\item
\sp
На~прямой сидят два кузнечика.
Каждую минуту один из~кузнечиков перепрыгивает ровно через одного другого.
Могут~ли все кузнечики оказаться на~своих местах ровно через 777 прыжков?
\\
\sp
То~же, но~кузнечиков не~два, а~три.

\item
На~шахматной доске $5 \times 5$ расставили максимальное число коней так, чтобы
они не~били друг друга.
Докажите, что такая расстановка~--- единственная.

\item
\sp
На~шахматной доске стоят две одинаковых фишки.
За~один ход можно сдвинуть одну из~фишек на~соседнее поле по~вертикали или
горизонтали.
Могут~ли фишки перейти в~симметричную относительно средней линии позицию ровно
за~2015 ходов?
\\
\sp
На~шахматной доске стоят пять одинаковых фишек.
За~один ход можно сдвинуть одну из~фишек на~соседнее поле по~вертикали или
горизонтали.
Могут~ли фишки перейти в~центрально симметричную позицию ровно за~2015 ходов?

\end{problems}

\subsection*{Сумма степеней вершин}

\claim{Факт 1}
Сумма степеней вершин графа вдвое больше числа его ребер.

\claim{Факт 2}
В~конечном графе число вершин нечетной степени четно. 

\begin{problems}

\item
В~классе учится 10 мальчиков и~15 девочек.
Они отправили друг другу смски.
Может~ли у~каждого число отправленных смсок отличается на~1 от~числа
полученных?

\item
В~пустые клетки доски $5 \times 5$ Петя по~одному вписывал числа.
Вписанное число равнялось количеству соседних по~стороне клеток, в~которые уже
был вписаны числа.
Петя заполнил всю доску.
Найдите сумму все чисел и~докажите, что она не~зависит от~порядка заполнения.

\item
В~однокруговом турнире участвовали 15 команд.
Докажите, что хотя~бы в~одной игре встретились команды, которые перед этой
игрой участвовали в~сумме в~нечетном числе игр этого турнира.

\end{problems}

\subsection*{Чередование и обходы}

\definition
Граф~--- \emph{двудольный,} если его вершины можно раскрасить в~два цвета так,
что не~будет ребер с~концами одинакового цвета.
Пример: любое дерево.

\begin{problems}

\item
Докажите, что следующие графы~--- двудольные:
\\
\sp
Вершины графа~--- расстановка пары фишек на~шахматной доске.
Две расстановки связаны ребром, если позиции получаются друг из~друга ходом
фишки на~одну клетку по~вертикали или горизонтали.
\\
\sp
То~же, что в~предыдущем пункте, но~для $n$ фишек.
\\
\sp
Вершины~--- перестановки из~$n$ чисел $1, 2, \ldots, n$, ребра~---
расположения, получающиеся друг из~друга перестановкой двух соседних чисел.
\\
\sp
Вершины~--- перестановки из~$n$ чисел $1, 2, \ldots, n$, ребра~---
расположения, получающиеся друг из~друга перестановкой двух любых чисел.

\item
Пусть $\Gamma$~--- двудольный граф с~черными и~белыми вершинами.
Докажите, что
\\
\sp
Если в~$\Gamma$ есть замкнутый цикл, проходящий через каждую вершину ровно
по~одному разу, то~вершин каждого цвета~--- поровну.
\\
\sp
Если в~$\Gamma$ есть путь, проходящий через каждую вершину ровно по~одному
разу, то~число белых вершин отличается от~числа черных вершин не~более чем
на~1.

\item
Замок в~форме треугольника со~стороной 50 метров разбит на~100 треугольных
залов со~сторонами $5\,\text{м}$.
В~каждой стенке между залами есть дверь.
Какое наибольшее число залов сможет обойти турист, не~заходя ни~в~какой зал
дважды?

\item
Для игры в~классики на~земле нарисован ряд клеток, в~которые вписаны по~порядку
числа от~1 до~10, как на~рисунке.
\begin{center}
\jeolmfigure{../jumping-ground}
\end{center}
Маша прыгнула снаружи в~клетку~1, затем попрыгала по~остальным клеткам
(каждый прыжок~--- на~соседнюю по~стороне клетку) и~выпрыгнула наружу
из~клетки~10.
Известно, что на~клетке~1 Маша была один~раз, на~клетке~2~--- два~раза, \ldots,
на~клетке~9~--- девять~раз.
Сколько раз побывала Маша на~клетке~10?

\item
\sp
Отмечены вершины и~центры граней куба и~проведены диагонали всех граней.
Можно~ли по~отрезкам этих диагоналей обойти все отмеченные точки, побывав
в~каждой из~них ровно по~одному разу?
\\
\sp
В~кубике Рубика $3 \times 3 \times 3$ отмечены вершины клеток, середины сторон
клеток и~центры клеток.
Центры клеток соединены отрезками с~серединами сторон клеток.
Можно~ли по~проведенным отрезкам и~сторонам клеток обойти все отмеченные точки,
побывав в~каждой из~них ровно по~одному разу?

\end{problems}

