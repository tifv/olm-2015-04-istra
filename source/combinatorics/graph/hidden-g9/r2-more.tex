% $date: 2015-04-05
% $timetable:
%   g9r2:
%     2015-04-05: {}

% $matter[-contained,no-header]:
% - verbatim: \section*{Увидеть граф}
% - .[contained]

\subsection*{Домашнее задание}

% $authors:
% - Александр Шаповалов

% $matter[-no-ashap-link,contained,no-header]:
% - .[no-ashap-link]
% - /_ashap-link

\begin{problems}

\item
На~шахматной доске $5 \times 5$ расставили максимальное число коней так, чтобы
они не~били друг друга.
Докажите, что такая расстановка~--- единственная.

\item
На~шахматной доске стоят две одинаковых фишки.
За~один ход можно сдвинуть одну из~фишек на~соседнее поле по~вертикали или
горизонтали.
Так ходили, пока не~прошли через все возможные позиции.
Докажите, что какая-то позиция встретилась не~менее двух раз.

\item
\subproblem
Отмечены вершины и~центры граней куба и~проведены диагонали всех граней.
Можно~ли по~отрезкам этих диагоналей обойти все отмеченные точки, побывав
в~каждой из~них ровно по~одному разу?
\\
\subproblem
В~кубике Рубика $3 \times 3 \times 3$ отмечены вершины клеток, середины сторон
клеток и~центры клеток.
Центры клеток соединены отрезками с~серединами сторон клеток.
Можно~ли по~проведенным отрезкам и~сторонам клеток обойти все отмеченные точки,
побывав в~каждой из~них ровно по~одному разу?

% spell кружковцы кружковцев

\item
10 кружковцев образовали дежурную команду для решения домашних задач.
В~команде всегда не~менее 3~человек.
Каждый вечер в~команду добавляется один человек либо из~нее исключается один
человек.
Можно~ли будет перебрать все допустимые составы команды ровно по~одному разу?

\end{problems}

