% $date: 2015-03-30
% $timetable:
%   g11r1:
%     2015-03-30:
%       3:

\section*{Графы}

% $authors:
% - Олег Орлов

\begin{problems}

\item
В~графе $2 n$~вершин, причем степень каждой вершины четна.
Докажите, что существуют такие две вершины, что число их~общих соседей четно.
% Рассмотрим одну вершину и смежные с ней.

\item
Хроматическое число графа~$G$ равно $k$.
Известно, что существует такая правильная раскраска~$G$ такая, что вершин
каждого цвета хотя~бы $2$.
Докажите, что существует такая правильная раскраска в~$k$~цветов, что вершин
каждого цвета хотя~бы две.
(Правильной раскраской вершин графа называется такая раскраска, что никакие две
вершины одного цвета не~смежны.
Хроматическим числом графа называется такое минимальное натуральное~$k$, что
существует правильная раскраска вершин в~$k$~цветов).
% Перекрашивание.

\item
Вершины графа~$G$ можно единственным образом разбить на~$5$~групп так, что
никакие две вершины из~одной группы не~смежны (группы могут быть пустыми).
Пусть $n$~--- число вершин в~$G$.
Докажите, что ребер в~графе $G$ не~меньше, чем $4 n - 10$.
% Двудольный граф, где доли --- группы, должен быть связным.

\item
В~графе $3333$ вершины, и~для любых двух его вершин существует гамильтонов путь
(то~есть путь, проходящий через каждую из~вершин графа ровно один раз)
с~концами в~этих вершинах.
Какое наименьшее число ребер может быть у~такого графа?
% Степень каждой вершины не меньше 3. Ответ 5000.

\item
Ребра графа раскрашены в~$k$~цветов.
Известно, что любые два ребра одного цвета имеют общую вершину.
Докажите, что вершины можно разбить на~$k + 2$ группы так, чтобы никакие две
вершины из~одной группы не~были смежны.
% Индукция. В случае треугольников можно считать количество ребер между
% группами, но вроде можно тоже индукцией.

\item
В~связном графе $1000$ вершин, причем степень каждой равна трем.
Докажите, что можно удалить такие $101$ попарно не~смежную вершину, вместе
с~выходящими из~них ребрами, чтобы граф остался связным.
% Если циклы пересекаются по вершине О, то О можно удалить. Так и действуем,
% пока можем.

\item
В~графе на~$n$~вершинах нет двух циклов, пересекающихся ровно по~одному ребру.
Найдите максимальное возможное количество ребер в~нем.

\itemx{*}
Ребра полного графа раскрашены в~три цвета, причем между любыми двумя вершинами
есть пути каждого цвета.
Докажите, что найдется треугольник, ребра которого разноцветны.

\end{problems}

