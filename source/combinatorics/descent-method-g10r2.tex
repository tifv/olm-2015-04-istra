% $date: 2015-04-03
% $timetable:
%   g10r2:
%     2015-04-03:
%       1:

\section*{Метод спуска}

% $authors:
% - Николай Крохмаль

\begin{problems}

\item
Решите в~целых числах уравнение
\begin{equation*}
    x^3 - 2 y^3 - 4 z^3 = 0
\end{equation*}

\item
В~последовательности троек целых чисел $(2, 3, 5)$, $(6, 15, 10)$, \ldots
каждая тройка получается из~предыдущей таким образом: первое число умножается
на~второе, второе~--- на~третье, а~третье~--- на~первое, и~полученные
произведения дают новую тройку.
Докажите, что ни~одно из~чисел, получаемых таким образом, не~будет степенью
целого числа: квадратом, кубом и~т.~д.

\item
В~парламенте у~каждого не~более трех врагов.
Докажите, что парламент можно разделить на~две палаты так, что у~каждого
парламентария в~своей палате будет не~более одного врага.

\item
Плоскость разбита тремя сериями параллельных прямых на~равные между собой
равносторонние треугольники.
Существуют~ли четыре вершины этих треугольников, образующие квадрат?

\item
В~клетки таблицы $n \times m$ вписаны некоторые числа.
Разрешается одновременно менять знак у~всех чисел одного столбца или одной
строки.
Докажите, что несколькими такими операциями можно добиться того, чтобы суммы
чисел, стоящих в~любой строке и~в~любом столбце, были неотрицательны.

\item
На~плоскости дано $N$~точек.
Некоторые точки соединены отрезками.
Если два отрезка пересекаются, то~их~можно заменить двумя другими с~концами
в~тех~же точках.
Может~ли этот процесс продолжаться бесконечно?

\item
В~колоде $N$~карт.
Часть из~них лежит рубашками вверх, остальные рубашками вниз.
За~один ход разрешается взять несколько карт сверху, перевернуть полученную
стопку и~снова положить ее~сверху колоды.
За~какое наименьшее число ходов при любом начальном расположении карт можно
добиться того, чтобы все карты лежали рубашками вниз?

\item
В~стране N $1998$ городов и~из~каждого осуществляются беспосадочные
перелеты в~три других города.
Известно, что из~любого города, сделав несколько пересадок, можно долететь
до~любого другого.
Министерство Безопасности хочет объявить закрытыми $200$ городов, никакие два
из~которых не~соединены авиалинией.
Докажите, что это можно сделать так, чтобы можно было долететь из~любого
незакрытого города в~любой другой, не~делая пересадок в~закрытых городах.

\end{problems}

