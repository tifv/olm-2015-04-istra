% $date: 2015-04-09
% $timetable:
%   g10r2:
%     2015-04-09:
%       2:

\section*{Комбинаторная геометрия}

% $authors:
% - Николай Крохмаль

\begin{problems}

\item
Докажите, что замкнутая ломаная длины~$1$, расположенная на~плоскости, может
быть накрыта кругом радиуса $\frac{1}{4}$.

\item
Единичный квадрат покрыт $n$~фигурами так, что каждая его точка принадлежит
по~крайней мере $q$~фигурам.
Докажите, что хотя~бы одна фигура имеет площадь, не~меньшую $q / n$.

\item
Треугольник разрезан на~несколько выпуклых многоугольников.
Докажите, что среди них либо есть треугольник, либо найдутся два многоугольника
с~одинаковым числом сторон.

\item
Из~$106$ точек, никакие три из~которых не~лежат на~одной прямой, четыре
являются вершинами единичного квадрата, а~остальные лежат внутри этого
квадрата.
Докажите, что имеется по~крайней мере $107$ треугольников с~вершинами в~этих
точках, имеющих площадь не~более $0{,}01$.

\item
На~плоскости даны $2 n + 3$ точки общего положения, никакие четыре из~которых
не~лежат на~одной окружности.
Докажите, что через три из~них можно провести окружность, внутри которой лежит
ровно $n$~точек из~данных.

\item
Докажите, что в~выпуклом многоугольнике с~четным числом сторон есть диагональ,
не~параллельная ни~одной из~его сторон.

\item
Каким наименьшим числом кругов радиуса $1$ можно покрыть круг радиуса~$1{,}5$?

\item
Каждая сторона правильного треугольника разбита на~$30$ равных частей.
Прямые, проведенные через точки деления параллельно сторонам треугольников,
разбивают его на~$900$ маленьких треугольников.
Каково максимальное количество вершин разбиения, никакие две из~которых
не~лежат на~проведенной прямой или стороне?

\end{problems}

