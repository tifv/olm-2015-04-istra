% $date: 2015-03-31
% $timetable:
%   g11r1:
%     2015-03-31:
%       1:

\section*{Комбинаторика-1}

% $authors:
% - Олег Орлов

\begin{problems}

\item
В~клетках доски $100 \times 100$ в~некотором порядке написаны числа
$1, 2, \ldots, 10000$ (в~каждой клетке по~числу).
Докажите, что найдутся соседние по~стороне клетки, в~которых написаны числа,
отличающиеся не~менее, чем на~$100$.
% Bollobas coffee time. Задача 21.
% решения:
% - индукция
% - рассмотрим ряд, в котором максимальное число минимально по всем рядам

\item
Дана таблица $n \times n$.
В~ней стоит $(n - 1)$ единица, остальные нули.
За~ход разрешается выбрать клетку, уменьшить значение в~ней на~$1$
и~одновременно увеличить на~$1$ значение во~всех клетках, находящихся с~ней
в~одном столбце и~одной строке.
Можно~ли все числа сделать равными?
\\
\subproblem $n = 2$.
\qquad
\subproblem $n$ произвольное.

\item
В~каждой клетке доски $2000 \times 2002$ находится по~лампочке.
В~начале, на~доске горит больше чем $1999 \cdot 2001$ лампочек.
За~одну операцию можно взять квадратик $2 \times 2$, в~котором выключено ровно
3 лампочки и~выключить четвертую в~этом квадратике (конечно только если такой
найдется).
Докажите, что такими операциями нельзя выключить все лампочки на~доске.
% 102 c.p. (usa-imo-training) adv.prob.13
% решения:
% - граф
% - общее количество квадратов 2 ⨯ 2

\item
Пусть дано множество $\Omega = \{ \pm a_1, \ldots, \pm a_{10} \}$, где
$a_1, \ldots, a_{10}$~— целые ненулевые числа.
Докажите, что существует такое подмножество $S \subset \Omega$, что в~$S$ нет
одновременно чисел $\pm a_i$ и~сумма чисел в~$S$ делится на~$1001$.
% решение: принцип Дирихле

\item
В~$n$~коробках лежат $n^2$~монет (коробка может быть пустой, $n \geq 3$).
За~ход можно взять три коробки, в~которых в~сумме лежит число монет, кратное
трем, и~равномерно поделить эти монеты по~этим трем коробкам.
При каких $n$, при любом начальном расположении монет по~коробкам, можно
получить позицию, в~которой во~всех коробках по~$n$~монет?

\item
Круг разделен на~$2 n$~секторов, $n$~синих и~$n$~красных.
В~красные по~часовой стрелке вписаны числа от~$1$ до~$n$, в~синие~— те~же
числа против часовой.
Докажите, что найдется полукруг с~числами от~$1$ до~$n$.

\item
Пусть $n$~— фиксированное натуральное.
При каком натуральном $m > n$, в~зависимости от~$n$, множество
$\{ n, n + 1, n + 2, \ldots, m \}$ можно разбить на~такие подмножества, чтобы
в~каждом получившемся подмножестве было число, равное сумме остальных чисел
этого подмножества?

\item
В~квадрате $(n - 1) \times (n - 1)$ проведена замкнутая несамопересекающаяся
ломаная по~линиям сетки, причем проходящая по~каждому из~$n^2$ узлов сетки
ровно по~одному разу.
Докажите, что можно провести такой разрез по~ребру сетки, чтобы получилось
две фигуры, периметр которых не~меньше четверти периметра изначальной фигуры.

\end{problems}

