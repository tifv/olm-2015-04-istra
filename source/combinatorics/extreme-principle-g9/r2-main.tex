% $date: 2015-04-08
% $timetable:
%   g9r2:
%     2015-04-08:
%       2:

\section*{Принцип крайнего}

% $authors:
%  - Александр Шаповалов

% $matter[-no-ashap-link,no-header]:
% - .[no-ashap-link]
% - /_ashap-link

\claim{Идея 1}
Обратите внимание на~объекты <<с~краю>>, где край понимается геометрически
(граница, вершина, угол) или арифметически (наибольшее, наименьшее).
Можно рассматривать и~несколько крайних объектов.
Так, для получения оценки бывает полезным выбрать два крайних объекта: для
разностей~--- наибольший и~наименьший, для расстояний~--- наиболее удаленные
друг от~друга.

\begin{problems}

\item
На~листке написаны несколько натуральных чисел.
Известно, что для любых двух найдется на~листке число, которое на~каждое из~них
делится.
Докажите, что на~листке найдется число, которое делится на~все числа.

\item
В~порядке возрастания длин лежат несколько палочек.
Можно взять любые три и~проверить, складывается~ли из~них треугольник.
За~какое наименьшее число проверок можно доказать или опровергнуть утверждение
о~том, что из~любой тройки палочек складывается треугольник?

\end{problems}

Критический момент часто случается в~конце процесса.

\begin{problems}

\item
Петя разложил 10~фруктов на~две чаши весов.
Далее он~7~раз сделал такую операцию: поменял два фрукта с~правой чаши с~одним
фруктом с~левой.
Могли~ли весы быть в~равновесии вначале и~после каждой операции?

\end{problems}

\claim{Идея 2}
\emph{Минимальный контрпример.}
Условие минимальности облегчает поиск (или построение) контрпримера, а~доказав,
что нет минимального, мы~докажем и~отсутствие контрпримеров вообще.
В~частности, полезно сократить на~общий делитель.

\begin{problems}

\item
Пусть $x^3 + x = 5$.
Докажите, что $x$~--- иррационально.

\end{problems}

