% $date: 2015-04-08
% $timetable:
%   g9r2:
%     2015-04-08: {}
%   g9r1:
%     2015-04-08: {}

% $matter[-contained,no-header]:
% - verbatim: \section*{Принцип крайнего}
% - .[contained]

\subsection*{Домашнее задание}

% $authors:
%  - Александр Шаповалов

% $matter[-no-ashap-link,contained,no-header]:
% - .[no-ashap-link]
% - /_ashap-link

\begin{problems}

\itemy{Кр1}
За~день в~библиотеке побывало 100 читателей, каждый по~разу.
Оказалось, что из~любых трех по~крайней мере двое там встретились.
Докажите, что библиотекарь мог сделать важное объявление в~такие два момента
времени, чтобы все 100 читателей его услышали.

\itemy{Кр2}
Числа $p$ и~$q$~--- целые, $x^2 + p x + q > 0$ при всех целых $x$.
Докажите, что $x^2 + p x + q > 0$ и~при всех нецелых $x$.

\itemy{Кр3}
Пусть $a$, $b$, $c$~--- натуральные числа.
Могут~ли $\text{НОК}(a, b)$ и~$\text{НОК}(a + c, b + c)$ быть равны?

\end{problems}

