% $date: 2015-04-08
% $timetable:
%   g9r1:
%     2015-04-08: {}

% $matter[-contained,no-header]:
% - verbatim: \section*{Принцип крайнего}
% - verbatim: \setproblem{3}
% - .[contained]

\subsection*{Дополнительные задачи}

% $authors:
%  - Александр Шаповалов

% $matter[-no-ashap-link,contained,no-header]:
% - .[no-ashap-link]
% - /_ashap-link

\begin{problems}

\itemy{Кр4}
На~каждой клетке шахматной доски вначале стоит по~ладье.
Каждым ходом можно снять с~доски ладью, которая бьет нечетное число ладей.
Какое наибольшее число ладей можно снять?
(Ладьи бьют друг друга, если они стоят на~одной вертикали или горизонтали
и~между ними нет других ладей).

\itemy{Кр5}
Первоначально на~доске написано число $2004!$.
Два игрока ходят по~очереди.
Игрок в~свой ход вычитает из~написанного числа какое-нибудь натуральное число,
которое делится не~более чем на~20 различных простых чисел (так, чтобы разность
была неотрицательна), записывает на~доске эту разность, а~старое число стирает.
Выигрывает тот, кто получит 0.
Кто из~играющих~--- начинающий или его соперник~--- может гарантировать себе
победу, и~как ему следует играть?

\end{problems}

