% $date: 2015-04-08
% $timetable:
%   g9r1:
%     2015-04-08:
%       1:

\section*{Принцип крайнего}

% $authors:
%  - Александр Шаповалов

% $matter[-no-ashap-link,no-header]:
% - .[no-ashap-link]
% - /_ashap-link

\claim{Идея 1}
Обратите внимание на~объекты <<с~краю>>, где край понимается геометрически
(граница, вершина, угол) или арифметически (наибольшее, наименьшее).
Можно рассматривать и~несколько крайних объектов.
Так, для получения оценки бывает полезным выбрать два крайних объекта: для
разностей~--- наибольший и~наименьший, для расстояний~--- наиболее удаленные
друг от~друга.

\begin{problems}

\item
В~порядке возрастания весов лежат несколько камней.
Есть чашечные весы без гирь.
За~какое наименьшее число взвешиваний можно проверить, правда~ли, что любая пара
камней тяжелее любого одного камня?

\item
На~доске выписано 100 различных чисел.
Докажите, что среди них можно выбрать пять чисел так, что их~среднее
арифметическое не~будет равно среднему арифметическому никаких шести
из~выписанных чисел.

\end{problems}

Критический момент часто случается в~конце процесса.

\begin{problems}

\item
Аня, Боря и~Витя сидят по~кругу за~столом и~едят орехи.
Сначала все орехи у~Ани.
Она делит их~поровну между Борей и~Витей, а~остаток (если он~есть) съедает.
Затем все повторяется: каждый следующий (по~часовой стрелке) делит имеющиеся
у~него орехи поровну между соседями, а~остаток съедает.
Вначале было больше 100 орехов.
Докажите, что хотя~бы один орех \emph{не} будет съеден.

\end{problems}

\claim{Идея 2}
\emph{Минимальный контрпример.}
Условие минимальности облегчает поиск (или построение) контрпримера, а~доказав,
что нет минимального, мы~докажем и~отсутствие контрпримеров вообще.
В~частности, полезно сократить на~общий делитель.

\begin{problems}

\item
Пусть $2^x = 10$.
Докажите, что $x$~--- иррационально.

\item
На~доске выписано 100 целых чисел.
Известно, что для любых пяти из~этих чисел найдутся такие шесть из~этих чисел,
что равны средние арифметические этой пятерки и~этой шестерки.
Докажите, что все числа равны.

\end{problems}

