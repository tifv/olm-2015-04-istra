% $date: 2015-04-03
% $timetable:
%   g11r1:
%     2015-04-03:
%       3:

\section*{Комбинаторика-2}

% $authors:
% - Олег Орлов

\begin{problems}

\item
Каждая целочисленная точка плоскости окрашена в~один из~трех цветов, причем все
три цвета присутствуют.
Докажите, что найдется прямоугольный треугольник с~вершинами трех разных
цветов.

\item
Пусть $2 S$~--- суммарный вес некоторого набора гирек.
Назовем натуральное число $k$ средним, если в~наборе можно выбрать $k$ гирек,
суммарный вес которых равен $S$.
Какое наибольшее количество средних чисел может иметь набор из~$100$ гирек?

\item
Докажите, что можно разбить все натуральные числа на~$100$ непустых подмножеств
так, что для любых таких чисел $a, b, c \in \mathbb{N}$, что $a + 99 b = c$,
хотя~бы два из~этих чисел принадлежали~бы одному подмножеству.

\item
$2 N$ участников заполняют психологическую анкету из~$2 k$ вопросов, на~которые
можно ответить <<да>> или <<нет>>.
Для любой пары вопросов известно, что ровно половина участников ответила
одинаково на~оба вопроса (либо оба ответа <<да>>, либо оба~--- <<нет>>).
Докажите, что количество участников, которые ответили ровно на~половину
вопросов <<да>>, не~превосходит $(2 N - N / k)$.

\item
Пусть $m$ и~$n$~--- натуральные числа.
Дан прямоугольник.
Известно, что этот прямоугольник можно разрезать на~горизонтальные полоски
$1 \times m$ и~вертикальные полоски $n \times 1$.
Докажите, что на~самом деле прямоугольник можно разрезать на~полоски только
одного из~этих двух типов.

\item
Пусть множество~$A$ состоит из~$N$ различных вычетов по~модулю $N^2$, где
$N \geq 2$.
Докажите, что существует такое множество~$B$, состоящее из~$N$ вычетов
по модулю $N^2$, что множество
\(
    A + B = \{ a + b \}_{a \in A, \, b \in B}
\)
содержит не~менее половины всех вычетов по модулю $N^2$.

\item
Докажите, что существует такое натуральное число~$n$, что если правильный
треугольник со~стороной $n$ разбить прямыми, параллельными его сторонам,
на~$n^2$ правильных треугольников со~стороной $1$, то~среди вершин этих
треугольников можно выбрать $2015 \, n$ точек, никакие три из~которых
не~являются вершинами правильного треугольника (не~обязательно со~сторонами,
параллельными сторонам исходного треугольника).

\end{problems}

