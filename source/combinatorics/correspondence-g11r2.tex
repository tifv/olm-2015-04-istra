% $date: 2015-04-04
% $timetable:
%   g11r2:
%     2015-04-04:
%       1:

\section*{Соответствия}

% $authors:
% - Олег Орлов

\begin{problems}

\item
Дан выпуклый $n$-угольник такой, что никакие три его диагонали не~пересекаются
в~одной точке.
Найдите количество точек пересечения диагоналей данного многоугольника
(не~являющиеся вершинами многоугольника).

\item
На~клетчатой бумаге изображен квадрат, сторона которого умещает ровно
$n$~клеток.
Сколькими способами в~этом квадрате можно поместить произвольный
\\
\sp квадрат?
\qquad
\sp прямоугольник?
\\
\sp
произвольную букву <<Г>> (в~том числе и~как угодно перевернутую)?
Здесь буква <<Г>>~--- это объединение двух прямоугольников $1 \times n$
и~$m \times 1$ с~единственной общей клеткой-концом ($m, n > 1$).
\\
(Стороны всех фигур проходят по~сторонам сетки.)

\item
Может~ли оканчиваться на~$3$ сумма делителей числа (считая единицу и~само
число), оканчивающегося на~$3$?

\item
Докажите, что количество разбиений натурального числа~$n$ в~сумму не~более чем
$k$ натуральных слагаемых, равно количеству разбиений числа~$n$ в~сумму
натуральных слагаемых, не~превосходящих $k$.

%\item
%Докажите, что количество разбиений числа~$n$ в~сумму различных слагаемых равно
%количеству разбиений числа $n$ в~сумму нечетных слагаемых.

\item
\sp
Рассмотрим набор чисел $\{ a_1, \ldots, a_{2n + 1} \}$, где каждое равно
$\pm 1$ и~сумма всех чисел набора равна единице.
Докажите, что набор можно циклически сдвинуть так, что все частичные суммы
$a_1, a_1 + a_2, \ldots,  a_1 + a_2 + \ldots + a_{2n}$ будут положительны.
\\
\sp
Сколько последовательностей $\{ a_1, a_2, \ldots, a_{2n} \}$, состоящих
из~единиц и~минус единиц, обладают тем свойством, что
$a_1 + a_2 + \ldots + a_{2n} = 0$, а~все частичные суммы
$a_1, a_1 + a_2, \ldots,  a_1 + a_2 + \ldots + a_{2n}$ неотрицательны?
\\
\sp
Сколькими способами можно пройти из~левого нижнего угла доски
$(n + 1) \times (n + 1)$ в~ее~правый верхний угол, сдвигаясь каждым ходом
на~одну клетку вправо или вверх и~не~разу не~оказавшись выше главной диагонали
доски?
\\
\sp
Сколько существует способов разрезать выпуклый $(n + 2)$-угольник
на~треугольники непересекающимися диагоналями?

%\item
%Рассмотрим последовательность из~$n$ натуральных чисел.
%Будем называть её~\emph{уморительной}, если вместе с~каждым $k \geq 2$
%в~последовательность входит также и~число $(k - 1)$, причем первое вхождение
%$(k - 1)$ расположено до~последнего вхождения $k$.
%Сколько уморительных последовательностей существует?

\item
Боковая поверхность прямоугольного параллелепипеда с~основанием $a \times b$
и~высотой $c$ ($a$, $b$ и~$c$~--- натуральные числа) оклеена по~клеточкам без
наложений и~пропусков прямоугольниками со~сторонами, параллельными ребрам
параллелепипеда, каждый из~которых состоит из~четного числа единичных квадратов
(верх и~низ клеить не~надо).
При этом разрешается перегибать прямоугольники через боковые ребра
параллелепипеда.
Докажите, что если $c$ нечетно, то~число способов оклейки четно.

\end{problems}

