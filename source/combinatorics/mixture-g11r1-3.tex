% $date: 2015-04-06
% $timetable:
%   g11r1:
%     2015-04-06:
%       3:

\section*{Добавка по комбинаторике}

% $authors:
% - Олег Орлов

\begin{problems}

\item
Несколько ребят стоят в~круг.
У~каждого есть некоторое количество конфет.
Сначала у~каждого четное количество конфет.
По~команде каждый передает половину своих конфет стоящему справа.
Если после этого у~кого-нибудь оказалось нечетное количество конфет, то~ему
извне добавляется одна конфета.
Это повторяется много раз.
Докажите, что настанет время, когда у~всех будет поровну конфет.

\item
По~одной стороне бесконечного коридора расположено бесконечное количество
комнат, занумерованных числами от~минус бесконечности до~плюс бесконечности.
В~комнатах живут $2015$ пианистов (в~одной комнате могут жить несколько
пианистов), кроме того, в~каждой комнате находится по~роялю.
Каждый день какие-то два пианиста, живущие в~соседних комнатах
($K$-той и~$(K + 1)$-ой), приходят к~выводу, что они мешают друг другу,
и~переселяются соответственно в~$(K - 1)$-ую и~$(K + 2)$-ую комнаты.
Докажите, что через конечное число дней эти переселения прекратятся.
(Пианисты, живущие в~одной комнате, друг другу не~мешают).

\item
Каждая клетка прямоугольника $2 m \times 2 n$ раскрашена в~один из~двух цветов.
Известно, что если поставить ладью на~любую клетку, то~она будет бить больше
клеток не~своего цвета (считаем, что ладья бьет клетку на~которой стоит).
Докажите, что в~каждой строке и~в~каждом столбце обоих цветов поровну.

\item
Дано натуральное число~$k$.
На~бесконечной клетчатой плоскости каждая клетка является суверенным
государством, а~на~каждом ребре стоит таможня, взимающая натуральное число
талеров в~качестве взятки за~ее~пересечение (в~обоих направлениях~---
одинаковое, но, возможно, различное для разных границ).
Докажите, что существует такой замкнутый маршрут, не~заходящий ни~в~какую
клетку дважды, что суммарная взятка на~нем кратна $k$.

\end{problems}

