% $date: 2015-04-03
% $timetable:
%   g10r1:
%     2015-04-03:
%       2:

\section*{Инварианты}

% $authors:
% - Фёдор Бахарев

\begin{problems}

\item
Имеется три кучи камней.
Сизиф таскает по~одному камню из~кучи в~кучу.
За~каждое перетаскивание он~получает от~Зевса количество монет, равное разности
числа камней в~куче, в~которую он~кладет камень, и~числа камней в~куче,
из~которой он~берет камень (сам перетаскиваемый камень при этом
не~учитывается).
Если указанная разность отрицательна, то~Сизиф возвращает Зевсу соответствующую
сумму.
(Если Сизиф не~может расплатиться, то~великодушный Зевс позволяет ему совершать
перетаскивание в~долг.)
В~некоторый момент оказалось, что все камни лежат в~тех~же кучах, в~которых
лежали первоначально.
Каков наибольший суммарный заработок Сизифа на~этот момент?

\item
На~прямой стоят две фишки, слева~— красная, справа~— синяя.
Разрешается производить любую из~двух операций:
вставку двух фишек одного цвета подряд в~любом месте прямой
и~удаление любых двух соседних одноцветных фишек.
Можно~ли за~конечное число операций оставить на~прямой ровно две фишки: красную
справа, а~синюю~— слева?

\item
На~окружности имеются синие и~красные точки.
Разрешается добавить красную точку и~поменять цвета ее~соседей, а~также убрать
красную точку и~изменить цвета ее~бывших соседей.
Пусть первоначально было всего две красные точки (менее двух точек оставлять
не~разрешается).
Доказать, что за~несколько разрешенных операций нельзя получить картину,
состоящую из~двух синих точек.

\item
На~доске написаны три функции:
\[
    f_1(x) = x + \frac{1}{x}
\, , \quad
    f_2(x) = x^2
\, , \quad
    f_3(x) = (x - 1)^2
\, . \]
Можно складывать, вычитать и~перемножать эти функции (в~том числе возводить
в~квадрат, в~куб, \ldots), умножать их~на~произвольное число, прибавлять к~ним
произвольное число, а~также проделывать эти операции с~полученными выражениями.
Получите таким образом функцию $1 / x$.
Докажите, что если стереть с~доски любую из~функций $f_1$, $f_2$, $f_3$,
то~получить $1 / x$ невозможно.

\item
Ножки циркуля находятся в~узлах бесконечного листа клетчатой бумаги, клетки
которого~— квадраты со~стороной~1.
Разрешается, не~меняя раствора циркуля, поворотом его вокруг одной из~ножек
перемещать вторую ножку в~другой узел на~листе.
Можно~ли за~несколько таких шагов поменять ножки циркуля местами?
%http://www.problems.ru/view_problem_details_new.php?id=109957

\item
На~бесконечной в~обе стороны полосе из~клеток, пронумерованных целыми числами,
лежит несколько камней (возможно, по~нескольку в~одной клетке).
Разрешается выполнять следующие действия:
\begin{enumerate}
\item
Снять по~одному камню с~клеток $(n - 1)$ и~$n$ и~положить один камень
в~клетку $(n + 1)$;
\item
Снять два камня с~клетки~$n$ и~положить по~одному камню в~клетки $(n + 1)$,
$(n - 2)$.
\end{enumerate}
\subproblem
Докажите, что при любой последовательности действий мы~достигнем ситуации,
когда указанные действия больше выполнять нельзя.
\\
\subproblem
Докажите, что эта конечная ситуация не~зависит от~последовательности действий
(а~зависит только от~начальной раскладки камней по~клеткам).

\end{problems}

