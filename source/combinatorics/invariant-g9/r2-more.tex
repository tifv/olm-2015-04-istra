% $date: 2015-04-02
% $timetable:
%   g9r2:
%     2015-04-02: {}

% $matter[-contained,no-header]:
% - verbatim: \section*{Переправы и инварианты}
% - .[contained]

\section*{Переправы и инварианты (домашнее задание)}

% $authors:
% - Александр Шаповалов

% $matter[-no-ashap-link,contained,no-header]:
% - .[no-ashap-link]
% - /_ashap-link

\begin{problems}

\item
Натуральные числа от~1 до~2015 покрашены в~красный и~синий цвета.
Есть пара красных и~пара синих чисел с~одинаковыми произведениями.
Докажите, что можно выбрать пару красных и~пару синих чисел с~одинаковыми
суммами.

\itemy{4а}
Три ладьи стояли на~клетках \texttt{a1}, \texttt{b1} и~\texttt{a2}.
За~несколько ходов они перешли в~клетки у~противоположного угла доски:
с~\texttt{a1}~--- на~\texttt{h8},
с~\texttt{b1}~--- на~\texttt{h7},
с~\texttt{a2}~--- на~\texttt{g8}.
Докажите, что после какого-то из~ходов какая-то из~ладей не~била других ладей.

\item
По~шахматной доске прокатили кубик.
Он~встал той~же гранью на~ту~же клетку.
Может~ли кубик оказаться повернутым на~$90^\circ$ вокруг вертикальной оси?

\item
\emph{(Сдать письменно 2 или 3 апреля.)}
Есть три одинаковых больших сосуда.
В~одном~--- $3\,\text{л}$ сиропа, в~другом~--- $N\,\text{л}$ воды,
третий~--- пустой.
Можно выливать из~одного сосуда всю жидкость в~другой или в~раковину.
Можно выбрать два сосуда и~доливать в~один из~них из~третьего, пока уровни
жидкости в~выбранных сосудах не~сравняются.
При каких целых $N$ можно получить $10\,\text{л}$ разбавленного $30\%$-го
сиропа?

\end{problems}

