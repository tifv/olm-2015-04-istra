% $date: 2015-04-02
% $timetable:
%   g9r1:
%     2015-04-02:
%       3:

\section*{Переправы и инварианты}

% $authors:
% - Александр Шаповалов

% $matter[-no-ashap-link,no-header]:
% - .[no-ashap-link]
% - /_ashap-link

\begin{flushright}
    Сколь ни вдоль, а поперек изволь.
    \emph{Поговорка}
\end{flushright}

\begin{problems}

\itemy{0}\emph{Затравка.}
Дан жесткий проволочный контур квадрата площади $1\,\text{дм}^2$.
Его разрезали на части и спаяли заново.
Получился контур плоского многоугольника.
\\
\sp
Резали на 3 части.
Какова наибольшая возможная площадь нового многоугольника?
\\
\sp
Резали на 4 части.
Могла ли получиться площадь не менее $105\,\text{см}^2$?

\end{problems}

Если объекты или ситуации задачи четко делятся на~две категории
(два берега, два цвета), и~если путь начинается на~одном берегу,
а~заканчивается на~другом, то~неизбежно придется переправляться.

\begin{problems}

\item
Плоскость раскрашена в~два цвета.
Докажите, что есть точки разного цвета на~расстоянии $1\,\text{мм}$. 

\item
Можно~ли на~всех полях шахматной доски расставить коней четырех мастей так,
чтобы вороные не~били соловых, соловые~--- гнедых, гнедые~--- каурых,
а~каурые~--- вороных?

\end{problems}

Вместо цветов используют значения какой-нибудь величины, например, остатки.
Переправа может оказаться ключевым местом решения: надо только суметь привязать
к~ней вопрос задачи.

\begin{problems}

\item
Можно ли расставить в таблице $8 \times 8$ числа от 1 до 64 так, чтобы
ни в какой паре клеток с общей стороной или вершиной сумма не делилась 
\quad
\sp на 4;
\quad
\sp на 3?

\item
Натуральные числа раскрашены в синий и красный цвета, причем чисел каждого
цвета бесконечно много.
Докажите, что найдется бесконечно много чисел, представимых как сумма двух
различных синих, так и как сумма двух различных краснах.

\end{problems}

Типичная ситуация: есть набор позиций (состояний) и~переходы между ними.
Это можно рассматривать как граф.
Пусть с~каждой позицией можно связать некоторую величину.
Если величина при переходах не~меняется, она~--- инвариант.
Значения инварианта разбивают граф на~компоненты связности, и~нет маршрута
между позиций с~разными значениями инварианта.
Соответственно, можно доказывать невозможность действия: например, нельзя
доехать на~поезде от~Нью-Йорка до~Москвы, поскольку поезда из~Америки ходят
только в~Америку.
Но~можно доказать и~существование: если добраться таки удалось, то~был либо
перелет, либо плавание. 

Типичные инварианты: четность, общий делитель, сумма. 

\begin{problems}

\item
В банке 1100 долларов.
Разрешаются две операции: взять из банка 370 долларов или положить в него
111 долларов.
Эти операции можно проводить много раз, при этом, однако, никаких денег, кроме
тех, что первоначально лежат в банке, нет.
Какую максимальную сумму можно извлечь из банка и как это сделать?

\item
Есть три кучки камней: в первой 51 камень, во второй~--- 49, а в третьей~--- 5.
Разрешается объединять любые кучки в одну, а также разделять кучку, состоящую
из четного количества камней, на две равные.
Можно ли получить 37 кучек камней?

\end{problems}

Нечисловые инварианты чаще всего связаны с~чередованием или с~невозможностью
уничтожить элемент с~каким-то свойством.

\begin{problems}

\item
Картонный треугольник катают по~плоскости, перекатывая через сторону.
После 2015 перекатываний он~попал в~точности на~исходное место.
Докажите, что треугольник равнобедренный.

\item
Маляр-хамелеон ходит по~клетчатой доске как обычная шахматная ладья.
Попав в~очередную клетку, он~либо перекрашивается в~ее~цвет, либо перекрашивает
клетку в~свой цвет.
Белого маляра-хамелеона кладут на~черную доску размерами $8 \times 8$ клеток.
Сможет~ли он~раскрасить ее~в~шахматном порядке?

\item
Остроугольный треугольник разрезали прямолинейным разрезом на две
(не обязательно треугольные) части, затем одну из этих частей~--- опять на две
части, и так далее: на каждом шагу выбирали любую одну из уже имеющихся частей
и разрезали ее (по прямой) на две.
Через несколько шагов оказалось, что исходный треугольник распался на несколько
треугольников.
Могут ли все они быть тупоугольными?

\end{problems}

