% $date: 2015-04-02
% $timetable:
%   g9r1: {}

\section*{Переправы и инварианты: допзадачи покруче}

% $authors:
% - Александр Шаповалов

% $matter[-no-ashap-link,no-header]:
% - .[no-ashap-link]
% - /_ashap-link

% $build$matter[print]: [[.], [.]]
% $build$style[print]:
% - .[tiled4,-print]

\begin{problems}

\itemy{Ин5}
Дана пустая таблица размера $13 \times 17$.
Двое по очереди ставят в нее фишки в ее пустые клетки.
Первый может поставить фишку на пересечение строки и столбца, если в строке
и столбце в сумме четное число фишек, второй~--- если нечетное.
Игра заканчивается, когда ход невозможен.
Докажите, что последний ход всегда делает один и тот же игрок.
Кто именно?

\itemy{Ин6}
Можно ли расставить в таблице $50 \times 50$ числа от 1 до 2500 так, чтобы
ни в какой паре клеток с общей стороной или вершиной сумма не делилась на 4?

\itemy{Ин7}
На главной диагонали квадрата $101 \times 101$ стоят пять ладей.
Каждым ходом можно переставить одну из ладей в любом направлении по вертикали
или горизонтали вплотную к ближайшей ладье или стенке квадрата
(перепрыгивать через ладьи нельзя).
Может ли в итоге одна ладья оказаться в центральной клетке квадрата,
а остальные четыре~--- в клетках, соседних по стороне с центральной?

\itemy{Ин8}
Ножки циркуля поставили в узлы бесконечной клетчатой решетки.
За один ход можно передвинуть одну ножку в другой узел, не отрывая другой
и не меняя раствора циркуля.
Можно ли вернуться в исходные вершины, поменяв местами ножки?

\end{problems}

