% $date: 2015-03-31
% $timetable:
%   g11r2:
%     2015-03-31:
%       2:

\section*{Игры}

% $authors:
% - Олег Орлов

\begin{problems}

\item
Прямоугольная полоса размером $1 \times n$ ($n \geq 4$) составлена из единичных
полей, занумерованных числами $1, 2, \ldots, n$.
На полях с номерами $(n - 2)$, $(n - 1)$, $n$ стоит по одной фишке.
Двое играют в следующую игру: каждый игрок своим ходом может перенести любую
фишку на любое свободное поле с меньшим номером.
Проигрывает тот, кто не может сделать очередного хода.
Кто выигрывает при правильной игре?

\item
Двое игроков отмечают точки на плоскости.
Сначала первый отмечает точку красным цветом, затем второй отмечает $100$ точек
синим, затем первый снова одну точку красным, второй $100$ точек синим и так
далее.
(Перекрашивать уже отмеченные точки нельзя.)
Докажите, что первый может построить правильный треугольник с красными
вершинами.

\item
В углу шахматной доски размером $m \times n$ полей стоит ладья.
Двое по очереди передвигают ее по вертикали или по горизонтали на любое число
полей;
при этом не разрешается, чтобы ладья стала на поле или прошла через поле,
на котором она уже побывала (или через которое уже проходила).
Проигрывает тот, кому некуда ходить.
Кто из играющих может обеспечить себе победу: начинающий или его партнер, и как
ему следует играть?

\item
Двое играют на доске $99 \times 2014$ клеток.
Каждый по очереди отмечает квадрат по линиям сетки (любого возможного размера)
и закрашивает его.
Выигрывает тот, кто закрасит последнюю клетку.
Дважды закрашивать клетки нельзя.
Кто выиграет при правильной игре и как надо играть?

%\item
%Двое играют в следующую игру: они ставят по очереди в таблицу $100 \times 100$
%плюс и минус единицы, пока она не заполнится.
%Затем подсчитывается сумма произведений чисел во всех строках и столбцах.
%Начинающий выигрывает, если эта сумма окажется неотрицательной, а проигрывает
%в противном случае.
%Кто выигрывает при правильной игре?

\item
Имеется $99!$ молекул.
Двое по очереди за один ход съедают не меньше одной, но не больше $1 \%$
молекул.
Проигрывает тот, кто не может сделать ход.
Кто выигрывает при правильной игре?

\item
Два игрока по очереди проводят диагонали в правильном $(2 n + 1)$-угольнике
($n > 1$).
Разрешается проводить диагональ, если она пересекается (по внутренним точкам)
с четным числом ранее проведенных диагоналей (и не была проведена раньше).
Проигрывает игрок, который не может сделать очередной ход.
Кто выигрывает при правильной игре?

\item
Клетчатый квадрат $100 \times 100$ разрезан на доминошки
(прямоугольники $1 \times 2$).
Двое играют в игру.
Каждым ходом игрок склеивает две соседних по стороне клетки, между которыми был
проведен разрез.
Игрок проигрывает, если после его хода фигура получилась связной, т.~е. весь
квадрат можно поднять со стола, держа его за одну клетку.
Кто выиграет при правильной игре~--- начинающий или его соперник?

\end{problems}

