% $date: 2015-03-31
% $timetable:
%   g10r2:
%     2015-03-31:
%       1:

\section*{Принцип крайнего}

% $authors:
% - Николай Крохмаль

\begin{problems}

\item
Пусть $A$~--- некоторое множество в~трехмерном пространстве.
Известно, что для любой точки~$X$ из~множества~$A$ найдутся еще две точки
$Y$ и~$Z$ из~этого~же множества такие, что $X$~--- середина отрезка~$YZ$.
Докажите, что в~множестве~$A$ бесконечно много точек.

\item
Существуют~ли такие 2015 различных натуральных чисел, что сумма каждых 2014
из~них не~меньше квадрата оставшегося?

\item
На~прямой дано 50 отрезков.
Докажите, что либо некоторые 8~отрезков имеют общую точку, либо найдутся
8~отрезков, никакие два из~которых не~имеют общей точки.

\item
8~школьников смотрят 8-серийный сериал.
Оказалось, что каждую серию смотрели ровно 5 школьников.
Докажите, что найдется такая пара школьников, что каждую серию смотрел хотя~бы
один из~них.

\item
У~Коли есть отрезок длины~$k$, а~у~Лёвы~--- отрезок длины~$l$.
Сначала Коля делит свой отрезок на~три части, а~потом Лёва делит на~три части
свой отрезок.
Если из~получившихся шести отрезков можно сложить два треугольника,
то~выигрывает Лёва, а~если нет~--- Коля.
Кто из~играющих, в~зависимости от~отношения $k : l$, может обеспечить себе
победу, и~как ему следует играть?

\item
Докажите, что для любого натурального $n$ число
\[
    1 + \frac{1}{3} + \frac{1}{5} + \ldots + \frac{1}{2 n + 1}
\]
не~является целым.

\item
На~плоскости дано $n$~точек.
Известно, что площадь треугольника, образованного любыми тремя из~них,
не~превосходит 1.
Докажите, что все точки можно поместить в~треугольник площади 4.

\item
Пусть элементами таблицы $n \times n$ являются нули и~единицы.
Пусть при этом выполнено следующее условие: если на~некотором месте таблицы
записан нуль, то~сумма чисел столбца и~строки, содержащих этот нуль,
не~меньше $n$.
Докажите, что сумма всех $n^2$ чисел не~меньше $n^2 / 2$.

\end{problems}

