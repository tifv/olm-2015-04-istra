% $date: 2015-04-01
% $timetable:
%   g9r2:
%     2015-04-01:
%       3:
%   g9r1:
%     2015-04-01:
%       1:

\section*{Полуинвариант}

% $authors:
% - Александр Шаповалов

% $matter[-no-ashap-link,no-header]:
% - .[no-ashap-link]
% - /_ashap-link

Пусть есть последовательность объектов, или процесс, в котором позиции
последовательно сменяются.
Полуинвариант~— это связанное с позицией число, которое при разрешенных
действиях все время растет или все время убывает (возможно, нестрого).
Выбор полуинварианта зависит от цели.

\begin{problems}

\item
На Архипелаге Сыщик гоняется за Шпионом.
Оба используют только маршрутные корабли, которые курсируют ежедневно между
некоторыми островами.
Каждый корабль отплывает утром и приплывает на остров назначения к вечеру.
С пересадками можно добраться с любого острова на любой.
Сыщик всегда знает, где сейчас Шпион, и поймает его, если окажется с ним
на одном острове.
Сыщик может плыть в любой день, Шпион не плавает по пятницам.
Как Сыщику поймать Шпиона?

\end{problems}

Оценив изменение полуинварианта на одном шаге или на группе шагов, можно
оценить, за сколько шагов процесс закончится.

\begingroup
\ifx\problemfigurewidth\undefined
\newlength\problemfigurewidth
\newlength\problemtextwidth
\newlength\problemspacewidth
\fi
\setlength\problemfigurewidth{2.5cm}
\setlength\problemspacewidth{1em}
\setlength\problemtextwidth{\linewidth}
\addtolength\problemtextwidth{-\problemfigurewidth}
\addtolength\problemtextwidth{-\problemspacewidth}
\begin{minipage}{\problemtextwidth}
\begin{problems}

\item
\subproblem
На шахматной доске $100 \times 100$ коню разрешено ходить только в четырех
направлениях (рис. справа).
Докажите, что с какой бы клетки он ни начал, ему удастся сделать лишь конечное
число ходов.
\\
\subproblem
Какое наибольшее число ходов конь может сделать на этой доске?

\end{problems}
\end{minipage}\hspace{\problemspacewidth}%
\begin{minipage}{\problemfigurewidth}
    \jeolmfigure[width=\linewidth]{../knight-restricted-moves}
\end{minipage}
\endgroup

Типичные полуинварианты: сумма, произведение, модуль разности, сумма модулей,
сумма квадратов и их комбинации (например, суммы).

\begin{problems}

\item
На доске написаны 100 натуральных чисел.
За ход можно либо заменить два числа на их сумму, либо разложить число
в произведение двух меньших различных чисел и заменить его на эти два числа.
Докажите, что рано или поздно на доске останется одно число.

\item
На доске написаны 3 различных натуральных числа.
За ход можно взять одно из крайних чисел (наибольшее или наименьшее) и заменить
на среднее арифметическое, геометрическое или гармоническое его с каким-то
другим из чисел (при условии, что это среднее~— натурально, и все числа
остаются различными).
Докажите, что удастся сделать лишь конечное число ходов.

\end{problems}

Для строки целых чисел (или объектов) полуинвариантом может быть многозначное
число в некоторой системе счисления.

\begin{problems}

\item
В двух коробках лежат по 9~шариков.
За один ход можно убрать из любой коробки 1~шарик или убрать 1~шарик из левой
коробки и положить 9~шариков в правую.
\\
\subproblem
Докажите, что ходы рано или поздно закончатся.
\\
\subproblem
Какое наибольшее число ходов могло быть сделано?

\item
\subproblem
В строке записаны 100 цифр.
Петя находит пару рядом стоящих цифр, где правая меньше левой, и меняет
их местами.
Докажите, что рано или поздно перестановки прекратятся.
\\
\subproblem
В строку в беспорядке записаны по разу числа $1, 2, \ldots, 100$.
Петя находит пару рядом стоящих чисел, где правое меньше левого, и меняет
их местами.
Докажите, что рано или поздно числа расположатся по порядку
$1, 2, \ldots, 100$.

\end{problems}

Если есть строгий полуинвариант, то позиция не может повториться
(и, в частности, процесс не может зациклиться).
В большинстве игр наличие полуинварианта гарантирует, что игра закончится.

\begin{problems}

\item
Есть 10 различных чисел.
За одну операцию можно два неравных числа заменить на два равных с той же
суммой.
\\
\subproblem
Может ли процесс продолжаться бесконечно?
\\
\subproblem
Может ли один и тот же набор чисел возникнуть дважды?

\end{problems}

Если полуинвариант может принимать лишь конечное число значений, или убывает,
принимая лишь натуральные значения, то он достигнет «крайнего» значения.
Это может обеспечить искомую позицию.

\begin{problems}

\item
\subproblem
В клетках таблицы $99 \times 99$ расставлены плюсы и минусы.
Если в каком-то прямоугольнике из 99~клеток минусов больше чем плюсов,
разрешается поменять в нем все знаки на противоположные.
Докажите, что через некоторое время во всех таких прямоугольниках плюсов будет
больше чем минусов.
\\
\subproblem
В клетках таблицы $99 \times 99$ расставлены целые числа.
Если в каком-то прямоугольнике из 99~клеток сумма отрицательна, разрешается
поменять в нем знаки всех чисел на противоположные.
Докажите, что через некоторое время сумма чисел в каждом из таких
прямоугольников сумма будет неотрицательной.
\\
\subproblem
Как в предыдущем пункте, но числа~— действительные.

\end{problems}

