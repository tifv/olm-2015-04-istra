% $date: 2015-04-01
% $timetable:
%   g9r2:
%     2015-04-01: {}

\section*{Полуинвариант: на дом}

% $authors:
% - Александр Шаповалов

% $build$matter[print]: [[.], [.]]

% $matter[-no-link,no-header]:
% - .[no-link]
% - /_ashap-link

\begin{problems}

\itemy{ПИ1}
Есть 10 различных целых чисел (не~обязательно положительных).
За~одну операцию можно два не~равных числа одинаковой четности заменить
на~два равных с~той~же суммой.
Может~ли процесс продолжаться бесконечно?

\itemy{ПИ2}
Вначале на~доске было написано натуральное число, меньшее 1000.
Каждым ходом число увеличивали в~$q$ раз.
\\
\sp
После каждого из~первых 9 ходов получалось целое число.
Верно~ли, что и~дальше будут получаться только целые числа?
\\
\sp
То~же, но~целые числа получались после каждого из~первых 10 ходов?

\itemy{ПИ3}
Аня, Боря и~Витя сидят по~кругу за~столом и~едят орехи.
Сначала все орехи у~Ани.
Она делит их~поровну между Борей и~Витей, а~остаток (если он~есть) съедает.
Затем все повторяется: каждый следующий (по~часовой стрелке) делит имеющиеся
у~него орехи поровну между соседями, а~остаток съедает.
Вначале было больше 100 орехов.
Докажите, что хотя~бы один орех будет съеден.

\itemy{ПИ4}
На~окружности расставлено несколько положительных чисел, каждое из~которых
не~больше 1.
Докажите, что можно разделить окружность на~три дуги так, что суммы чисел
на~соседних дугах будут отличаться не~больше чем на~1.
(Если на~дуге нет чисел, то~сумма на~ней считается равной нулю.)

\end{problems}

