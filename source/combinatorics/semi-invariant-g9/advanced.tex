% $date: 2015-04-01
% $timetable:
%   g9r1: {}

% $matter[-contained,no-header]:
% - verbatim: \section*{Полуинвариант}
% - verbatim: \setproblem{4}
% - .[contained]

\subsection*{Дополнительная задача}

% $authors:
% - Александр Шаповалов

% $matter[-no-ashap-link,contained,no-header]:
% - .[no-ashap-link]
% - /_ashap-link

\begin{problems}

\item
На~окружности сидят 12~кузнечиков в~различных точках.
Эти точки делят окружность на~12 дуг.
По~сигналу кузнечики одновременно прыгают по~часовой стрелке, каждый~—
из~конца своей дуги в~ее~середину.
Образуются новые 12~дуг, прыжки повторяются, и~т.~д.
\\
\subproblem
Все кузнечики одновременно вернулись в~исходные точки, каждый в~свою.
Какое наименьшее число прыжков мог сделать каждый кузнечик?
\\
\subproblem
Может~ли хотя~бы один кузнечик вернуться в~свою исходную точку после того, как
им~сделано 12~прыжков?
\\
\subproblem
Как в~предыдущем пункте, но~13 прыжков?

\end{problems}

