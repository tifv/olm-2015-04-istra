% $date: 2015-04-01
% $timetable:
%   g9r1: {}

\section*{Полуинвариант: задача для второгодника}

% $authors:
% - Александр Шаповалов

% $matter[-no-link,no-header]:
% - .[no-link]
% - /_ashap-link

\begin{problems}

\itemy{ПИ5}
На окружности сидят 12 кузнечиков в различных точках. Эти точки делят окружность на 12 дуг. По сигналу кузнечики одновременно прыгают по часовой стрелке, каждый – из конца своей дуги в ее середину. Образуются новые 12 дуг, прыжки повторяются, и т. д. 
\\
\sp
Все кузнечики одновременно вернулись в исходные точки, каждый в свою. Какое наименьшее число прыжков мог сделать каждый кузнечик?
\\
\sp
Может ли хотя бы один кузнечик вернуться в свою исходную точку после того, как им сделано 12 прыжков?
\\
\sp
Как в предыдущем пункте, но 13 прыжков?

\end{problems}

