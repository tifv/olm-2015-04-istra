% $date: 2015-04-01
% $timetable:
%   g9r1:
%     2015-04-01: {}

% $matter[-contained,no-header]:
% - verbatim: \section*{Полуинвариант}
% - .[contained]

\subsection*{Домашнее задание}

% $authors:
% - Александр Шаповалов

% $build$matter[print]: [[.], [.]]

% $matter[-no-ashap-link,contained,no-header]:
% - .[no-ashap-link]
% - /_ashap-link

\begin{problems}

\item
Есть 10 различных целых чисел (не~обязательно положительных).
За~одну операцию можно два не~равных числа одинаковой четности заменить
на~два равных с~той~же суммой.
Может~ли процесс продолжаться бесконечно?

\item
Вначале на~доске было написано натуральное число, меньшее 1000.
Каждым ходом число увеличивали в~$q$ раз.
\\
\subproblem
После каждого из~первых 9 ходов получалось целое число.
Верно~ли, что и~дальше будут получаться только целые числа?
\\
\subproblem
То~же, но~целые числа получались после каждого из~первых 10 ходов?

\item
Аня, Боря и~Витя сидят по~кругу за~столом и~едят орехи.
Сначала все орехи у~Ани.
Она делит их~поровну между Борей и~Витей, а~остаток (если он~есть) съедает.
Затем все повторяется: каждый следующий (по~часовой стрелке) делит имеющиеся
у~него орехи поровну между соседями, а~остаток съедает.
Вначале было больше 100 орехов.
Докажите, что хотя~бы один орех будет съеден.

% spell Оз

\item
В~Стране Оз все города подняли над ратушами флаги~— голубые либо золотистые.
Каждый день жители узнают цвета флагов у~соседей в~радиусе $100\,\text{км}$.
Один из~городов, где у~большинства соседей флаги другого цвета, меняет свой
флаг на~этот другой цвет.
Докажите, что со~временем смены цвета флагов прекратятся.

\end{problems}

