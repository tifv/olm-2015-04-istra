% $date: 2015-04-01
% $timetable:
%   g10r2:
%     2015-04-01:
%       1:

% $caption: Оценка + пример (комбинаторика)

\section*{Оценка + пример}

% $authors:
% - Николай Крохмаль

\begin{problems}

\item
В~квадратной таблице размером $100 \times 100$ некоторые клетки закрашены.
Каждая закрашенная клетка является единственной закрашенной клеткой либо
в~своем столбце, либо в~своей строке.
Какое наибольшее количество клеток может быть закрашено?

\item
В~клетках таблицы $3 \times 3$ расставлены числа так, что сумма чисел в~каждом
столбце и~в~каждой строке равна нулю.
Какое наименьшее количество чисел, отличных от~нуля, может быть в~этой таблице,
если известно, что оно нечетно?

\item
В~футбольном чемпионате участвовали $16$~команд.
Каждая команда сыграла с~каждой по~одному разу, за~победу давалось $3$~очка,
за~ничью $1$~очко, за~поражение~--- $0$.
Назовем команду \emph{успешной,} если она набрала хотя~бы половину
от~наибольшего возможного количества очков.
Какое наибольшее количество успешных команд могло быть в~турнире?

\item
Имеется 24~карандаша четырех цветов~--- по~6~карандашей каждого цвета.
Их~раздали 6~ребятам так, что каждый получил по~4~карандаша.
Какое наименьшее количество ребят всегда можно выбрать, чтобы у~них
гарантированно нашлись карандаши всех цветов, вне зависимости от~распределения
карандашей?

\item
У~каждого из~$1000$ гномов есть колпак, синий снаружи и~красный внутри
(или наоборот).
Если на~гноме надет красный колпак, то~он~может только лгать, а~если синий~---
только говорить правду.
На~протяжении одного дня каждый гном сказал каждому:
<<На~тебе красный колпак!>>
(при этом некоторые гномы в~течение дня выворачивали свой колпак наизнанку).
Найдите наименьшее возможное количество выворачиваний.

\item
Изначально на~столе лежит 111 кусков пластилина одинаковой массы.
За~одну операцию можно выбрать несколько групп по~одинаковому количеству кусков
и~в~каждой группе весь пластилин слепить в~один кусок.
За~какое наименьшее количество операций можно получить ровно 11~кусков, любые
два из~которых имеют различные массы?

\item
На~окружности выбрано $2012$ точек, делящих ее~на~равные дуги.
Из~них выбрали $k$~точек и~построили выпуклый $k$-угольник с~вершинами
в~выбранных точках.
При каком наибольшем $k$ могло оказаться, что у~этого многоугольника нет
параллельных сторон?

\item
Прямую палку длиной 2~метра распилили на~$N$~палочек, длина каждой из~которых
выражается целым числом сантиметров.
При каком наименьшем $N$ можно гарантировать, что, использовав все получившиеся
палочки, можно, не~ломая их, сложить контур некоторого прямоугольника?

\item
На~доске нарисован выпуклый $2015$-угольник.
Петя последовательно проводит в~нем диагонали так, чтобы каждая вновь
проведенная диагональ пересекала по~внутренним точкам не~более одной
из~проведенных ранее диагоналей.
Какое наибольшее количество диагоналей может провести Петя?

%\item
%Пусть $P(x)$~--- многочлен со~старшим коэффициентом $1$, а~последовательность
%целых чисел $a_1, a_2, \ldots$ такова, что
%$P(a_1) = 0$, $P(a_2) = a_1$, $P(a_3) = a_2$ и~т.~д.
%Числа в~последовательности не~повторяются.
%Какую степень может иметь $P(x)$?

%\item
%На~доске написаны девять приведённых квадратных трехчленов:
%$x^2 + a_1 x + b_1, x^2 + a_2 x + b_2, \ldots, x^2 + a_3 x + b_3$.
%Известно, что последовательности
%$a_1, a_2, \ldots, a_9$ и~$b_1, b_2, \ldots, b_9$~--- арифметические
%прогрессии.
%Оказалось, что сумма всех девяти трехчленов имеет хотя~бы один корень.
%Какое наибольшее количество исходных трехчленов может не~иметь корней?

%\item
%На~доске написали $2015$ попарно различных натуральных чисел
%$a_1, a_2, \ldots, a_{2015}$.
%Затем под каждым числом $a_{i}$ написали $b_{i}$, полученное прибавлением
%к~$a_{i}$ наибольшего общего делителя остальных $2014$ исходных чисел.
%Какое наименьшее количество попарно различных чисел может быть среди
%$b_1, b_2, \ldots, b_{2015}$?

\end{problems}

