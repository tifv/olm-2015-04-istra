% $date: 2015-04-06
% $timetable:
%   g10r2:
%     2015-04-06:
%       3:
%   g10r1:
%     2015-04-06:
%       2:

\section*{Теория Рамсея --- 1: полные подграфы}

% $authors:
% - Владимир Шарич

\begin{problems}

\itemy{-3}
На~плоскости даны 6~точек общего положения, все расстояния между которыми
различны.
Докажите, что найдутся два треугольника с~вершинами в~этих точках и~с~общей
стороной, такой что в~одном треугольнике она~--- самая короткая сторона,
а~в~другом~--- самая длинная.

\itemy{-2}
17 ученых переписываются по~трем дисциплинам.
При этом любая пара ученых переписывается ровно по~одной дисциплине.
Докажите, что можно выбрать троих ученых, которые переписываются по~одной
дисциплине.

%\item
%Каждое из~ребер полного графа с~9 вершинами покрашено в~синий или красный цвет.
%Докажите, что либо есть четыре вершины, все ребра между которыми~--- синие,
%либо есть три вершины, все ребра между которыми~--- красные.

\itemy{-1}
Каждое из~ребер полного графа с~18~вершинами покрашено в~один из~двух цветов.
Докажите, что найдутся 4~вершины, все ребра между которыми одного цвета.

\end{problems}

\medskip
\hrule

\begin{problems}

\item
Десять попарно различных ненулевых чисел таковы, что для любых двух из~них либо
сумма этих чисел, либо их~произведение~--- рациональное число.
Докажите, что квадраты всех чисел рациональны.

\item
Все ребра полного графа с~бесконечным количеством вершин покрашены
в~$k$ цветов.
Докажите, что найдется бесконечный полный подграф этого графа, все ребра
которого покрашены в~один и~тот~же цвет.

\item
Назовем тройку людей \emph{хорошей,} если ее~можно отправить в~поход
(так что люди не~поссорятся) и~\emph{плохой} в~противном случае.
Докажите, что из~бесконечного числа людей можно выбрать 100 человек, так чтобы
любая тройка из~них была хорошей, либо 100 человек, так чтобы любая тройка
из~них была плохой.

\item
В~компании из~$2 n + 1$ человек для любых $n$~человек найдется отличный от~них
человек, знакомый с~каждым из~них.
Докажите, что в~этой компании есть человек, знающий всех.

\item
Пусть в~графе не~менее чем $(3 n - 2)$ вершины и~не~более чем $(3 n - 2)$ ребра
($n \geq 2$).
Тогда найдутся $n$ вершин, между которыми нет ребер.
% http://problems.ru/view_problem_details_new.php?id=64362

\end{problems}

\medskip
\hrule

\begin{problems}

\item
Пусть в~графе более $(m - 1) \cdot (n - 1)$ вершин, но~среди любых $m$~вершин
найдутся две, соединенные ребром.
Тогда в~этом графе встречается любое дерево на~$n$~вершинах.

\item
Дано натуральное число $n \geq 2$.
Рассмотрим все такие покраски клеток доски $n \times n$ в~$k$~цветов, что
каждая клетка покрашена ровно в~один цвет и~все $k$~цветов встречаются.
При каком наименьшем $k$ в~любой такой покраске найдутся четыре окрашенных
в~четыре разных цвета клетки, расположенные в~пересечении двух строк и~двух
столбцов?
% http://problems.ru/view_problem_details_new.php?id=65124

\item
Вершины графа занумерованы натуральными числами от~$1$ до~$10^5$, причем каждое
натуральное число встречается ровно один раз.
Известно, что в~этом графе нет циклов из~четырех вершин.
Докажите, что существует арифметическая прогрессия из~5 не~превосходящих $10^5$
натуральных чисел такая, что никакие две вершины с~номерами из~этой прогрессии
не~соединены ребром.

\end{problems}

