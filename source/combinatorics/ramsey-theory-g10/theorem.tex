% $date: 2015-04-07
% $timetable:
%   g10r2:
%     2015-04-07:
%       1:
%   g10r1:
%     2015-04-07:
%       2:

\section*{Теория Рамсея — 2: числа Рамсея}

% $authors:
% - Владимир Шарич

\claim{Определение чисел Рамсея}
Пусть даны числа $a$, $b$, $k$.
Наименьшее число $n$ такое, что как~бы мы~ни~покрасили все
$k$-элементные подмножества $n$-элементного множества в~черный и~белый цвета,
найдется $a$-элементное подмножество, все $k$-элементные подмножества
которого покрашены в~белый цвет, либо найдется $b$-элементное подмножество,
все $k$-элементные подмножества которого покрашены в~черный цвет, называется
\emph{числом Рамсея} и~обозначается $n = r_k(a, b)$.

\begin{problems}

\item
Найдите
\quad
\subproblem $r_1(a, b)$;
\quad
\subproblem $r_2(3, 4)$.

\item
Докажите неравенство
\[
    r_{k}(a, b)
\leq
    r_{k-1} \bigl( r_{k}(a - 1, b), r_k(a, b - 1) \bigr) + 1
\, . \]
Объясните, как из~этого неравенства следует существование чисел Рамсея для всех
допустимых значений параметров.

\item
Докажите, что
\(
    r_2(a, b) \leq \dbinom{a + b - 2}{a - 1}
\)\;.

\item
Докажите, что $r_2(c, c) \geq 2^{c / 2}$.

\item\claim{Многоцветная теорема Рамсея}
Существует число $r_k(a, b, c, \ldots, z)$.

%\item
%Докажите, что если число вершин в~графе не~меньше $\dbinom{2 n - 2}{n - 1}$,
%то~либо он~сам, либо двойственный к~нему граф содержит полный подграф
%из~$n$~вершин.

\item\claim{Задача Эрдёша}
\subproblem
Докажите, что если любые 4 из~5 точек образуют выпуклый четырехугольник,
то~все 5 образуют выпуклый пятиугольник.
\\
\subproblem
Докажите, что если на~плоскости дано много точек, никакие три из~которых
не~лежат на~одной прямой, то~среди них можно выбрать 5, которые являются
вершинами выпуклого пятиугольника.

\end{problems}

\medskip
\hrule

\begin{problems}

\item\claim{Теорема Хватала}
Обозначим: $T_k$~— дерево на~$k$~вершинах, а~$K_k$~— полный граф
на~$k$~вершинах.
Каждое ребро полного графа на~$r$~вершинах покрашено либо в~красный, либо
в~зеленый цвет.
\\
\subproblem
Докажите, что если $r > (m - 1) \cdot (n - 1)$,
\[
    \text{то}
\quad
    \left[\begin{aligned} &
        \text{есть подграф $T_n$, все ребра которого красные}
    \\ &
        \text{есть подграф $K_m$, все ребра которого зеленые}
    \end{aligned}\right.
\]
\subproblem
Приведите пример, в~котором $r = (m - 1) \cdot (n - 1)$,
\[
    \text{но}
\quad
    \left\{\begin{aligned} &
        \text{нет подграфа $T_n$, все ребра которого красные}
    \\ &
        \text{нет подграфа $K_m$, все ребра которого зеленые}
    \end{aligned}\right.
\]

\item
Вершины бесконечного графа занумерованы всеми натуральными числами, причем
каждое натуральное число является номером ровно одной вершины.
Известно, что в~этом графе нет двух непересекающихся множеств, по~100 вершин
в~каждом, таких, что каждая вершина из~первого множества соединена с~каждой
вершиной из~второго.
Докажите, что существует сколь угодно длинная арифметическая прогрессия такая,
что никакие две вершины с~номерами из~этой прогрессии не~соединены ребром.

\end{problems}

