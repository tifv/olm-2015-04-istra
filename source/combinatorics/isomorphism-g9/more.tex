% $date: 2015-03-30
% $timetable:
%   g9r2:
%     2015-03-30: {}
%   g9r1:
%     2015-03-30: {}

% $matter[-contained,no-header]:
% - verbatim: \section*{Перевод на другой язык (изоморфизм)}
% - .[contained]

\subsection*{Домашнее задание}

% $authors:
% - Александр Шаповалов

% $matter[-no-ashap-link,contained,no-header]:
% - .[no-ashap-link]
% - /_ashap-link

\begin{problems}

\item
Докажите неравенство
\begingroup
\let\oldsqrt\sqrt
\renewcommand\sqrt[1]{\oldsqrt{\mathrlap{\phantom{x_1^2}}{#1}}}
\begin{align*}
    \hspace{5em}&\hspace{-5em}
    \sqrt{
        (x_1 + x_2 + \ldots + x_n)^2 +
        (y_1 + y_2 + \ldots + y_n)^2
    }
\leq\\&{}\leq
    \sqrt{x_1^2 + y_1^2} + \sqrt{x_2^2 + y_2^2}
    + \ldots +
    \sqrt{x_n^2 + y_n^2}
\, . \end{align*}
\endgroup

\item
На~доске написаны в~порядке возрастания два натуральных числа $x$ и~$y$
($x \leq y$).
Петя записывает на~бумажке $x^2$ (квадрат первого числа), а~затем заменяет
числа на~доске числами $x$ и~$y - x$, записывая их~в~порядке возрастания.
С~новыми числами на~доске он~снова проделывает ту~же операцию, и~так далее,
до~тех пор, пока одно из~чисел на~доске не~станет нулем.
Чему будет в~этот момент равна сумма чисел на~Петиной бумажке?

\item
На~доске выписаны числа $1, 1/2, 1/3, \ldots, 1/99$.
За~одну операцию пара выбранных чисел $a$ и~$b$ заменяется на~$a b + b + a$.
После 98 операций осталось одно число.
Какое?

\item
В~противоположных углах шахматной доски стоят белая и~черная фишки.
Ходят по~очереди на~соседнюю по~стороне клетку, начинают белые.
Белые стремятся получить прямоугольный треугольник с~вершинами в~центре доски
и~центрах клеток с~фишками.
Могут~ли черные им~помешать?

\end{problems}

