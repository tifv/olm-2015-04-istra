% $date: 2015-03-30
% $timetable:
%   g9r2:
%     2015-03-30:
%       3:
%   g9r1:
%     2015-03-30:
%       2:

\section*{Перевод на другой язык (изоморфизм)}

% $authors:
% - Александр Шаповалов

% $matter[-no-ashap-link,no-header]:
% - .[no-ashap-link]
% - /_ashap-link

\begin{flushright}
% spell How much is five plus seven
How much is five plus seven?
\end{flushright}

Задача, переведенная на~другой язык, может оказаться гораздо легче.
Не~забудьте только перевести решение обратно!

\begin{problems}

\item
\subproblem
На~пустой шахматной доске двое играющих по~очереди двигают коня.
Двигать можно только влево-вниз.
Кто не~сможет сделать ход~--- проиграл.
Найдите все клетки, начав с~которых первый может выиграть, как~бы хорошо
не~играл противник.
\\
\subproblem
В~двух коробочках лежат орехи, в~каждой не~более семи.
Играют двое.
За~один ход нужно взять три ореха~--- два из~одной коробочки, третий~---
из~другой.
Найдите все позиции, начав с~которых первый может выиграть, как~бы хорошо
не~играл противник.

\end{problems}

Геометрические задачи можно перевести в~алгебраические, введя координаты.
Но~с~помощью координат можно и~алгебраические задачи решать на~геометрическом
языке!

\begin{problems}

\item
Сколько решений может быть у~системы уравнений при различных значениях
параметров $a$, $b$, $c$, $d$, $r$, $R$?

\item
По~прямой в~одном направлении на~некоторых расстояниях (возможно, разных) друг
от~друга движутся 20 одинаковых шариков, а~навстречу им~движутся 20 других
таких~же шариков.
Скорости всех шариков одинаковы.
При столкновении любых двух шариков они разлетаются в~противоположные стороны
с~той~же скоростью, с~какой двигались до~столкновения.
Сколько всего столкновений произойдет между шариками?

\end{problems}

Переводят обычно на~знакомый язык, где начинает работать интуиция!

\begin{problems}

\item
\subproblem
Капитаны Боб и~Иван состязаются в~изготовлении и~употреблении крепких напитков.
Боб сделал коктейль из~рома и~портвейна, а~Иван смешал водку с~брагой.
Известно, что ром крепче водки, а~портвейн крепче браги.
Наутро Ивану было много хуже.
Он~подозревает, что его смесь оказалась крепче коктейля Боба.
Обоснованы~ли подозрения?
(Крепость~--- это процент алкоголя в~смеси.)
\\
\subproblem
Имеется два дома, в~каждом по~два подъезда.
Жильцы держат кошек и~собак, причем доля кошек (отношение числа кошек к~общему
числу кошек и~собак) в~первом подъезде первого дома больше, чем доля кошек
в~первом подъезде второго дома, а~доля кошек во~втором подъезде первого дома
больше, чем доля кошек во~втором подъезде второго дома.
Обязательно~ли доля кошек в~первом доме больше доли кошек во~втором доме?

\item
Следующая игра является переводом на~другой язык одной очень популярной игры.
Какой?
\begin{quote}
На~столе лежат 9 карточек с~числами от~1 до~9.
Двое играющих по~очереди берут по~одной карточке.
Выигрывает тот, кто первым после своего хода сможет выложить три карточки
с~суммой 15.
\end{quote}

\end{problems}

Переводят для того, чтобы обойти препятствие: так, туристы, идущие вдоль берега
и~натолкнувшиеся на~скалы, могут обойти их, временно переправившись на~другой
берег.

\begin{problems}

\item
\subproblem
Из~$N$ прямоугольных плиток (возможно, неодинаковых) составлен прямоугольник,
у~которого одна сторона вдвое больше другой.
Докажите, что можно разрезать каждую плитку на~две части и~части и~разложить
части каждой плитки в~две разные кучки так, чтобы из~всех частей каждой кучки
можно было сложить квадрат.
\subproblem
Из~$N$ прямоугольных плиток (возможно, неодинаковых) составлен прямоугольник
с~неравными сторонами.
Верно~ли, что можно так разрезать каждую плитку на~две части и~разложить части
каждой плитки в~две разные кучки, чтобы из~$N$ частей одной кучки можно было
сложить квадрат, а~из~$N$ частей другой кучки~--- прямоугольник?

\item
\subproblem
На~доске выписаны числа $1, 1/2, 1/3, \ldots, 1/99$.
За~одну операцию пара выбранных чисел $a$ и~$b$ заменяется на~отношение
их~произведения к~их~сумме.
После нескольких операций осталось одно число.
Какое?
\\
\subproblem
Тот~же вопрос для чисел $1, 2, 4, 8, \ldots, 1024$.
\\
\subproblem
Тот~же вопрос для чисел
$1 \cdot 2, 2 \cdot 3, 3 \cdot 4, \ldots, n \cdot (n + 1)$.

\end{problems}

