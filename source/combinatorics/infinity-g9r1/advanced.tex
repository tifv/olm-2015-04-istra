% $date: 2015-04-03
% $timetable:
%   g9r1:
%     2015-04-03: {}

\section*{Конечное и бесконечное: допзадачи для крутых}

% $authors:
% - Александр Шаповалов

% $matter[-no-link,no-header]:
% - .[no-link]
% - /_ashap-link

% $build$matter[print]: [[.], [.], [.], [.]]
% $build$style[print]:
% - .[tiled4,-print]

\begin{problems}

\itemy{КБ5}
Верно ли, что среди сумм вида $1 + 1 / 2^2 + 1 / 3^2 + \ldots + 1 /n^2$ есть
сколь угодно большие?

\itemy{КБ6}
Несколько прямых, никакие две из которых не параллельны, разрезают плоскость
на части.
Внутри одной из этих частей отметили точку $A$.
Доказать, что точка, лежащая с $A$ по разные стороны от всех данных прямых,
существует тогда и только тогда, когда часть, содержащая $A$, неограничена.

\itemy{КБ7}
Известно, что человечество бессмертно, а каждый человек смертен и имеет
конечное число детей.
Докажите, что найдется бесконечная мужская цепочка, начинающаяся с Адама.

\itemy{КБ8}
С центрами в целых точках прямой расположены ямы, шириной $0{,}01$ каждая.
Длина прыжков блохи постоянна и равна $\sqrt{2}$.
Докажите, что блоха рано или поздно попадет в яму.
(Блоха начинает из произвольной точки.)

\end{problems}

