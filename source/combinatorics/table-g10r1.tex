% $date: 2015-04-02
% $timetable:
%   g10r1:
%     2015-04-02:
%       1:

% $caption: Таблицы (комбинаторика)

\section*{Таблицы}

% $authors:
% - Фёдор Бахарев

\begin{problems}

\item
Все клетки квадратной таблицы $n \times n$ пронумерованы в~некотором порядке
числами от~1 до~$n^2$.
Петя делает ходы по~следующим правилам.
Первым ходом он~ставит фишку в~любую клетку.
Каждым последующим ходом Петя может либо поставить новую фишку на~какую-то
клетку, либо переставить фишку из~клетки с~номером~$a$ ходом по~горизонтали или
по~вертикали в~клетку с~номером большим, чем $a$.
Каждый раз, когда фишка попадает в~клетку, эта клетка немедленно закрашивается;
ставить фишку на~закрашенную клетку запрещено.
Какое наименьшее количество фишек потребуется Пете, чтобы независимо
от~исходной нумерации он~смог за~несколько ходов закрасить все клетки таблицы?

\item
64 неотрицательных числа, сумма которых равна 1956, расположены в~форме
квадратной таблицы по~восемь чисел в~каждой строке и~в~каждом столбце.
Сумма чисел, стоящих на~двух диагоналях, равна 112.
Числа, расположенные симметрично относительно любой диагонали, равны.
Докажите, что сумма чисел в~любой строке меньше 518.

\item
В~каждой клетке квадратной таблицы $m \times m$ клеток стоит либо натуральное
число, либо нуль.
При этом, если на~пересечении строки и~столбца стоит нуль, то~сумма чисел
в~<<кресте>>, состоящем из~этой строки и~этого столбца, не~меньше $m$.
Докажите, что сумма всех чисел в~таблице не~меньше, чем $m^2 / 2$.

\item
В~каждой клетке квадратной таблицы написано по~числу.
Известно, что в~каждой строке таблицы сумма двух наибольших чисел равна $a$,
а~в~каждом столбце таблицы сумма двух наибольших чисел равна $b$.
Докажите, что $a = b$.

\item
Клетки таблицы $100 \times 100$ окрашены в~4 цвета так, что в~каждой строке
и~в~каждом столбце ровно по~25 клеток каждого цвета.
Докажите, что найдутся две строки и~два столбца, все четыре клетки
на~пересечении которых окрашены в~разные цвета.

\item
В~таблице размером $m \times n$ записаны числа так, что для каждых двух строк
и~каждых двух столбцов сумма чисел в~двух противоположных вершинах образуемого
ими прямоугольника равна сумме чисел в~двух других его вершинах.
Часть чисел стерли, но~по~оставшимся можно восстановить стертые.
Докажите, что осталось не~менее $(n + m - 1)$ чисел.

\item
В~таблице $N \times N$, заполненной числами, все строки различны (две строки
называются различными, если они отличаются хотя~бы в~одном элементе).
Докажите, что из~таблицы можно вычеркнуть некоторый столбец, так что
в~оставшейся таблице опять все строки будут различны.

\item
\subproblem
Существует~ли последовательность натуральных чисел $a_1, a_2, a_3, \ldots$,
обладающая следующим свойством: ни~один член последовательности не~равен сумме
нескольких других и~$a_n \leq n^{10}$ при любом $n$?
\\
\subproblem
Тот~же вопрос, если $a_n \leq n \sqrt{n}$ при любом $n$.
% http://www.problems.ru/view_problem_details_new.php?id=79370

\end{problems}

