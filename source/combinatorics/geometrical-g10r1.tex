% $date: 2015-04-08
% $timetable:
%   g10r1:
%     2015-04-09:
%       1:

\section*{Комбинаторная геометрия}

% $authors:
% - Фёдор Бахарев

\begin{problems}

\item
В квадрате со стороной 15 расположено 20 попарно непересекающихся квадратиков
со стороной~1.
Докажите, что в большом квадрате можно разместить круг радиуса~1 так, чтобы
он не пересекался ни с одним из квадратиков.

\item
В круге радиуса 16 расположено 650 точек.
Докажите, что найдется кольцо с внутренним радиусом 2 и внешним радиусом 3,
в котором лежат не менее 10 из данных точек.

\item
На окружности отметили $n$~точек, разбивающие ее на $n$~дуг.
Окружность повернули вокруг центра на угол $\pi k / n$ (при некотором
натуральном~$k$), в результате чего отмеченные точки перешли в $n$ новых точек,
разбивающих окружность на $n$ новых дуг.
Докажите, что найдется новая дуга, которая целиком лежит в одной из старых дуг.
(Считается, что концы дуги ей принадлежат.)

\item
На координатной плоскости нарисовано $n$~парабол, являющихся графиками
квадратных трехчленов, никакие две из них не касаются.
Они делят плоскость на несколько областей, одна из которых расположена над
всеми параболами.
Докажите, что у границы этой области не более $2 (n - 1)$ углов (то есть точек
пересечения пары парабол).

\item
На доске нарисован выпуклый $2015$-угольник.
Петя последовательно проводит в нем диагонали так, чтобы каждая вновь
проведенная диагональ пересекала по внутренним точкам не более одной
из проведенных ранее диагоналей.
Какое наибольшее количество диагоналей может провести Петя?

\item
На плоскости отмечено несколько точек, каждая покрашена в синий, желтый или
зеленый цвет.
На любом отрезке, соединяющем одноцветные точки, нет точек этого же цвета,
но есть хотя бы одна другого цвета.
Каково максимально возможное число всех точек?

\item
На плоскости нарисовано несколько прямоугольников со сторонами, параллельными
осям координат.
Известно, что каждые два прямоугольника можно пересечь вертикальной или
горизонтальной прямой.
Докажите, что можно провести одну горизонтальную и одну вертикальную прямую
так, чтобы любой прямоугольник пересекался хотя бы с одной из этих двух прямых.

\end{problems}

