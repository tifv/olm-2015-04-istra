% $groups$delegate: false
% $groups$delegate$into: false
% $groups$matter: false
% $groups$matter$into: false

% $matter[-header,-no-header]:
% - .[no-header]

% $matter[-matter-guard,no-header]:
% - verbatim: \begingroup\let\ifsourcelinks\iftrue
%   condition: source-link
% - .[matter-guard]
% - verbatim: \endgroup
%   condition: source-link

\begingroup
\providecommand\ifsourcelinks{\iffalse}
\providecommand\url{\texttt}

\strut

\vfill

\begin{center}
\scalebox{2}{\(\displaystyle
\int\limits_{\text{30 марта 2015}}^{\text{10 апреля 2015}}
    \dbinom{\text{\ Московские сборы\ }}{\text{\ секция математики\ }}
    \, \mathrm{d} t
\)}
\end{center}

\vfill

\strut

\clearpage


\subsection*{Немного о~группах}

Занятия проходили в шести группах:
\begingroup\multicolsep=\parskip
\begin{multicols}{3}
9-2  <<Белые Ферзи>>
\\
9-1  <<Чёрные Ферзи>>
\\
10-2 <<Белые Ладьи>>
\\
10-1 <<Чёрные Ладьи>>
\\
11-2 <<Белые Слоны>>
\\
11-1 <<Чёрные Слоны>>
\end{multicols}
\endgroup

Группы \mbox{9-*} были собраны из~школьников, обучающихся в~9 классе;
аналогично для \mbox{10-*} и~\mbox{11-*}.

Группы \mbox{*-1} были собраны из~предположительно более сильных школьников,
и~задачи там в среднем сложнее, чем в~соответствующих группах \mbox{*-2}.


\subsection*{Немного о~структуре}

Материалы разбиты по~группам, в~пределах каждой группы отсортированы по~темам:
\begin{itemize}
    \item \emph{тренировочные олимпиады;}
    \item алгебра;
    \begin{itemize}
        \item теория чисел;
        \item многочлены;
        \item неравенства;
    \end{itemize}
    \item геометрия;
    \item комбинаторика;
    \begin{itemize}
        \item теория графов.
    \end{itemize}
\end{itemize}

Материалы, общие для нескольких групп, дублируются.
\ifsourcelinks
Все материалы сопровождаются ссылками на~исходные файлы \LaTeX.
\fi


\subsection*{Немного об~авторах}

Материалы Аркадия Борисовича Скопенкова не~представлены в~сборнике.
Их~обновляемые версии можно найти по~следующим ссылкам:
\\
\url{http://www.mccme.ru/circles/oim/discrbook.pdf}
\\
\url{http://www.mccme.ru/circles/oim/exalong.pdf}
\\
\url{http://www.mccme.ru/circles/oim/algor.pdf}

Материалы Александра Васильевича Шаповалова, представленные в~этом сборнике,
можно также найти (вместе с~его материалами предыдущих лет) по~ссылке:
\\
\url{http://www.ashap.info/Uroki/Mosbory}

\endgroup

