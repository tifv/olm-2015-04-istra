% $date: 2015-04-01
% $timetable:
%   g11r1:
%     2015-04-01:
%       1:
%       2:

\section*{Тренировочная олимпиада}

% $authors:
% - Андрей Кушнир
% - Олег Орлов

\begin{problems}

% spell проход\'{и}м

\item
\emph{Лабиринтом} назовем расстановку перегородок между некоторыми парами
соседних по~стороне клеток шахматной доски.
Лабиринт \emph{проход\'{и}м}, если из~любой его клетки можно попасть в~любую
другую, перемещаясь каждый раз в~соседнюю по~стороне клетку и~не~пересекая
перегородок.
Каких лабиринтов больше, проходимых или непроходимых?

\item
Существуют~ли $2015$ непересекающихся непостоянных арифметических прогрессий
натуральных чисел, таких что каждая из~них содержит простое число,
превосходящее $2015$, и~лишь конечное число натуральных чисел в~них не~лежит?

\item
В~остроугольном неравнобедренном треугольнике $ABC$ отметили середины
$C_1$, $B_1$, $A_1$ сторон $AB$, $AC$, $BC$ соответственно.
Серединные перпендикуляры к~$AB$ и~$AC$ пересекают $A A_1$ в~точках
$B_2$, $C_2$ соответственно.
Прямые $B B_2$ и~$C C_2$ пересекаются в~точке~$X$, лежащей внутри треугольника.
Докажите, что точки $A$, $B_1$, $C_1$, $X$ лежат на~одной окружности.

\item
Докажите, что существует многочлен $P(x, y, z)$ степени не~более $3$, задающий
биекцию $P \colon \mathbb{N}^3 \to \mathbb{N}$.

\item
На~плоскости даны $n$~точек общего положения.
Докажите, что можно вбить $(2 n - 5)$ гвоздей так, чтобы в~каждый треугольник
с~вершинами в~этих точках был~бы вбит хотя~бы один гвоздь (запрещено вбивать
в~гвозди в~прямые, соединяющие исходные точки).

\end{problems}

