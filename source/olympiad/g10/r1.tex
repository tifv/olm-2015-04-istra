% $date: 2015-04-04
% $timetable:
%   g10r1:
%     2015-04-04:
%       1:
%       2:

\section*{Тренировочная олимпиада}

% $authors:
% - Алексей Доледенок
% - Николай Крохмаль
% - Андрей Меньщиков

\begin{problems}

\item
Бесконечная последовательность $a_0, a_1, \ldots$ натуральных чисел такова, что
\(
    a_{n}^2 > a_{n-1} \cdot a_{n+1}
\)
при любом натуральном~$n$.
Докажите, что $a_k > k$ при любом натуральном~$k$.

\item
Точка~$I$~--- центр вписанной окружности треугольника $ABC$.
Точка~$M$ на~прямой~$BC$ такова, что $MI \perp AI$.
Точка~$D$~--- основание перпендикуляра из~$I$ на~$AM$.
Докажите, что точки $A$, $B$, $C$, $D$ лежат на~одной окружности.

\item
Имеется 100 образцов, среди которых ровно 50 радиоактивных.
Есть прибор, в~который за~одну операцию можно положить два образца, и~если
ровно один из~них радиоактивен, то~прибор укажет на~него, а~если нет,
то~на~любой из~двух образцов.
За~какое наименьшее количество операций можно найти хотя~бы один радиоактивный
образец?

\item
Докажите, что функция $f(n) = (n^{2015} - n!)$ в~разных натуральных точках
принимает разные значения.

\end{problems}

