% $date: 2015-04-04
% $timetable:
%   g10r2:
%     2015-04-04:
%       1:
%       2:

\section*{Тренировочная олимпиада}

% $authors:
% - Алексей Доледенок
% - Николай Крохмаль
% - Андрей Меньщиков

\begin{problems}

\item
Бесконечная последовательность $a_0, a_1, \ldots$ натуральных чисел такова, что
\(
    a_{n}^2 > a_{n-1} \cdot a_{n+1}
\)
при любом натуральном~$n$.
Докажите, что $a_k > k$ при любом натуральном~$k$.

\item
Точка~$I$~— центр вписанной окружности треугольника $ABC$.
Точка~$M$ на~прямой~$BC$ такова, что $MI \perp AI$.
Точка~$D$~— основание перпендикуляра из~$I$ на~$AM$.
Докажите, что точки $A$, $B$, $C$, $D$ лежат на~одной окружности.

\item
На~доске написано число~0.
Два игрока по~очереди приписывают справа к~выражению на~доске:
первый~— знак «$+$» или «$-$», второй~— одно из~натуральных чисел
от~1 до~2015.
Игроки делают по~2015 ходов, причем второй записывает каждое число от~1 до~2015
ровно по~одному разу.
В~конце игры второй игрок получает выигрыш, равный модулю алгебраической суммы,
написанной на~доске.
Какой наибольший выигрыш он~может себе гарантировать?

\item
Найдите все пары $(a, b)$ натуральных чисел такие, что при любом
натуральном~$n$ число $a^n + b^n$ является точной $(n + 1)$-й степенью.

\end{problems}

