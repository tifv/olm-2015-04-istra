% $date: 2015-04-03
% $timetable:
%   g9r1:
%     2015-04-03:
%       1:
%       2:

\section*{Тренировочная олимпиада --- 1}

% $authors:
% - Александр Блинков
% - Александр Шаповалов

% $build$matter[print]: [[.], [.], [.], [.]]
% $build$style[print]:
% - .[tiled4,-print]

\begin{problems}

\item
Ученику дано число $x$, записанное как обыкновенная дробь с~однозначным
знаменателем.
Числа $2 x$, $4 x$ и~$5 x$ оказались не~целыми и~не~полуцелыми.
Он~округлил каждое из~этих трех чисел до~ближайшего целого и~результаты сложил.
Получилось 120.
Найдите $x$.

\item
Найдите все такие натуральные числа, у~которых самый большой собственный
делитель на~50 больше квадрата самого маленького собственного делителя.
(\emph{Собственными} называются все натуральные делители числа кроме него самого
и~единицы).

\item
Можно~ли на~доску $2015 \times 2015$ выставить несколько ладей так, что каждое
пустое поле и~каждая ладья была побита одинаковым числом ладей?
(Ладья бьет поле или другую ладью на~той~же вертикали или горизонтали если
между ними нет других фигур.
Ладья себя не~бьет.)

\item
Внутри треугольника отмечена точка.
Докажите, что сумма расстояний от~нее до~вершин треугольника не~превосходит
суммы двух наибольших сторон.

\end{problems}

