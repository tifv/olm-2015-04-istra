% $date: 2015-04-03
% $timetable:
%   g9r1:
%     2015-04-03:
%       1:
%       2:

\section*{Тренировочная олимпиада --- 1}

% $authors:
% - Александр Блинков
% - Александр Шаповалов

% $build$matter[print]: [[.], [.], [.], [.]]
% $build$style[print]:
% - .[tiled4,-print]

% $matter[with-solutions]:
% - verbatim: \begingroup\let\ifincludesolutions\iftrue
% - .[-with-solutions]
% - verbatim: \endgroup % \let\ifincludesolutions

\begingroup\providecommand\ifincludesolutions{\iffalse}

\ifincludesolutions
\subsection*{Версия с решениями}
\fi

\begin{problems}

\item
Ученику дано число $x$, записанное как обыкновенная дробь с~однозначным
знаменателем.
Числа $2 x$, $4 x$ и~$5 x$ оказались не~целыми и~не~полуцелыми.
Он~округлил каждое из~этих трех чисел до~ближайшего целого и~результаты сложил.
Получилось 120.
Найдите $x$.
\begingroup\em\small\par\strut\hfill
    А.\,Шаповалов
\endgroup
\end{problems}

\ifincludesolutions
\emph{Ответ:} $10 \frac{8}{9}$.
\par
Давайте число $a$ после округления обозначать $R(a)$, а сумму
$R(2 x) + R(4 x) + R(5 x)$ обозначим $S$.
Докажем, что $10 \frac{6}{7} < x < 11$.
Воспользуемся тем, что если $a \leq b$, то и $R(a) \leq R(b)$.
Если $x \leq 10 \frac{6}{7}$,
то \(
    S
\leq
    R(21 \frac{5}{7}) + R(43\frac{3}{7}) + R(54 \frac{2}{7})
=
    22 + 43 + 54
=
    119
<
    120
\), а при $x \geq 11$
\(
    S
\geq
    R(22) + R(44) + R(55)
=
    22 + 44 + 55
=
    121 > 120
\).
Значит, $x = 11 - y$, где $0 < y < 1 / 7$ и записывается как правильная дробь
с однозначным знаменателем.
Значит, $y = 1 / 8$ или $y = 1 / 9$.
Первое невозможно, так как тогда $4 x = 43{,}5$~--- полуцелое число.
Значит, $y = 1 / 9$.
Отсюда ответ $x = 10 \frac{8}{9}$, который, как легко проверить, подходит.
\fi % \ifincludesolutions

\begin{problems}

\item
Найдите все такие натуральные числа, у~которых самый большой собственный
делитель на~50 больше квадрата самого маленького собственного делителя.
(\emph{Собственными} называются все натуральные делители числа кроме него самого
и~единицы).
\begingroup\em\small\par\strut\hfill
    А.\,Шаповалов, по мотивам задачи олимпиады им.\,Кукина
\endgroup
\end{problems}

\ifincludesolutions
\emph{Ответ:} 108 и 177.
\par
Пусть $x$~--- наименьший собственный делитель числа $N$, а $y$~--- наибольший.
Ясно, что $y = N / x$ , и поэтому $N = x y$.
Очевидно также, что число $x$~--- простое.
Рассмотрим 2 случая.
\\
\emph{Случай 1.} $x \neq 3$.
Тогда $x^2$ дает остаток 1 при делении на 3.
Поэтому $y = x^2 + 50$ делится на 3.
Тогда и у $N$ есть собственный делитель 3.
Так как $x$~--- наименьший собственный делитель, $x < 3$.
Значит, $x = 2$.
Тогда $y = 2^2 + 50 = 54$, $N = 2 \cdot 54 = 108$.
\\
\emph{Случай 2.} $x = 3$.
Тогда $y = 3^2 + 50 = 59$, $N = 3 \cdot 59 = 177$.
\fi % \ifincludesolutions

\begin{problems}

\item
Можно~ли на~доску $2015 \times 2015$ выставить несколько ладей так, что каждое
пустое поле и~каждая ладья была побита одинаковым числом ладей?
(Ладья бьет поле или другую ладью на~той~же вертикали или горизонтали если
между ними нет других фигур.
Ладья себя не~бьет.)
\begingroup\em\small\par\strut\hfill
    А.\,Шаповалов
\endgroup
\end{problems}

\ifincludesolutions
\emph{Ответ:} нет.
\par
Пусть каждое поле и ладья побиты $k$ ладьями.
Ясно, что $k > 0$.
Кроме того, $k \leq 2$: угловое поле побито не более чем двумя ладьями.
Если $k = 1$, две бьющие друг друга ладьи не бьют всю доску, а третья ладья
побьет кого-то из них либо уже побитое поле.
Противоречие.
Значит, $k \neq 1$, поэтому $k = 2$.
\par
Докажем, что на каждой горизонтали или вертикали стоит не больше двух ладей.
Допустим противное: ладьи $X$, $Y$ и $Z$ стоят на одной горизонтали
в вертикалях $x$, $y$, $z$ соответственно, $y$ между $x$ и $z$.
Тогда на вертикали $y$ других ладей нет.
Ладьи $X$, $Y$ и $Z$ входят в <<цикл>> бьющих друг друга ладей.
Обходя цикл, мы пересечем вертикаль y еще раз.
Поле на пересечении побито трижды.
Противоречие.
\par
Если есть пустая вертикаль, то каждое поле на ней побито дважды по горизонтали.
Значит, на каждой горизонтали ровно две ладьи, и всего их 4030.
Но тогда занятых вертикалей меньше 2015, поэтому на некоторой вертикали стоит
более двух ладей.
Противоречие.
Если есть вертикаль с ровно одной ладьей, то эта ладья дважды побита
по горизонтали~--- но трех ладей на горизонтали быть не может.
Противоречие.
\par
Остался случай, когда на каждой вертикали и каждой горизонтали стоит ровно
по две ладьи.
Допустим, две бьющие друг друга ладьи стоят не рядом (скажем, на одной
верикали).
Тогда между ними есть пустое поле.
Оно уже побито дважды по вертикали, и еще хотя бы раз по горизонтали.
Противоречие.
\par
Рассмотрим цикл из бьющих друг друга ладей.
Две самые верхние ладьи в нем стоят рядом, а ладьи в тех же вертикалях стоят
на одну клетку ниже.
Значит, цикл состоит из этих четырех ладей, и это верно для всех циклов.
Поэтому общее число ладей кратно 4.
Однако на доске $2015 \times 2015$ их должно быть $2 \cdot 2015$, что
не кратно 4.
Противоречие.
\fi % \ifincludesolutions

\begin{problems}

\item
\ifincludesolutions\label{olympiad/g9-1/r1:n4:problem}\fi
Внутри треугольника отмечена точка.
Докажите, что сумма расстояний от~нее до~вершин треугольника не~превосходит
суммы двух наибольших сторон.
\begingroup\em\small\par\strut\hfill
    Н.\,Седракян, 3-я Устная олимпиада по геометрии, 2005\,г.
\endgroup
\end{problems}

\ifincludesolutions
\begin{figure}[ht]\begin{center}
    \jeolmfigure[width=0.5\textwidth]{n4-solution}
    \caption{к решению задачи~\ref{olympiad/g9-1/r1:n4:problem}}%
    \label{olympiad/g9-1/r1:n4:solution:fig}
\end{center}\end{figure}
Пусть стороны треугольника $a \leq b \leq c$, а отрезки из точки к вершинам~---
$k$, $l$ и $m$ (рис.~\ref{olympiad/g9-1/r1:n4:solution:fig}).
Проведем через отмеченную точку отрезок, параллельный стороне $a$.
Мы отсекли подобный треугольник со сторонами $a' \leq b' \leq c'$.
Обозначим отрезки, на которые разделились стороны, как на рисунке.
Тогда $a' = p + q$, $b = b' + b''$, $c = c' + c''$.
По неравенству треугольника $k < b'' + p$, $l < q + c''$.
Кроме того, чевиана $m$ не превосходит большей из боковых сторон отсеченного
треугольника: $m \leq c'$.
Поэтому
\begin{align*}
    k + l + m
<{}&
    (b'' + p) + (q + c'') + c'
=
    (p + q) + b'' + (c' + c'')
=\\={}&
    a' + b'' + c
\leq
    b' + b'' + c
=
    b + c
\,.\end{align*}
\fi % \ifincludesolutions

\endgroup % \providecommand\ifincludesolutions

