% $date: 2015-04-03
% $timetable:
%   g9r2:
%     2015-04-03:
%       1:
%       2:

\section*{Тренировочная олимпиада --- 1}

% $authors:
% - Александр Блинков
% - Александр Шаповалов

% $build$matter[print]: [[.], [.], [.], [.]]
% $build$style[print]:
% - .[tiled4,-print]

% $matter[with-solutions]:
% - verbatim: \begingroup\let\ifincludesolutions\iftrue
% - .[-with-solutions]
% - verbatim: \endgroup % \let\ifincludesolutions

\begingroup\providecommand{\ifincludesolutions}{\iffalse}

\newenvironment{problem-source}{\em\small\par\strut\hfill}{}

\begin{problems}

\item
Найдите все такие натуральные числа, у~которых самый большой собственный
делитель на~55 больше самого маленького собственного делителя.
(\emph{Собственными} называются все натуральные делители числа кроме него самого
и~единицы).
\begin{problem-source}
    Олимпиада им.\,Кукина
\end{problem-source}
\end{problems}

\ifincludesolutions
\emph{Ответ:} 114.
\par
Пусть $x$~--- наименьший собственный делитель числа~$N$.
Числа $x$ и $x + 55$~--- разной четности, поэтому одно из них четно.
Оба числа~--- делители $N$, значит, $N$~--- четно.
Поэтому наименьший собственный делитель $N$ равен 2.
Так как $x = 2$, то наибольший собственный делитель равен $2 + 55 = 57$.
Ясно, что наибольший собственный делитель равен $N / x$.
Поэтому $N = 2 \cdot 57 = 114$.
\fi % \ifincludesolutions

\begin{problems}

\item
Ученику дано число $x$, записанное как обыкновенная дробь с~однозначным
знаменателем.
Числа $2 x$, $4 x$ и~$5 x$ оказались не~целыми и~не~полуцелыми.
Он~округлил каждое из~этих трех чисел до~ближайшего целого и~результаты сложил.
Получилось 120.
Найдите $x$.
\begin{problem-source}
    А.\,Шаповалов
\end{problem-source}
\end{problems}

\ifincludesolutions
\emph{Ответ:} $10 \frac{8}{9}$.
\par
Давайте число $a$ после округления обозначать $R(a)$, а сумму
$R(2 x) + R(4 x) + R(5 x)$ обозначим $S$.
Докажем, что $10 \frac{6}{7} < x < 11$.
Воспользуемся тем, что если $a \leq b$, то и $R(a) \leq R(b)$.
Если $x \leq 10 \frac{6}{7}$,
то \(
    S
\leq
    R(21 \frac{5}{7}) + R(43\frac{3}{7}) + R(54 \frac{2}{7})
=
    22 + 43 + 54
=
    119
<
    120
\), а при $x \geq 11$
\(
    S
\geq
    R(22) + R(44) + R(55)
=
    22 + 44 + 55
=
    121 > 120
\).
Значит, $x = 11 - y$, где $0 < y < 1 / 7$ и записывается как правильная дробь
с однозначным знаменателем.
Значит, $y = 1 / 8$ или $y = 1 / 9$.
Первое невозможно, так как тогда $4 x = 43{,}5$~--- полуцелое число.
Значит, $y = 1 / 9$.
Отсюда ответ $x = 10 \frac{8}{9}$, который, как легко проверить, подходит.
\fi % \ifincludesolutions

\begin{problems}

\item
На~сторонах $AB$, $BC$ и~$AC$ треугольника $ABC$ выбраны
точки $C'$, $A'$ и~$B'$ соответственно так, что угол $A'C'B'$~--- прямой.
Докажите, что отрезок $A'B'$ длиннее диаметра вписанной окружности
треугольника $ABC$.
\begin{problem-source}
    М.\,Волчкевич, 4-я устная олимпиада по геометрии, 2006\,г.
\end{problem-source}
\end{problems}

\ifincludesolutions
Пусть $A'B' = 2 m$.
Надо доказать, что $m > r$~--- радиуса вписанной окружности.
Отметим середину~$M$ отрезка~$A'B'$.
Тогда $MA' = MB' = MC' = m$.
Перпендикуляры, опущенные из $M$ на стороны треугольника $ABC$,
не превосходят $m$.
Точка~$M$ лежит внутри треугольника $ABC$.
Соединим $M$ с вершинами $A$, $B$, $C$.
\[
    S_{ABC}
=
    S_{ABM} + S_{ACM} + S_{BCM}
\leq
    \frac{AB \cdot m}{2}
    +
    \frac{AC \cdot m}{2}
    +
    \frac{BC \cdot m}{2}
=
    p \cdot m
\,,\]
где $p$~--- полупериметр треугольника.
Равенство достигается только в случае, когда все расстояния до сторон
равны~$m$.
Но тогда все отрезки $MA'$, $MB'$ и $MC'$ перпендикулярны сторонам.
Это невозможно, так как тогда стороны $AC$ и $BC$ были бы обе перпендикулярны
$A'B'$, то есть параллельны.
Значит, $S_{ABC} < p \cdot m$.
Но $S_{ABC} = p \cdot r$.
Отсюда $p \cdot m > p \cdot r$ и $m > r$.
\fi % \ifincludesolutions

\begin{problems}

\item
На~доске $50 \times 50$ выставлены более десяти ладей так, что каждое пустое
поле и~каждая ладья побита одинаковым числом ладей.
Сколько всего ладей?
(Ладья бьет поле или другую ладью на~той~же вертикали или горизонтали, если
между ними нет других фигур.
Ладья себя не~бьет.)
\begin{problem-source}
    А.\,Шаповалов
\end{problem-source}
\end{problems}

\ifincludesolutions
\emph{Ответ:} 100.
\par
Пусть каждое поле и ладья побиты $k$ ладьями.
Ясно, что $k > 0$.
Кроме того, $k \leq 2$: угловое поле побито не более чем двумя ладьями.
Если $k = 1$, две бьющие друг друга ладьи не бьют всю доску, а третья ладья
побьет кого-то из них либо уже побитое поле.
Противоречие.
Значит, $k \neq 1$, поэтому $k = 2$.
\par
Докажем, что на каждой горизонтали или вертикали стоит не больше двух ладей.
Допустим противное: ладьи $X$, $Y$ и $Z$ стоят на одной горизонтали
в вертикалях $x$, $y$, $z$ соответственно, $y$ между $x$ и $z$.
Тогда на вертикали $y$ других ладей нет.
Ладьи $X$, $Y$ и $Z$ входят в <<цикл>> бьющих друг друга ладей.
Обходя цикл, мы пересечем вертикаль y еще раз.
Поле на пересечении побито трижды.
Противоречие.
\par
Если есть пустая вертикаль, то каждое поле на ней побито дважды по горизонтали.
Значит, на каждой горизонтали ровно две ладьи, и всего их 100.
Если есть вертикаль с ровно одной ладьей, то эта ладья дважды побита по горизонтали – но трех ладей на горизонтали быть не может. Противоречие.
Остался случай, когда на каждой вертикали ровно по две ладьи, и тогда всего
их 100.
\par
\emph{Замечание.}
Хотя это и не требуется для решения задачи, такая расстановка ладей существует:
выделим вдоль главной диагонали 25 квадратов $2 \times 2$ и заполним
их ладьями.
\fi % \ifincludesolutions

