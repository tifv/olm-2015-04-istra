% $date: 2015-04-09
% $timetable:
%   g9r1:
%     2015-04-09:
%       1:
%       2:

\section*{Тренировочная олимпиада --- 2}

% $authors:
% - Фёдор Бахарев
% - Владимир Брагин
% - Александр Шаповалов

\begin{problems}

\item
Пусть $f(x) = x^2 + p x + q$.
Этот квадратный трехчлен имеет два корня, причем ровно один из~них лежит
в~интервале от~0 до~1.
Докажите, что $f(q) < 0$.

\item
Хорда~$CD$ окружности с~центром~$O$ перпендикулярна ее~диаметру~$AB$,
а~хорда~$AE$ делит пополам радиус~$OC$.
Докажите, что хорда~$DE$ делит пополам хорду~$BC$.

\item
На~плоскости отмечены 125 точек.
Если выкинуть любую точку, то~остальные можно зачеркнуть 11-ю прямыми.
Докажите, что все точки можно зачеркнуть 11-ю прямыми.

\item
Чему равна сумма всевозможных произведений четного количества дробей
\[
    \frac{1}{2}, \frac{1}{3}, \ldots, \frac{1}{100}
\,?\]
(В каждом произведении все дроби различны.)

\end{problems}

