% $date: 2015-04-09
% $timetable:
%   g9r2:
%     2015-04-09:
%       1:
%       2:

\section*{Тренировочная олимпиада --- 2}

% $authors:
% - Фёдор Бахарев
% - Владимир Брагин
% - Александр Шаповалов

\begin{problems}

\item
Положительные числа $a$, $b$, $c$ таковы, что система из~двух уравнений
$a x + b = c$ и~$x / a + 1 / b = 1 / c$ имеет решение.
Докажите, что
\[
    a / (a + b) + b / (a + 2 c) = a / (a + c) + c / (a + 2 b)
\,.\]

\item
Хорда~$CD$ окружности с~центром~$O$ перпендикулярна ее~диаметру~$AB$,
а~хорда~$AE$ делит пополам радиус~$OC$.
Докажите, что хорда~$DE$ делит пополам хорду~$BC$.

\item
Натуральные числа $m$ и~$n$ и~целое число $k$ таковы, что
$(k + m) \cdot (k + n) = k + m + n$.
Докажите, что $m < 2 n$.

\item
На~плоскости отмечены $58$ точек.
Если выкинуть любую точку, то~остальные можно зачеркнуть семью прямыми.
Докажите, что все точки можно зачеркнуть семью прямыми.

\end{problems}

