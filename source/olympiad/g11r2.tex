% $date: 2015-04-01
% $timetable:
%   g11r2:
%     2015-04-01:
%       1:
%       2:

\section*{Тренировочная олимпиада}

% $authors:
% - Андрей Кушнир
% - Олег Орлов

\begin{problems}

\item
Пусть $M$~— конечное множество чисел.
Известно, что среди любых трех его элементов найдутся два, сумма которых
принадлежит $M$.
Какое наибольшее число элементов может быть в~$M$?

\item
Совершенное число, большее $6$, делится на~$3$.
Докажите, что оно делится на~$9$.
(Натуральное число называется \emph{совершенным,} если оно равно сумме всех
своих делителей, отличных от~самого числа; например, $6 = 1 + 2 + 3$.)

\item
Дан прямоугольный треугольник $ABC$, где $\angle{A} = 90^{\circ}$.
В~нем проведены медианы $B B_1$ и~$C C_1$, которые пересекаются в~точке~$M$.
На~лучах $B B_1$ и~$C C_1$ за~точки $B_1$ и~$C_1$ взяты такие точки
$B_2$ и~$C_2$ соответственно, что $\angle{C C_2 A} = \angle{ACB}$
и~$\angle{B B_2 A} = \angle{ABC}$.
Докажите, что окружности, описанные вокруг треугольников $C B_2 M$ и~$B C_2 M$,
пересекаются на~прямой~$BC$ (в~точке, отличной от~$M$).

\item
На~прямоугольном столе лежат равные картонные квадраты $n$ различных цветов
со~сторонами, параллельными сторонам стола.
Если рассмотреть любые $n$~квадратов различных цветов, то~какие-нибудь два
из~них можно прибить к~столу одним гвоздем.
Докажите, что все квадраты некоторого цвета можно прибить к~столу
$(2 n - 2)$ гвоздями.

\end{problems}

