\section*{Геометрия}

% $authors:
% - Андрей Кушнир

% $matter[with-solutions]:
% - verbatim: \begingroup\let\ifincludesolutions\iftrue
% - .[-with-solutions]
% - verbatim: \endgroup % \let\ifincludesolutions

\begingroup\providecommand\ifincludesolutions{\iffalse}

\ifincludesolutions
\subsection*{Версия с решениями}
\fi

\begin{problems}

\item
В~остроугольном треугольнике $ABC$ точка~$O$~— центр описанной окружности,
а~точки $A_1$, $B_1$, $C_1$~— середины сторон $BC$, $CA$, $AB$
соответственно.
Окружности с~центрами в~$A_1$, $B_1$, $C_1$, проходящие через $O$, пересекаются
в~точках $D$, $E$, $F$, отличных от~$O$.
Докажите, что $O$~— центр вписанной окружности треугольника $DEF$.

\end{problems}

\ifincludesolutions
Пусть $D$~— отличная от~$O$ точка пересечения окружностей $\omega_B$
и~$\omega_C$ с~центрами $B_1$ и~$C_1$ из~условия;
$E$~— $\omega_C$ и~$\omega_B$; аналогично $F$.
Достаточно показать, что $DO$~— биссектриса угла $EDF$
(прямые $EO$ и~$FO$ также будут биссектрисами своих углов и~$O$ будет точкой
пересечения биссектрис).
Заметим, что $\angle EDO = \frac{1}{2} \angle E C_1 O = \angle O C_1 B_1$.
Первое равенство~— соотношение вписанного и~центрального углов окружности
$\omega_C$, второе вытекает из~равенства треугольников
$E C_1 B_1$ и~$O C_1 B_1$.
Аналогично $\angle FDO = \frac{1}{2} \angle F A_1 O = \angle O A_1 B_1$.
Общеизвестно, что центр~$O$ описанной окружности треугольника $ABC$ является
также точкой пересечения высот треугольника $A_1B_1C_1$, а~значит $\angle
OC_1B_1 = 90^{\circ} - \angle C_1 B_1 A_1 = \angle O A_1 B_1$.
Из~выписанных равенств следует, что $\angle EDO = \angle FDO$ и~что прямая~$DO$
действительно является биссектрисой угла $EDF$.
\fi % \ifincludesolutions

\begin{problems}

\item
Вписанная в~неравнобедренный треугольник $ABC$ окружность с~центром~$I$
касается его сторон $AB$, $BC$, $CA$ в~точках $C_1$, $A_1$, $B_1$
соответственно.
На~прямой~$BC$ выбраны такие точки $X$ и~$Y$, что четырехугольники
$X C_1 B_1C$ и $Y B_1 C_1B$~— вписанные.
Прямые $X C_1$ и~$Y B_1$ пересекаются в~точке~$Z$.
Докажите, что $Z$, $I$, $A_1$ лежат на~одной прямой.

\end{problems}

\ifincludesolutions
Из~вписанности четырехугольников из~условия и~равнобедренности треугольника
$B_1 A C_1$ следуют равенства углов:
$\angle CXZ = \angle C_1 B_1 A = \angle B_1 C_1 A = \angle BYZ$.
Следовательно, треугольник $XYZ$ равнобедренный, $XZ = YZ$.
Из~сумм углов треугольников $XYZ$ и~$C_1 B_1 A$ следует, что
$C_1 Z B_1 = C_1 A B_1$, а~значит точки $A$, $C_1$, $B_1$, $Z$ лежат на~одной
окружности.
Заметим, что $I$ тоже лежит на~этой окружности
($\angle A B_1 I = \angle A C_1 I = 90^{\circ}$).
Более того, $I$~— середина дуги~$B_1 C_1$, из~чего вытекает,
что $ZI$~— биссектриса угла $B_1 Z C_1$.
Получили, что $ZI$~— биссектриса равнобедренного треугольника $XYZ$,
а~значит, и~высота.
$ZI \perp BC$, $I A_1 \perp BC$ как угол между радиусом и~касательной.
Таким образом, $Z$, $I$, $A_1$ действительно лежат на~одной прямой.
\fi % \ifincludesolutions

\begin{problems}

\item
На~меньших дугах $AB$ и~$AC$ остроугольного треугольника $ABC$ отметили
точки $C_0$ и~$B_0$ соответственно, причем $B_0 C_0 \parallel BC$.
Докажите, что центры вписанных окружностей треугольников $A B C_0$ и~$A C B_0$
равноудалены от~середины дуги $BAC$ описанной окружности исходного
треугольника.

\end{problems}

\ifincludesolutions
\emph{Лемма (о~трилистнике).}
В~треугольнике $ABC$ с~центром вписанной окружности $I$ и~серединой~$A_0$
дуги~$BC$ описанной окружности треугольника $ABC$ выполнено равенство
$B A_0 = I A_0 = C A_0$.
\\
\emph{Используем ее~далее без доказательства.}
\par
Отметим середины $P$ и~$Q$ меньших дуг $B C_0$ и~$C B_0$ соответственно,
$I_C$ и~$I_B$~— центры вписанных окружностей треугольников
$A B C_0$ и~$A C B_0$.
Воспользуемся леммой (для треугольников $A B C_0$ и~$A C B_0$):
$P B = P I_C = P C_0$, $Q C = Q I_B = Q B_0$.
Из~параллельности $BC$ и~$B_0 C_0$ вытекает равенство дуг $B C_0$ и~$C B_0$,
откуда следует, что все шесть отрезков из~предыдущего предложения равны между
собой.
В~частности, $P I_C = Q I_B$.
Обозначим середину дуги $BAC$ описанной окружности исходного треугольника $X$,
и~рассмотрим треугольники $X P I_C$ и~$X Q I_B$.
Они равны: $P I_C = Q I_B$ доказано ранее; $XP = XQ$, что легко вывести
из~равенства соответствующих дуг описанной окружности;
$\angle X P I_C = \angle X Q I_B$ как углы, опирающиеся на~одну и~ту~же
дугу~$AX$.
Следовательно, $X I_C = X I_B$, что и~требовалось доказать.
\fi % \ifincludesolutions

\begin{problems}

\item
В~круге радиуса~$16$ расположено $650$ точек.
Докажите, что найдется кольцо со~внутренним радиусом~$2$ и~внешним
радиусом~$3$, в~котором лежит не~менее $10$ из~данных точек.

\end{problems}

\ifincludesolutions
Заметим, что точка~$X$ принадлежит кольцу с~центром~$O$ тогда и~только тогда,
когда точка~$O$ принадлежит такому~же кольцу с~центром~$X$.
Поэтому достаточно доказать, что если построить кольца с~центрами в~данных
точках, то~одну из~точек плоскости покроет не~менее $10$~колец.
Рассматриваемые кольца лежат внутри круга радиуса $16 + 3 = 19$, площадь
которого равна $19^2 \mathrm{\pi} = 361 \mathrm{\pi}$.
В~предположении противного (если все точки плоскости покрыты не~более $9$~раз)
суммарная площадь колец не~превосходит
$9 \cdot 361 \mathrm{\pi} = 3249 \mathrm{\pi}$.
Но~суммарная площадь колец равна
$650 \cdot (3^2 - 2^2) \mathrm{\pi} = 3250 \mathrm{\pi}$, противоречие.
\fi % \ifincludesolutions

\begin{problems}

\item
В~неправильном тетраэдре $ABCD$ все грани равны, $O$~— центр его описанной
сферы, $H$~— точка пересечения высот $BCD$.
Докажите, что $AOH \perp BCD$.

\end{problems}

\ifincludesolutions
Ясно, что из~равногранности тетраэдра следует, что $AB = CD$, $BC = AD$,
$AC = BD$.
Опустим перпендикуляры из~точек $A$, $O$, $H$ на~ребро~$BC$, основания
обозначим $A_D$, $O_D$, $H_D$ соответственно.
Раз $O$~— центр описанной сферы, то~$OB = OC$ и~$O_D$~— середина~$BC$.
Треугольники $ABC$ и~$DCB$ равны, а~значит отрезки $A_D B$ и~$H_D C$ тоже равны
как соответственные элементы равных фигур.
Получили, что $O_D$~— середина $A_D H_D$.
Аналогичное утверждения можно сформулировать для проекций тех~же точек
на~ребра $CD$, $DB$.
\par
Спроецируем точки $A$ и~$O$ на~плоскость $BCD$ ортогонально, получим
$A'$ и~$O'$.
$O'$~— центр описанной окружности треугольника $BCD$ как проекция центра
сферы.
Что касается точки $A'$, то~по~теореме о~трех перпендикулярах $A' A_D \perp BC$
(так как $AA' \perp BCD$ и~$A A_D \perp BC$).
Аналогично, $A' A_C \perp BD$ и~$A' A_B \perp CD$.
Рассмотрим точку $X'$, симметричную $H$ относительно $O'$.
Докажем, что $X = A'$.
Действительно, по~теореме Фалеса для проекций точек $O'$, $H$, $X$ на~сторону
$BC$ треугольника $BCD$ должно быть выполнено $H_DO_D = X_DO_D$, т.~е.
$X_D = A_D$, проекции точек $X$ и~$A'$ на~сторону $BC$ совпадают.
То~же самое верно и~для остальных сторон треугольника, а~по~трем своим
проекциям на~стороны треугольника любая точка плоскости восстанавливается
однозначно.
Следовательно, $X = A'$.
Точки $A$, $O$, $H$ спроецировались в~одну прямую, а~значит они лежали в~плоскости, перпендикулярной $BCD$.
\fi % \ifincludesolutions

\begin{problems}

\item
В~треугольнике $ABC$ провели биссектрисы $B B_1$ и~$C C_1$, которые
пересекаются в~точке $I$.
Прямая $B_1 C_1$ пересекает описанную окружность треугольника $ABC$
в~точках $M$ и~$N$.
Докажите, что радиус описанной окружности треугольника $IMN$ вдвое больше
радиуса описанной окружности треугольника $ABC$.

\end{problems}

\ifincludesolutions
\begin{figure}[ht]\begin{center}
    \jeolmfigure[width=0.5\textwidth]{n6-solution}
\end{center}\end{figure}
\emph{Лемма (о~трилистнике, усиленная).}
Точка $I$~— центр вписанной в~треугольник $ABC$ окружности, $I_A$~— центр
вневписанной окружности (касающейся стороны~$BC$ и~продолжений сторон
$AB$ и~$AC$).
$A_0$~— середина дуги $BC$ описанной окружности треугольника.
Тогда $B A_0 = I A_0 = C A_0 = I_A A_0$.
\\
\emph{Используем ее~далее без доказательства.}
\par
Продлим биссектрисы $BB_1$ и~$CC_1$ до~пересечения с~описанной окружностью
треугольника $ABC$, точки пересечения обозначим $B_0$ и~$C_0$ соответственно.
Обозначим также центры вневписанных окружностей, соответствующих вершинам $B$
и~$C$, буквами $I_B$ и~$I_C$ соответственно.
Тогда из~леммы о~трилистнике и~принадлежности точек $I_B$ и~$B_0$ биссектрисе
угла $B$ вытекает, что $II_B = 2IB_0$, и~аналогично $II_C = 2IC_0$.
Кроме того, треугольники $AB_0C_0$ и~$IB_0C_0$ равны по~трем сторонам (еще одно
следствие трилистника), а~значит и~равны их~радиусы описанных окружностей.
Треугольники $IB_0C_0$ и~$II_BI_C$ подобны с~коэффициентом $2$, радиусы их~описанных окружностей относятся так~же.
Остается доказать, что радиусы описанных окружностей треугольников $IMN$ и~$II_BI_C$ равны.
Для этого покажем, что точки $I$, $M$, $N$, $I_B$, $I_C$ лежат на~одной
окружности.
\par
Обозначим точки пересечения описанных окружностей треугольников $ABC$
и~$I I_B I_C$ за~$M'$ и~$N'$.
Тогда достаточно показать, что отрезки $M N$ и~$M'N'$ совпадают;
а~для этого, в~свою очередь, достаточно показать, что точки $B_1$ и~$C_1$ лежат
на~прямой~$M'N'$, то~есть на~радикальной оси описанных окружностей
треугольников $ABC$ и~$I I_B I_C$.
\par
Из~леммы о~трилистнике следует, что точки $I$, $I_B$, $A$ и~$C$ лежат на~одной
окружности~— обозначим ее~$\omega_B$.
Заметим, что $B_1$ лежит на~радикальной оси окружностей $ABC$ и~$\omega_B$,
а~также на~радикальной оси окружностей $I I_B I_C$ и~$\omega_B$.
Следовательно, $B_1$~— радикальный центр упомянутых трех окружностей,
и~лежит на~радикальной оси $ABC$ и~$I I_B I_C$.
Аналогично для $C_1$.
\fi % \ifincludesolutions

\endgroup % \providecommand\ifincludesolutions

