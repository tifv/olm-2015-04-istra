% $date: 2015-04-03
% $timetable:
%   g11r2:
%     2015-04-03:
%       3:

\section*{Радикальные оси}

% $authors:
% - Фёдор Ивлев

\begin{problems}

\item
\subproblem
Докажите, что середины четырех общих касательных к~двум непересекающимся кругам
коллинеарны.
\\
\subproblem
Через две из~точек касания общих внешних касательных с~двумя окружностями
проведена прямая.
Докажите, что окружности высекают на~этой прямой равные хорды.

\item
\subproblem
Точки $A_1$ и~$A_2$ лежат на~стороне~$BC$,
$B_1$ и~$B_2$~--- на~стороне~$AC$,
$C_1$ и~$C_2$~--- на~стороне~$AB$.
Известно, что точки $A_1$, $A_2$, $B_1$, $B_2$ лежат на~одной окружности;
$B_1$, $B_2$, $C_1$, $C_2$ лежат на~одной окружности;
$C_1$, $C_2$, $A_1$, $A_2$ лежат на~одной окружности.
Докажите, что все 6 точек лежат на~одной окружности.
\\
\subproblem
Пользуясь предыдущим пунктом, докажите существование \emph{окружности Эйлера}:
докажите, что основания высот и~медиан любого треугольника лежат на~одной
окружности.

\item
Высоты $AA'$ и~$CC'$ треугольника $ABC$ пересекаются в~точке~$H$.
На~описанной окружности треугольника $ABC$ выбрана точка~$X$ со~свойством, что
$\angle HXB = 90^\circ$.
Докажите, что прямые $BX$, $AC$ и~$A'C'$ пересекаются в~одной точке.

\item
Прямая~$OA$ касается окружности в~точке~$A$, а~хорда~$BC$ параллельна~$OA$.
Прямые~$OB$ и~$OC$ вторично пересекают окружность в~точках $K$ и~$L$.
Докажите, что прямая~$KL$ делит отрезок~$OA$ пополам.

\item
На~плоскости даны окружность~$\omega$, точка~$A$, лежащая внутри $\omega$,
и~точка~$B$ ($B \neq A$).
Рассматриваются всевозможные треугольники $BXY$, такие что точки $X$ и~$Y$
лежат на~$\omega$ и~хорда~$XY$ проходит через точку~$A$.
Докажите, что центры окружностей,описанных около треугольников $BXY$, лежат
на~одной прямой.
% (П. Кожевников) ВМО окружной этап 99.4.10.2

\item
Пусть $I$~--- центр вписанной окружности треугольника $ABC$.
Перпендикуляр, восстановленный в~точке~$I$ к~отрезку~$AI$, пересекает
прямую~$BC$ в~точке~$P$.
$Q$~--- основание перпендикуляра, опущенного из~$I$ на~$AP$.
Докажите, что $Q$ лежит на~описанной окружности треугольника $ABC$.

\item
Точка~$M$~--- середина хорды~$AB$.
Хорда~$CD$ пересекает~$AB$ в~точке~$M$.
На~отрезке~$CD$ как на~диаметре построена полуокружность.
Точка~$E$ лежит на~этой полуокружности, и~$ME$~--- перпендикуляр к~$CD$.
Найдите угол $AEB$.

\item
Постройте окружность, проходящую через две заданные точки и~касающуюся данной
прямой.

\end{problems}

