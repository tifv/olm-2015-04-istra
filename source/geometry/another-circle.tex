% $date: 2015-04-03
% $timetable:
%   g9r2:
%     2015-04-03:
%       3:
%   g9r1:
%     2015-04-04:
%       2:
%   g10r1:
%     2015-04-07:
%       1:

\section*{Дополнительная окружность}

% $authors:
% - Фёдор Бахарев

% $build$matter[print]: [[.], [.]]
% $build$style[print]:
% - .[tiled4,-print]

\begin{problems}

\item
На~плоскости даны прямая~$\ell$ и~две точки $A$ и~$B$ по~одну сторону от~нее.
На~прямой~$\ell$ выбраны точка~$M$, сумма расстояний от~которой
до~точек $A$ и~$B$ наименьшая, и~точка~$N$, для которой расстояния
от~$A$ и~$B$ равны: $AN = BN$.
Докажите, что точки $A$, $B$, $M$, $N$ лежат на~одной окружности.

\item
Продолжение биссектрисы~$AD$ остроугольного треугольника $ABC$ пересекает
описанную окружность в~точке~$E$.
Из~точки~$D$ на~стороны $AB$ и~$AC$ опущены перпендикуляры $DP$ и~$DQ$.
Докажите, что $S_{ABC}= S_{APEQ}$.

\item
В~треугольнике $ABC$ на~стороне~$BC$ выбрана точка~$M$ так, что точка
пересечения медиан треугольника $ABM$ лежит на~описанной окружности
треугольника $ACM$, а~точка пересечения медиан треугольника $ACM$ лежит
на~описанной окружности треугольника $ABM$.
Докажите, что медианы треугольников $ABM$ и~$ACM$ из~вершины~$M$ равны.

\item
В~треугольнике $ABC$ окружность, проходящая через вершины $A$ и~$B$, касается
прямой~$BC$, а~окружность, проходящая через вершины $B$ и~$C$, касается
прямой~$AB$ и~пересекает первую окружность в~точке~$K$, отличной от~$B$.
Пусть~$O$~--- центр окружности, описанной около треугольника $ABC$.
Докажите, что угол $BKO$~--- прямой.

\item
Точки $A_2$, $B_2$ и~$C_2$ --– середины высот $A A_1$, $B B_1$ и~$C C_1$
остроугольного треугольника $ABC$.
Найдите сумму углов
$\angle B_2 A_1 C_2$, $\angle C_2 B_1 A_2$ и~$\angle A_2 C_1 B_2$.

\item
На~сторонах $AB$ и~$BC$ параллелограмма $ABCD$ выбраны точки $A_1$ и~$C_1$
соответственно.
Отрезки $A C_1$ и~$C A_1$ пересекаются в~точке~$P$.
Описанные окружности треугольников $A A_1 P$ и~$C C_1 P$ вторично пересекаются
в~точке~$Q$, лежащей внутри треугольника $ACD$.
Докажите, что $\angle PDA = \angle QBA$.

\item
На~сторонах $AP$ и~$PD$ остроугольного треугольника $APD$ выбраны
соответственно точки $B$ и~$C$.
Диагонали четырехугольника $ABCD$ пересекаются в~точке~$Q$.
Точки $H_1$ и~$H_2$ являются ортоцентрами треугольников $APD$ и~$BPC$
соответственно.
Докажите, что если прямая~$H_1 H_2$ проходит через точку~$X$ пересечения
описанных окружностей треугольников $ABQ$ и~$CDQ$, то~она проходит и~через
точку~$Y$ пересечения описанных окружностей треугольников $BQC$ и~$AQD$.
% 109794

\end{problems}

