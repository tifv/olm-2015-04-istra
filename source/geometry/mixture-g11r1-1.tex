% $date: 2015-04-01
% $timetable:
%   g11r1:
%     2015-04-01:
%       3:

\section*{Планиметрический разнобой}

% $authors:
% - Андрей Кушнир

\begin{enumerate}

\item
К~двум непересекающимся окружностям $\omega_1$ и~$\omega_2$ проведена общая
касательная, касающаяся их~в~точках $A$ и~$B$ соответственно.
Окружность, построенная на~$AB$ как на~диаметре, вторично пересекает
$\omega_1$ и~$\omega_2$ в~точках $D$ и~$C$ соответственно.
Докажите, что $AC$ и~$BD$ пересекаются на~линии, соединяющей центры
$\omega_1$ и~$\omega_2$.
% Дружба с~гомотетией

\item
Дана равнобедренная трапеция $ABCD$, $AD \parallel BC$.
$I$~— центр вписанной окружности треугольника $ACD$.
Точка~$X$ на~биссектрисе угла $ABD$ такова, что $IX \perp BC$.
Описанная окружность треугольника $XDB$ пересекает прямую~$AD$ в~точке~$Y$.
Докажите, что треугольник $BXY$~— равнобедренный.

\item
$B_1$ и~$A_1$~— точки касания соответственных вневписанных окружностей
треугольника $ABC$ со~сторонами $AC$ и~$BC$ соответственно.
Луч~$A A_1$ впервые пересекает вписанную в~треугольник окружность в~точке~$P$
и~прямую~$B B_1$ в~точке~$Q$.
Докажите, что $AP = A_1 Q$.
% Отношения отрезков, гомотетия и~баррицентрические координаты

\item
В~прямоугольном треугольнике $ABC$ с~прямым углом~$C$ отмечена точка
пересечения медиан $G$.
На~лучах $AG$, $BG$ отметили такие точки $P$ и~$Q$ соответственно, что
$\angle APC = \angle CAB$ и~$\angle BQC = \angle CBA$.
Докажите, что вторая точка пересечения описанных окружностей треугольников
$AGQ$ и~$BGP$ лежит на~гипотенузе~$AB$.
% Мастерам степеней точек

\item
Внутри вписанного четырехугольника $ABCD$ нашлась такая точка $X$, что
$\angle XAB = \angle XBC = \angle XCD = \angle XDA$.
Продолжения пар противоположных сторон пересекаются в~точках $P$ и~$Q$.
Докажите, что $\angle PXQ$ равен углу между диагоналями четырехугольника.
% Канада примерно 2014

\item
В~равнобедренном треугольнике $ABC$ на~основании~$BC$ отмечена точка~$D$,
$BD = 2 CD$.
На~отрезке~$AD$ отметили $E$, $\angle BAC = \angle BED$.
Докажите, что $\angle BED = 2 \angle CED$.

\item
Точки $M$ и~$N$~— середины большой и~малой дуг $BC$ описанной окружности
остроугольного неравнобедренного треугольника $ABC$ соответственно,
$BH$~— высота.
Точка~$K$ на~прямой~$AM$ такова, что $\angle NHK = 90^{\circ}$.
Докажите, что $BK = BM$.

\item
В~треугольнике $ABC$ на~стороне~$BC$ отмечены точки:
$X_0$, $X_3$ и~$X_6$ на~$BC$,
$X_1$, $X_4$ на~$AC$, $X_2$, $X_5$ на~$AB$.
Известно, что прямые $X_0 X_1$, $X_2 X_3$, $X_4 X_5$ параллельны сторонам
треугольника, на~которых эти точки не~лежат,
а~$X_1 X_2$, $X_3 X_4$, $X_5 X_6$~— антипараллельны.
$O$~— центр описанной окружности треугольника, $L$~— точка Лемуана
(пересечения симедиан) треугольника.
\\
\subproblem
Докажите, что $X_0 = X_6$.
\\
\subproblem
Докажите, что $X_0$, $X_1$, $X_2$, $X_3$, $X_4$, $X_5$ лежат на~одной
окружности.
\\
\subproblem
Докажите, что центр этой окружности лежит на~прямой~$OL$.
% Линейность, все дела.

\item
В~шестиугольнике $ABCDEF$ (несамопересекающемся, но~не~обязательно выпуклом)
внутренние уголки удовлетворяют равенствам
$\angle A = 3 \angle D$, $\angle C = 3 \angle F$, $\angle E = 3 \angle B$.
Более того, $AB = DE$, $BC = EF$, $CD = FA$.
Докажите, что прямые $AD$, $BE$, $CF$ пересекаются в~одной точке.
% USAMO 2011

\end{enumerate}

