% $date: 2015-04-06
% $timetable:
%   g9r2:
%     2015-04-06:
%       3:

\section*{Степени точки и радикальные оси}

% $authors:
% - Фёдор Бахарев

\begin{problems}

\item
Через вершину~$B$ остроугольного треугольника $ABC$ проведено две окружности,
которые касаются стороны~$AC$ в~точках $A$ и~$C$ и~пересекаются вторично
в~точке~$M$.
\\
\subproblem
Докажите, что $M$ лежит на~медиане треугольника, выходящей из~вершины~$B$.
\\
\subproblem
Докажите, что $A$, $C$, $M$ и~ортоцентр треугольника $H$ лежат на~одной
окружности.

\item
Высоты $A A_1$, $C C_1$ остроугольного треугольника $ABC$ пересекаются
в~точке~$H$.
Прямые $A_1 C_1$ и~$AC$ пересекаются в~точке~$K$.
Докажите, что прямая~$HK$ перпендикулярна медиане треугольника $ABC$
из~вершины~$B$.

\item
В~четырехугольнике $ABCD$ углы $A$ и~$C$~--- прямые.
На~сторонах $AB$ и~$CD$ как на~диаметрах построены окружности, пересекающиеся
в~точках $X$ и~$Y$.
Докажите, что прямая~$XY$ проходит через середину диагонали~$AC$.

\item
Дан такой выпуклый четырехугольник $ABCD$, что $AB = BC$ и~$AD = DC$.
Точки $K$, $L$ и~$M$~--- середины отрезков $AB$, $CD$ и~$AC$ соответственно.
Перпендикуляр, проведенный из~точки~$A$ к~прямой~$BC$, пересекается
с~перпендикуляром, проведенным из~точки~$C$ к~прямой~$AD$, в~точке~$H$.
Докажите, что прямые $KL$ и~$HM$ перпендикулярны.

\item
Пусть точки $B_1$ и~$C_1$~--- середины касательных $AB$ и~$AC$, проведенных
из~точки~$A$ к~окружности~$S$.
На~прямой~$B_1 C_1$ выбрана точка~$F$.
Докажите, что длина касательной из~точки~$F$ к~окружности~$S$ равна длине
отрезка~$AF$.

\item
Дана неравнобедренная трапеция $ABCD$ ($AB \parallel CD$).
Окружность, проходящая через точки $A$ и~$B$, пересекает боковые стороны
трапеции в~точках $P$ и~$Q$, а~диагонали~--- в~точках $M$ и~$N$.
Докажите, что прямые $PQ$, $MN$ и~$CD$ пересекаются в~одной точке.

\item
В~остроугольном треугольнике $ABC$ проведены высоты $AP$ и~$BQ$, а~также
медиана~$CM$.
Точка~$R$~--- середина~$CM$.
Прямая~$PQ$ пересекает прямую~$AB$ в~точке~$T$.
Докажите, что $OR \perp TC$, где $O$~--- центр описанной окружности
треугольника $ABC$.

\end{problems}

