% $date: 2015-04-08
% $timetable:
%   g9r2:
%     2015-04-08:
%       3:

\section*{Разнобой, повторение (геометрия)}

% $authors:
% - Фёдор Бахарев

\begin{problems}

\item
Четырехугольник $ABCD$ вписан в~окружность.
Пусть $H_A$, $H_B$, $H_C$, $H_D$~--- ортоцентры треугольников
$BCD$, $CDA$, $DAB$ и~$ABC$ соответственно.
Докажите, что
\\
\sp $H_A H_B = AB$;
\qquad
\sp $H_A H_B H_C H_D$~--- четырехугольник, равный $ABCD$.

\item
Четырехугольник $ABCD$ вписан в~окружность.
Пусть $I_A$, $I_B$, $I_C$, $I_D$~--- центры вписанных окружностей треугольников
$BCD$, $CDA$, $DAB$ и~$ABC$ соответственно.
Докажите, что $I_A I_B I_C I_D$~--- прямоугольник.

\item
Через вершину~$A$ остроугольного треугольника $ABC$ проведены касательная~$AK$
к~его описанной окружности, а~также биссектрисы $AN$ и~$AM$ внутреннего
и~внешнего углов при вершине~$A$ (точки $M$, $K$ и~$N$ лежат на~прямой $BC$).
Докажите, что $MK = KN$.

\item
Из~точки~$P$ проведены касательные $PA$ и~$PB$ к~окружности~$\omega$.
Точка~$M$~--- середина хорды~$AB$.
Прямая, проходящая через~$M$, пересекает окружность в~точках $C$ и~$D$.
Докажите, что $PM$~--- биссектриса угла $CPD$.

\item
Четырехугольник $ABCD$ без параллельных сторон вписан в~окружность.
Для каждой пары касающихся окружностей, одна из~которых имеет хорду~$AB$,
а~другая~--- хорду~$CD$, отметим их~точку касания~$X$.
Докажите, что все такие точки~$X$ лежат на~одной окружности.

\item
Из~некоторой точки~$D$ в~плоскости треугольника $ABC$ провели прямые,
перпендикулярные к~отрезкам $DA$, $DB$, $DC$, которые пересекают прямые
$BC$, $AC$, $AB$ в~точках $A_1$, $B_1$, $C_1$ соответственно.
Докажите, что середины отрезков $A A_1$, $B B_1$, $C C_1$ лежат на~одной
прямой.

\item
Серединный перпендикуляр к~диагонали~$AC$ вписанного четырехугольника
$ABCD$ пересекает прямые $AD$ и~$CD$ в~точках $P$ и~$Q$.
Докажите, что биссектрисы углов $ABC$ и~$PBQ$ совпадают.

\end{problems}

