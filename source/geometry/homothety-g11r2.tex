% $date: 2015-03-31
% $timetable:
%   g11r2:
%     2015-03-31:
%       3:

\section*{Гомотетия}

% $authors:
% - Андрей Кушнир

\begin{problems}

\item
На~плоскости даны два неравных треугольника с~параллельными соответственными
сторонами.
Докажите, что существует гомотетия, переводящая один треугольник в~другой.

\item
Дан треугольник $ABC$.
Вписанная окружность касается стороны~$BC$ в~точке~$K$.
Вневписанная окружность касается отрезка~$BC$ в~точке~$L$.
Точка~$K'$ лежит на~вписанной окружности и~диаметрально противоположна
точке~$K$.
Точка~$L'$ лежит на~вневписанной окружности и~диаметрально противоположна
точке~$L$.
Докажите, что прямые $K'L$ и~$KL'$ пересекаются в~точке~$A$.

\item
В~треугольнике $ABC$ из~середины стороны~$BC$ провели вторую касательную
к~вписанной окружности треугольника, точку касания обозначили $K$.
Прямая~$AK$ пересекает $BC$ в~точке~$L$.
Докажите, что $L$~— точка касания вневписанной окружности со~стороной~$BC$.

\item
Точку~$L$ касания вневписанной окружности треугольника $ABC$ со~стороной~$BC$
соединили линией с~центром~$I$ вписанной окружности.
Докажите, что линия проходит через середину высоты, опущенной из~$A$ на~$BC$.

\item
Внутри треугольника $ABC$ нарисованы четыре круга одинакового радиуса:
$\omega_A$, $\omega_B$, $\omega_C$ и~$s$, причем каждый из~кругов $\omega_i$
касается двух сторон треугольника и~$s$.
Докажите, что центр круга~$s$ принадлежит прямой, проходящей через центры
вписанной и~описанной окружностей треугольника $ABC$.

\item
В~остроугольном треугольнике $ABC$ провели высоты $A A_1$, $B B_1$, $C C_1$.
$H$~— точка пересечения высот.
Точки $A_2$, $B_2$, $C_2$ симметричны точке $H$ относительно прямых
$B_1 C_1$, $C_1 A_1$, $A_1 B_1$ соответственно.
Докажите, что $A A_2$, $B B_2$, $C C_2$ пересекаются в~одной точке.

\item
Середины высот треугольника соединили прямыми с~соответствующими центрами
вневписанных окружностей.
Доказать, что эти три прямые пересекаются в~одной точке.

\item
Касательные к~описанной окружности остроугольного треугольника $ABC$ к~точками
$B$ и~$C$ пересекаются в~точке~$P$.
$M$~— середина $BC$;
$A_1$, $B_1$, $C_1$~— основания понятно каких высот, $H$~— ортоцентр.
Докажите, что $MH$, $P A_1$, $B_1 C_1$ пересекаются в~одной точке.

\item
В~четырехугольник $ABCD$ вписана окружность с~центром~$I$.
Лучи $AB$, $DC$ пересекаются в~точке~$X$.
$P$~— точка касания вписанной окружности треугольника $XBC$ с~отрезком~$BC$.
$Q$~— точка касания вневписанной окружности треугольника $XAD$
с~отрезком~$AD$.
Оказалось, что прямая~$PQ$ проходит через $X$.
$M$ и~$N$~— середины $BC$ и~$AD$.
Докажите, что $M$, $N$, $I$ лежат на~одной прямой.

\end{problems}

