% $date: 2015-04-05
% $timetable:
%   g10r2:
%     2015-04-05:
%       2:

% $caption: Воробьи (геометрия)

\section*{Важная добавка. Воробьи (геометрия)}

% $authors:
% - Алексей Доледенок

\begin{problems}

\item
\subproblem
На~сторонах $AB$ и~$BC$ неравнобедренного треугольника $ABC$ выбраны точки
$C_0$ и~$A_0$ соответственно, точка~$B_1$~--- середина дуги~$ABC$ описанной
окружности треугольника $ABC$.
Докажите, что $A C_0 = C A_0$ тогда и~только тогда, когда
точки $A_0$, $C_0$, $B_1$, $B$ лежат на~одной окружности.
\\
\subproblem
На~сторонах $AB$ и~$BC$ неравнобедренного треугольника $ABC$ выбраны
точки $C_0$ и~$A_0$ соответственно, точка~$I$~--- центр вписанной окружности
треугольника $ABC$.
Докажите, что окружность, описанная вокруг $A_0 B C_0$, проходит через $I$
тогда и~только тогда, когда $A C_0 + C A_0 = AC$.

\item
$A_0$, $B_0$, $C_0$~--- точки касания вневписанных окружностей со~сторонами
$BC$, $AC$, $AB$.
$A_1$, $B_1$, $C_1$~--- точки пересечения описанных окружностей треугольников
$A B_0 C_0$, $A_0 B C_0$, $A_0 B_0 C$ с~описанной окружностью
треугольника $ABC$.
Докажите, что треугольник $A_1 B_1 C_1$ подобен треугольнику с~вершинами
в~точках касания вписанной окружности со~сторонами треугольника $ABC$.

\item
Точки $A_1$, $B_1$, $C_1$ выбраны на~сторонах $BC$, $AC$, $AB$ треугольника
$ABC$ таким образом, что
\(
    A B_1 - A C_1 = C A_1 - C B_1 = B C_1 - B A_1
\).
Пусть $I$~--- центр вписанной окружности треугольника $ABC$,
$I_A$, $I_B$, $I_C$ и~$O_A$, $O_B$, $O_C$~--- центры вписанных и~описанных
окружностей треугольников $A B_1 C_1$, $A_1 B C_1$, $A_1 B_1 C$.
\\
\subproblem
Докажите, что $I$~--- центр описанной окружности треугольника $I_A I_B I_C$.
\\
\subproblem
Докажите, что $I$~--- центр вписанной окружности треугольника $O_A O_B O_C$.

\item
На~сторонах $AB$ и~$BC$ треугольника $ABC$ выбраны точки $C_0$ и~$A_0$,
$M$ и~$M_0$~--- середины отрезков $AC$ и~$A_0 C_0$.
Докажите, что если $A_0 C = A C_0$, то~прямая~$M M_0$ параллельна биссектрисе
угла~$B$.

\item
Дан неравнобедренный треугольник $ABC$, $AB <BC$, $B_1$~--- середина дуги $ABC$
описанной окружности треугольника $ABC$, $M$~--- середина стороны~$AC$.
Докажите, что центры $I_A$, $I_C$ вписанных окружностей в~треугольники $ABM$
и~$CBM$, точки $B$ и~$B_1$ лежат на~одной окружности.

\end{problems}

