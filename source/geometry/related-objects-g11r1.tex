% $date: 2015-03-31
% $timetable:
%   g11r1:
%     2015-03-31:
%       2:

\section*{Геометрия с соотношениями}

% $authors:
% - Андрей Кушнир

% $matter[-required-packages-guard]:
% - preamble package: ulem
%   options: [normalem]
% - .[required-packages-guard]

\begin{problems}

\item
Точки $O$ и~$I$~--- центры описанной и~вписанной окружностей неравнобедренного
треугольника $ABC$ соответственно.
Две равные окружности касаются пар сторон $AB$ и~$BC$, $AC$ и~$BC$ и~друг друга
в~точке~$K$.
Оказалось, что $K$, $I$, $O$ лежат на~одной прямой.
Найдите $\angle BAC$.

\item
На~сторонах треугольника $ABC$ отметили точки $A_1$, $B_1$, $C_1$ таким
образом, что центры вписанных окружностей треугольников $ABC$ и~$A_1 B_1 C_1$
совпадают, а~радиусы относятся как $2 : 1$.
Докажите, что треугольник $ABC$~--- правильный.

\item
$A_1$, $B_1$, $C_1$~--- точки касания сторон треугольника $ABC$
с~соответствующими вневписанными окружностями,
$O$ и~$I$~--- центры описанной и~вписанной окружностей.
Докажите, что если $A B_1 A_1 C_1$~--- вписанный, то~$B_1 C_1$ проходит через
точку~$I$ и~точка~$A_1$ лежит на~прямой~$OI$.
% очень помогает задача про то, что восстановленные в точках
% $A_1$, $B_1$, $C_1$ перпендикуляры к сторонам треугольника пересекаются
% в одной точке.

\item
Внутри выпуклого четырехугольника $ABCD$ нашлись такие точки $P$ и~$Q$,
что четырехугольники $APQD$ и~$BPQC$ вписанные.
А~еще на~отрезке~$PQ$ нашлась нашлась такая точка~$E$, что
$\angle PBE = \angle QCE$ и~$\angle PAE = \angle QDE$.
Докажите, что $ABCD$~--- вписанный.

\item
Докажите, что отрезки, соединяющие середины противоположных сторон выпуклого
шестиугольника $ABCDEF$, пересекаются в~одной точке тогда и~только тогда, когда
площади треугольников $ACE$ и~$BDF$ равны.

\item
В~остроугольном треугольнике $ABC$ отметили центры $O$ и~$I$ описанной
и~вписанной окружностей соответственно и~ортоцентр~$H$.
Докажите, что $OI \parallel BC$ тогда и только тогда $\angle AIH = 90^{\circ}$.
% с мат. многоборья 2013

\item
Выпуклый четырехугольник $ABCD$ таков, что $AB = AC = BD$.
Диагонали $AC$ и~$BD$ пересекаются в~точке~$O$.
Описанные окружности треугольников $ABC$ и~$AOD$ вторично пересекаются
в~точке~$P$.
Прямые $AP$ и~$BC$ пересекаются в~точке~$Q$.
Докажите, что $OQ$~--- биссектриса угла $\angle COD$.

\item
Задача отозвана.
% не решил пока
\sout{%
В~треугольнике $ABC$ отметили центр~$O$ описанной окружности и~опустили
высоту~$AH$.
$B_1$, $C_1$~--- проекции точки~$H$ на~$AC$ и~$AB$.
Докажите, что если $AB = 2 B_1 O$, то~$AC = 2 C_1 O$.}

\item
В~треугольнике $ABC$ провели биссектрисы $A A_1$, $B B_1$, $C C_1$.
Оказалось, что прямые $A A_1$ и~$B_1 C_1$ пересекаются под углом $60^{\circ}$.
Докажите, что один из~углов треугольника $A_1 B_1 C_1$~--- прямой.

\item
В~треугольнике $ABC$ на~сторонах $AB$ и~$AC$ отмечены точки $E$ и~$F$
соответственно, $EF \parallel BC$.
$EF$ пересекает описанную окружность треугольника $ABC$ в~точках $U$ и~$V$.
Точка $M$~--- середина~$BC$.
Оказалось, что описанные окружности $\omega$ и~$\omega'$ треугольников $ABC$
и~$MUV$ соответственно равны.
$ME$ пересекает $\omega'$ вторично в~$T$, $TF$ пересекает $\omega'$ вторично
в~$S$.
Докажите, что $EF$ касается описанной окружности треугольника $MCS$.
% Иран 2002, раунд 3

\end{problems}

