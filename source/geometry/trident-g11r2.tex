% $date: 2015-04-04
% $timetable:
%   g11r2:
%     2015-04-04:
%       2:

\section*{На поляне трилистников}

% $authors:
% - Фёдор Ивлев

\begin{problems}

\item
В~треугольнике $ABC$ проведены биссектрисы $AD$, $BE$ и~$CF$, пересекающиеся
в~точке~$I$.
Серединный перпендикуляр к~отрезку~$AD$ пересекает прямые $BE$ и~$CF$ в~точках
$M$ и~$N$ соответственно.
Докажите, что точки $A$, $I$, $M$ и~$N$ лежат на~одной окружности.
% ВМО региональный этап 10.4.10.3

\item
$A_1$ и~$B_1$~--- середины дуг $BC$ и~$AC$ окружности~$\omega$.
С~центрами в~этих точках построили окружности, касающиеся ближайших сторон
треугольника.
Докажите, что одна из~общих внешних касательных к~этим окружностям проходит
через~$I$.

\item
Точки $K$, $L$, $M$ и~$N$~---  точки пересечения внешних биссектрис углов
$A$ и~$B$, $B$ и~$C$, $C$ и~$D$, $D$ и~$A$ выпуклого четырехугольника $ABCD$
соответственно.
Оказалось, что описанные окружности треугольников $ABK$ и~$CDM$ касаются.
Докажите, что описанные окружности треугольников $CBL$ и~$DAN$ тоже касаются.

\item
В~треугольнике $ABC$ ($AB < BC)$ точка~$I$~--- центр вписанной окружности,
$M$~--- середина стороны~$AC$, $N$~--- середина дуги~$ABC$ описанной
окружности.
Докажите, что $\angle IMA = \angle INB$.

\item
На~дугах $AB$ и~$BC$ окружности, описанной около треугольника $ABC$, выбраны
соответственно точки $K$ и~$L$ так, что прямые $KL$ и~$AC$ параллельны.
Докажите, что центры вписанных окружностей треугольников $ABK$ и~$CBL$
равноудалены от~середины дуги~$ABC$.
% ВМО 2006.5.10.6 (С. Берлов)

\item
Дан вписанный четырехугольник $ABCD$.
\\
\subproblem
Докажите, что центры вписанных окружностей треугольников
$BCD$, $CDA$, $DAB$ и~$ABC$ образуют прямоугольник.
\\
\subproblem
Докажите, что его стороны параллельны биссектрисам углов между диагоналями.

\item
%Биссектрисы углов $A$ и~$C$ треугольника $ABC$ пересекают его стороны в~точках
%$A_1$ и~$C_1$, а~описанную окружность этого треугольника~--- в~точках
%$A_0$ и~$C_0$ соответственно.
%Прямые $A_1 C_1$ и~$A_0 C_0$ пересекаются в~точке~$P$.
%Докажите, что отрезок, соединяющий $P$ с~центром вписанной окружности
%треугольника $ABC$, параллелен~$AC$.
% ВМО 2006.4.11.4 (Л. Емельянов)
Биссектрисы углов~$A$ и~$C$ треугольника~$ABC$ пересекают описанную окружность
этого треугольника в~точках~$A_0$ и~$C_0$ соответственно.
Прямая, проходящая через центр вписанной окружности треугольника~$ABC$
параллельно стороне~$AC$, пересекается с~прямой~$A_0C_0$ в~точке~$P$.
Докажите, что прямая~$PB$ касается описанной окружности треугольника~$ABC$.
% light version

\item
\subproblem\label{geometry/trident-g11r2:archimedes-lemma}%
\emph{Лемма Архимеда.}
В~окружности~$\omega$ проведена хорда~$AB$.
Окружность~$\gamma$ касается $\omega$ в~точке~$P$ и~$AB$ в~точке~$Q$;
$M$~--- середина дуги~$AB$.
Докажите, что $P$, $Q$ и~$M$ лежат на~одной прямой.
\\
\subproblem
На~окружности расположены точки $A$, $B$, $C$ и~$D$.
Обозначим через $K$ середину дуги~$AB$, не~содержащей точек $C$ и~$D$.
Прямые $CK$ и~$DK$ пересекают прямую $AB$ в~точках $M$ и~$N$.
Докажите, что точки $M$, $N$, $C$ и~$D$ лежат на~одной окружности.
\\
\subproblem
Обозначим основание внешней биссектрисы угла~$C$ через~$P$.
Докажите, что существует окружность, касающаяся~$AB$ в~точке~$P$, а~описанной
окружности треугольника $ABC$ в~точке~$C$.
\\
\subproblem
В~обозначениях пункта
\ref{geometry/trident-g11r2:archimedes-lemma}
докажите, что касательная из~точки $M$ к~$\gamma$
равна $MA$.
\end{problems}

