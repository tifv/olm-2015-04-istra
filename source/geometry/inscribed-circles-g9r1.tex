% $date: 2015-04-07
% $timetable:
%   g9r1:
%     2015-04-07:
%       2:

\section*{Инцентры и вписанные окружности}

% $authors:
% - Фёдор Бахарев

\begin{problems}

\item
Точка~$C$ движется по~окружности с~хордой~$AB$ с~угловой скоростью~$\omega$.
Докажите, что центр вписанной окружности~$I$ треугольника $ABC$ движется
по~множеству, состоящему из~объединения дуг двух окружностей.
Какова угловая скорость точки~$I$?
Где расположены центры дуг?
Опишите движение центров вневписанных окружностей.

\item
\sp
Точки $I_A$, $I_B$, $I_C$ и~$I_D$~--- центры вписанных окружностей
треугольников $BCD$, $CDA$, $DAB$ и~$ABC$ соответственно.
Докажите, что если четырехугольник $ABCD$ вписанный, то~$I_A I_B I_C I_D$~---
прямоугольник.
\\
\sp
Как располагаются относительно друг друга 16 центров вписанных и~вневписанных
окружностей треугольников $BCD$, $CDA$, $DAB$ и~$ABC$ в~случае, если
четырехугольник $ABCD$ вписанный?
\\
\sp
Для каких $ABCD$ четырехугольник $I_AI_BI_CI_D$ является квадратом?

\item
На~стороне $AC$ треугольника $ABC$ выбрана точка~$X$.
Докажите, что если вписанные окружности треугольников $ABX$ и~$BCX$ касаются
друг друга, то~точка~$X$ лежит на~окружности, вписанной в~треугольник $ABC$.

\item
Пусть $I$~--- центр вписанной окружности треугольника $ABC$,
$D$~--- точка касания ее~со~стороной~$AC$, $B_1$~--- середина стороны~$AC$.
Докажите, что прямая~$B_1 I$ делит отрезок~$BD$ пополам.

\item
Через точки пересечения продолжений сторон выпуклого четырехугольника $ABCD$
проведены две прямые, делящие его на~четыре четырехугольника.
Докажите, что если четырехугольники, примыкающие к~вершинам $B$ и~$D$,
описанные, то~четырехугольник $ABCD$ тоже описанный.

\item
На~дуге~$AC$ описанной окружности треугольника $ABC$ взята произвольная
точка~$P$.
Пусть $I_1$ и~$I_2$~--- центры вписанных окружностей треугольников
$ABP$ и~$CBP$.
Докажите, что описанная окружность треугольника $I_1 I_2 P$ проходит через
некоторую фиксированную точку, не~зависящую от~выбора~$P$.

\item
Окружность с~центром~$I$ касается сторон $AB$, $BC$, $AC$ неравнобедренного
треугольника $ABC$ в~точках $C_1$, $A_1$, $B_1$ соответственно.
Окружности $\omega_B$ и~$\omega_C$ вписаны в~четырехугольники
$B A_1 I C_1$ и~$C A_1 I B_1$ соответственно.
Докажите, что общая внутренняя касательная к~$\omega_B$ и~$\omega_C$,
отличная от~$I A_1$, проходит через точку~$A$.

\end{problems}

