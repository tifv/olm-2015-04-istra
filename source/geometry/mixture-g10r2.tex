% $date: 2015-04-08
% $timetable:
%   g10r2:
%     2015-04-08:
%       2:

\section*{Всеросный разнобой}

% $authors:
% - Алексей Доледенок

\begin{problems}

\item
В~неравнобедренном остроугольном треугольнике $ABC$ проведены высоты
$A A_1$ и~$C C_1$, $H$~--- точка пересечения высот, $O$~--- центр описанной
окружности, $B_0$~--- середина стороны~$AC$.
Прямая $BO$ пересекает сторону $AC$ в~точке $P$, а~прямые $BH$ и~$A_1C_1$
пересекаются в~точке $Q$.
Докажите, что прямые $HB_0$ и~$PQ$ параллельны.
\\\emph{(Всероссийская олимпиада 2008, 10.6)}

\item
Окружность~$\omega$, вписанная в~остроугольный неравнобедренный треугольник
$ABC$, касается стороны~$BC$ в~точке~$D$.
Пусть точка~$I$~--- центр окружности~$\omega$, $O$~--- центр окружности,
описанной около треугольника $ABC$.
Окружность, описанная около треугольника $AID$, пересекает вторично прямую~$AO$
в~точке~$E$.
Докажите, что длина отрезка~$AE$ равна радиусу окружности~$\omega$.
\\\emph{(Всероссийская олимпиада 2012, 10.2)}

\item
Дан остроугольный треугольник $ABC$.
На~продолжениях его высот $B B_1$ и~$C C_1$ за~точки $B_1$ и~$C_1$ выбраны
соответственно точки $P$ и~$Q$ такие, что угол $PAQ$~--- прямой.
Пусть $AF$~--- высота треугольника $APQ$.
Докажите, что угол $BFC$~--- прямой.
\\\emph{(Всероссийская олмпиада 2011, 10.6)}

\item
Внутри треугольника $ABC$ взята точка~$K$, лежащая на~биссектрисе угла $BAC$.
Прямая~$CK$ вторично пересекает окружность~$\omega$, описанную около
треугольника $ABC$, в~точке~$M$.
Окружность~$\Omega$ проходит через точку~$A$, касается прямой~$CM$ в~точке~$K$
и~пересекает вторично отрезок~$AB$ в~точке~$P$, а~окружность~$\omega$~---
в~точке $Q$.
Докажите, что точки $P$, $Q$ и~$M$ лежат на~одной прямой.
\\\emph{(Всероссийская олимпиада 2010, 10.6)}

\item
Точка~$M$~--- середина стороны~$AC$ треугольника $ABC$.
На~отрезках $AM$ и~$CM$ выбраны точки $P$ и~$Q$ соответственно таким образом,
что $PQ = AC / 2$.
Окружность, описанная около треугольника $ABQ$, пересекает сторону~$BC$
в~точке $X \neq B$, а~окружность, описанная около треугольника $BCP$,
пересекает сторону~$AB$ в~точке $Y \neq B$.
Докажите, что четырехугольник $BXMY$~--- вписанный.
\\\emph{(Всероссийская олимпиада 2014, 10.6)}

\item
Окружность с~центром~$I$, вписанная в~треугольник $ABC$, касается сторон
$BC$, $CA$, $AB$ в~точках $A_1$, $B_1$, $C_1$ соответственно.
Пусть $I_a$, $I_b$, $I_c$~--- центры вневписанных окружностей треугольника
$ABC$, касающихся соответственно сторон $BC$, $CA$, $AB$.
Отрезки $I_a B_1$ и~$I_b A_1$ пересекаются в~точке $C_2$.
Аналогично, отрезки $I_b C_1$ и~$I_c B_1$ пересекаются в~точке $A_2$, а~отрезки
$I_c A_1$ и~$I_a C_1$~--- в~точке $B_2$.
Докажите, что $I$ является центром окружности, описанной вокруг треугольника
$A_2 B_2 C_2$.
\\\emph{(Всероссийская олимпиада 2013, 10.7)}

\item
Треугольник $ABC$, $AB > BC$, вписан в~окружность~$\Omega$.
На~сторонах $AB$ и~$BC$ выбраны точки $M$ и~$N$ соответственно так, что
$AM = CN$.
Прямые $MN$ и~$AC$ пересекаются в~точке~$K$.
Пусть $P$~--- центр вписанной окружности треугольника $AMK$, а~$Q$~--- центр
вневписанной окружности треугольника $CNK$, касающейся стороны~$CN$.
Докажите, что середина дуги~$ABC$ окружности~$\Omega$ равноудалена
от~точек $P$ и~$Q$.
\\\emph{(Всероссийская олимпиада 2014, 10.4)}

\end{problems}

