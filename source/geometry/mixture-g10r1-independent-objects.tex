% $date: 2015-04-02
% $timetable:
%   g10r1:
%     2015-04-02:
%       3:

\section*{Независимость (геометрия)}

% $authors:
% - Фёдор Бахарев

\begin{problems}

\item
Даны две окружности, касающиеся внутренним образом в~точке~$N$.
Касательная к~внутренней окружности, проведенная в~точке $K$, пересекает
внешнюю окружность в~точках $A$ и~$B$.
Пусть $M$~--- середина дуги~$AB$, не~содержащей точку~$N$.
Докажите, что радиус описанной окружности треугольника $BMK$ не~зависит
от~выбора точки~$K$ на~внутренней окружности.

\item
На~биссектрисе угла с~вершиной~$O$ взята точка~$P$.
Прямая, проходящая через $P$, пересекает стороны угла в~точках $M$ и~$N$.
Докажите, что величина $1 / OM + 1 / ON$ не~зависит от~выбора прямой.

\item
Две окружности пересекаются в~точке~$A$.
Проходящая через $A$ прямая пересекает их~в~точках $B$ и~$C$.
Докажите, что угол между касательными в~точках $B$ и~$C$ не~зависит от~прямой.

\item
На~стороне~$AC$ треугольника $ABC$ отмечена точка~$D$.
Произвольный луч~$\ell$, выходящий из~вершины~$B$, пересекает отрезок~$AC$
в~точке~$K$, а~описанную окружность треугольника $ABC$~--- в~точке~$L$.
Докажите, что описанная окружность треугольника $DKL$ проходит через
фиксированную точку, отличную от~$D$ и~не~зависящую от~выбора луча~$\ell$.

\item
Дана окружность~$\omega$ и~точка~$P$ вне нее.
Проходящая через~$P$ прямая~$\ell$ пересекает~$\omega$ в~точках~$A$ и~$B$.
На~отрезке~$AB$ отмечена точка~$C$ такая, что~$PA\cdot PB=PC^2$.
Точки~$M$ и~$N$~--- середины двух дуг, на~которые хорда~$AB$ разбивает
окружность~$\omega$.
Докажите, что величина $\angle MCN$ не~зависит от~выбора прямой~$\ell$.

\item
Через центр вписанной окружности четырехугольника $ABCD$ проведена
прямая.
Она пересекает сторону~$AB$ в~точке~$X$ и~сторону~$CD$ в~точке~$Y$;
углы $\angle AXY$ и~$\angle DYX$ равны.
Докажите, что $AX / BX = CY / DY$.

\item
Окружности $S_{1}$ и~$S_{2}$ пересекаются в~точках $M$ и~$N$.
Через точку~$A$ окружности~$S_{1}$ проведены прямые $AM$ и~$AN$,
пересекающие~$S_{2}$ в~точках $B$ и~$C$, а~через точку~$D$
окружности~$S_{2}$~--- прямые $DM$ и~$DN$, пересекающие $S_{1}$
в~точках $E$ и~$F$, причем $A$, $E$, $F$ лежат по~одну сторону от~прямой~$MN$,
а~$D$, $B$, $C$~--- по~другую.
Докажите, что если $AB = DE$, то~точки $A$, $F$, $C$ и~$D$ лежат на~одной
окружности, положение центра которой не~зависит от~выбора точек $A$ и~$D$.

\item
В~ромб $ABCD$ вписана окружность~$\omega$.
Прямая~$\ell$ касается~$\omega$ и~пересекает стороны $AB$ и~$AD$
в~точках $K$ и~$L$, а~продолжения сторон $BC$ и~$CD$ в~точках $M$ и~$N$
соответственно.
Докажите, что произведение площадей треугольников $AKL$ и~$CMN$ не~зависит
от~положения прямой~$\ell$.

\item
На~дуге~$AC$ описанной окружности треугольника $ABC$ взята произвольная
точка~$P$.
Пусть $I_1$ и~$I_2$~--- центры вписанных окружностей
треугольников $ABP$ и~$CBP$.
Докажите, что описанная окружность треугольника $I_1 I_2 P$ проходит через
некоторую фиксированную точку, не~зависящую от~выбора~$P$.

\end{problems}

