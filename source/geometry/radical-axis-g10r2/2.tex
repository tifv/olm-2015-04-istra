% $date: 2015-04-02
% $timetable:
%   g10r2:
%     2015-04-02:
%       1:

\section*{Радикальная ось}

% $authors:
% - Андрей Меньщиков

\begin{problems}

\item
На~сторонах $BC$, $AC$, $AB$ остроугольного треугольника $ABC$ взяты
произвольные точки $A_1$, $B_1$, $C_1$.
Докажите, что три общие хорды пар окружностей с~диаметрами
$A A_1$, $B B_1$, $C C_1$ пересекаются в~ортоцентре треугольника $ABC$.

\item
В~треугольнике $ABC$ проведены высоты $A A_1$, $B B_1$ и~$C C_1$.
Прямые~$AB$ и~$A_1 B_1$, $BC$ и~$B_1 C_1$, $CA$ и~$C_1 A_1$ пересекаются
в~точках $C'$, $A'$ и~$B'$ соответственно.
Докажите, что точки $A'$, $B'$ и~$C'$ лежат на~одной прямой
(точнее, на~радикальной оси окружности девяти точек и~описанной окружности).

\item
Докажите, что радикальная ось вписанной и~вневписанной окружностей треугольника
проходит через середину его стороны и~перпендикулярна биссектрисе угла,
противоположного этой середине.

\item
Пусть $B_1$, $C_1$~--- точки касания вписанной окружности треугольника $ABC$
со~сторонами $AC$ и~$AB$.
На~продолжениях сторон $AB$, $AC$ за~точки $B$ и~$C$ отметили точки $X$, $Y$
соответственно, так, что $C_1 X = B_1 Y = BC$.
Докажите, что середины отрезков $C_1 X$, $B_1 Y$, $BC$ лежат на~одной прямой.

\item
Дан шестиугольник $ABCDEF$, в~котором $AB = BC$, $CD = DE$, $EF = FA$,
а~углы $A$ и~$C$~--- прямые.
Докажите, что прямые $FD$ и~$BE$ перпендикулярны.

\item
$A_1$, $B_1$, $C_1$~--- точки касания вписанной в~треугольник $ABC$ окружности
со~сторонами $BC$, $CA$, $AB$ соответственно.
Точка~$P$~--- произвольная.
Серединный перпендикуляр к~отрезку~$P A_1$ пересекает прямую~$BC$
в~точке~$A_2$.
Аналогично строятся точки $B_2$ и~$C_2$.
Докажите, что $A_2$, $B_2$, $C_2$ лежат на~одной прямой.

\item
\subproblem
Через точку~$P$, лежащую на~общей хорде~$AB$ двух пересекающихся окружностей,
проведены хорда~$A_1 B_1$ первой окружности и~хорда~$A_2 B_2$ второй
окружности.
Докажите, что четырехугольник $A_1 A_2 B_1 B_2$~--- вписанный.
\\
\subproblem\emph{Теорема о бабочке.}
Через середину~$P$ хорды~$AB$ окружности проведены секущие
$A_1 A_2$ и~$B_1 B_2$.
Хорды $A_1 B_1$ и~$A_2 B_2$ пересекают хорду~$AB$ в~точках $M$ и~$N$.
Докажите, что $PM = PN$.

\item
Пусть продолжения сторон $AB$ и~$CD$ четырехугольника $ABCD$ пересекаются
в~точке~$P$, а~продолжения сторон $BC$ и~$AD$ в~точке~$Q$.
\\
\subproblem
Докажите, что ортоцентры треугольников $BPC$, $APD$, $ABQ$ и~$CDQ$ лежат
на~одной прямой.
\\
\subproblem\emph{Прямая Гаусса.}
Докажите, что середины $AC$, $BD$ и~$PQ$ лежат на~одной прямой.

\item
Пусть вписанная окружность треугольника $ABC$ касается сторон $AB$, $AC$, $BC$
в~точках $C_1$, $B_1$, $A_1$.
Докажите, что средние линии треугольников $A_1 C B_1$ и~$A_1 B C_1$,
соответственно параллельные сторонам $A_1 B_1$ и~$A_1 C_1$, а~также серединный
перпендикуляр к~$BC$, пересекаются в~одной точке.

\item
На~стороне~$AC$ треугольника $ABC$ взята произвольная точка~$X$.
Пусть $M$ и~$N$~--- точки пересечения серединных перпендикуляров к~отрезкам
$AX$ и~$BX$ с~отрезками $AB$ и~$BC$ соответственно.
Докажите, что точка, симметричная $X$ относительно прямой~$MN$, лежит
на~описанной окружности треугольника $ABC$.

\item
Пусть $A A_1$ и~$B B_1$~--- высоты остроугольного неравнобедренного
треугольника $ABC$.
Отрезок~$A_1 B_1$ пересекает среднюю линию $ABC$, параллельную $AB$,
в~точке~$C'$.
Докажите, что отрезок $C C'$ перпендикулярен прямой, соединяющей ортоцентр
и~центр описанной окружности треугольника $ABC$.

%\item
%Внутри выпуклого многоугольника расположено несколько попарно непересекающихся
%кругов различных радиусов.
%Докажите, что многоугольник можно разрезать на~маленькие многоугольники так,
%чтобы все они были выпуклыми и~в~каждом из~них содержался ровно один из~данных
%кругов.

\end{problems}

