% $date: 2015-04-01
% $timetable:
%   g10r2:
%     2015-04-01:
%       3:

\section*{Степень точки}

% $authors:
% - Андрей Меньщиков

\begin{problems}

\item
На~плоскости даны точки $A$, $B$, $C$, $D$.
Докажите, что прямая~$CD$ перпендикулярна $AB$ тогда и~только тогда, когда
$AC^2 - BC^2 = AD^2 - BD^2$.

\item
Даны две точки $A$ и~$B$ и~число~$k$.
Найдите геометрическое место точек~$M$ таких, что $AM^2 - BM^2 = k$.

\end{problems}

\definition
\emph{Степенью точки} $M$ относительно окружности называется величина
$d^2 - R^2$, где $d$~--- расстояние от~$M$ до~центра окружности,
$R$~--- радиус окружности.

\begin{problems}

\item
\subproblem
Докажите, что степень точки, лежащей вне окружности, равна квадрату длины
отрезка касательной, проведенной из~точки к~окружности.
\\
\subproblem
Докажите, что степень точки равна произведению $\pm AM \cdot BM$, где
$A$ и~$B$~--- точки пересечения с~окружностью произвольной прямой~$l$,
проходящей через $M$ (знак <<$+$>> в~случае, если точка вне окружности;
знак <<$-$>>, если точка внутри окружности).

\item
Докажите, что геометрическим местом точек, имеющих одинаковую степень
относительно двух данных неконцентрических окружностей, является
\textbf{прямая,} перпендикулярная их~линии центров.

\end{problems}

Эта прямая называется \emph{радикальной осью} двух окружностей.

\claim{Упражнение}
Докажите, что радикальная ось двух пересекающихся окружностей проходит через точки их~пересечения.

\begin{problems}

\item
Две непересекающиеся окружности имеют четыре общие касательные.
Докажите, что середины четырех полученных отрезков касательных лежат на~одной
прямой.

\item
Центры трех окружностей образуют треугольник.
Проведены радикальные оси для каждой пары этих окружностей.
Докажите, что все три радикальные оси пересекаются в~одной \textbf{точке.}

\end{problems}

Эта точка называется \emph{радикальным центром} трех окружностей.

\begin{problems}

%\item
%Докажите, что прямая, проходящая через точки пересечения окружностей,
%построенных на~чевианах $AР$ и~$BQ$ как на~диаметрах, проходит через ортоцентр
%треугольника $ABC$.

\item
Пусть $O$~--- радикальный центр трех окружностей, каждая из~которых лежит вне
двух других.
Докажите, что точки касания шести касательных, проведенных из~точки~$O$, лежат
на~одной окружности.

\item
Даны две неконцентрические непересекающиеся окружности $S_1$ и~$S_2$.
Докажите, что множеством центров окружностей, пересекающих $S_1$ и~$S_2$ под
прямым углом, является их~радикальная ось.

\item
Дана окружность~$S$ и~точки $P$ и~$K$ вне ее.
Через точку проводится секущая $PAB$ ($A$ и~$B$~--- точки пересечения
с~окружностью).
Построим описанную окружность треугольника $KAB$.
Докажите, что все такие окружности имеют общую точку, отличную от~$K$.

\item
На~гипотенузе~$AB$ прямоугольного равнобедренного треугольника $ABC$ выбрана
произвольная точка~$M$.
Докажите, что общая хорда окружностей с~центром~$C$ и~радиусом~$CA$
и~с~центром~$M$ и~радиусом~$MC$ проходит через середину $AB$.

\item
В~угол вписаны две окружности.
Одна окружность касается одной стороны угла в~точке~$A$, вторая окружность
касается другой стороны угла в~точке~$B$.
Докажите, что прямая~$AB$ высекает на~окружностях равные хорды.

\item
С~центром в~точке~$O$ построены больш\'{а}я окружность и~маленькая окружность.
Из~точки~$A$ большой окружности проведены касательные $AB$, $AC$ к~маленькой
($B$, $C$~--- точки касания).
Окружность с~центром~$A$ и~радиусом~$AB$ пересекает большую окружность
в~точках $M$, $N$.
Докажите, что прямая~$MN$ содержит среднюю линию треугольника $ABC$.

\item
$AB$~--- диаметр окружности~$\omega$, $C$~--- точка на~ней~же.
Окружность с~центром в~точке~$C$ касается прямой~$AB$ в~точке~$D$ и~пересекает
$\omega$ в~точках $E$ и~$F$.
Докажите, что отрезок~$EF$ точкой пересечения делит отрезок~$CD$ пополам.

\item
На~боковых сторонах трапеции как на~диаметрах построены окружности.
Докажите, что отрезки касательных, проведенных к~этим окружностям из~точки
пересечения диагоналей, равны.

\end{problems}

