% $date: 2015-04-04
% $timetable:
%   g9r2:
%     2015-04-04:
%       1:

\section*{Геометрические места точек}

% $authors:
% - Фёдор Бахарев

\begin{problems}

\item
На~плоскости даны две точки $A$ и~$B$.
Кроме того, дано число $a > 0$.
Докажите, что геометрическое место точек~$X$ таких, что $AX^2 - BX^2 = a$,
является прямой, перпендикулярной~$AB$.

\item
На~плоскости дан треугольник $ABC$.
Рассмотрим геометрическое место точек плоскости $X$, для которых
$AX / BX = AC / BC$.
\\
\sp
Укажите четыре различных точки на~плоскости, принадлежащих этому ГМТ.
\\
\sp
Найдите это геометрическое место точек.

\item
Дана окружность $\omega$ и~точки $A$ и~$B$ на~ней.
Точка~$C$ движется по~окружности $\omega$.
Какую траекторию описывает
\\
\sp ортоцентр $H$ треугольника $ABC$?
\\
\sp точка пересечения медиан треугольника $ABC$?
\\
\sp центр вписанной окружности $I$?
\\
\sp центр вневписанной окружности $I_{c}$?
\\
\sp центр вневписанной окружности $I_{a}$?

\item
Даны две равные окружности, пересекающиеся в~точках $A$ и~$B$.
Через точку~$A$ проводится прямая~$\ell$, пересекающая вторично первую
окружность в~точке~$X$, а~вторую в~точке~$Y$.
Найдите геометрическое место середин отрезков~$XY$, когда прямая~$\ell$
меняется.

\item
Дан треугольник $ABC$.
Найдите геометрическое место середин отрезков с~концами на~контуре
треугольника $ABC$.

\item
Дан прямой угол с~вершиной~$O$.
Найдите геометрическое место середин отрезков единичной длины с~концами
на~сторонах угла.

\end{problems}

