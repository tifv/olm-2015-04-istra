% $date: 2015-04-03
% $timetable:
%   g11r1:
%     2015-04-03:
%       1:

\section*{Комбинаторная можнолиметрия}

% $authors:
% - Фёдор Ивлев

\begin{problems}

%\item
%Внутри круга расположены точки $A_1, A_2, \ldots, A_n$, а на его границе~—
%точки $B_1, B_2, \ldots, B_n$ так, что
%отрезки $A_1 B_1, A_2 B_2, \ldots, A_n B_n$ не пересекаются.
%Кузнечик может перепрыгнуть из точки~$A_i$ в точку~$A_j$, если
%отрезок~$A_i A_j$ не пересекается ни с одним
%из отрезков~$A_k B_k$, $k \neq i, j$.
%Всегда ли кузнечик может попасть из любой точки~$A_p$ в любую точку~$A_q$?
%Докажите, что за несколько прыжков кузнечик сможет попасть из любой точки~$A_p$
%в любую точку~$A_q$.
%% ВМО 94.4.11.8 (С. Мисник, Д. Фон-дер-Флаас)

\item
Существует ли невыпуклый многогранник, вписанный в сферу?
\\
(Многогранник вписан в сферу, если все его вершины лежат на сфере.)

\item
Существует ли такая выпуклая фигура, что она не покрывает полукруг радиуса~$1$,
но двумя ее экземплярами можно покрыть круг радиуса~$1$?

\item
\subproblem
Существуют ли два равных семиугольника, все вершины которых совпадают,
но никакие стороны не совпадают?
\\
\subproblem
А три таких семиугольника?
\\
(Напоминание: многоугольник на плоскости ограничен несамопересекающейся
замкнутой ломаной.)

\item
Пространство разбито на одинаковые кубики.
Верно ли, что для \emph{каждого} из этих кубиков обязательно найдется другой,
имеющий с ним общую грань?

\item
Можно ли намотать нерастяжимую ленту на бесконечный конус так, чтобы сделать
вокруг его оси бесконечно много оборотов?
Ленту нельзя наматывать на вершину конуса, а также разрезать и перекручивать.
При необходимости можно считать, что она бесконечна, а угол между осью
и образующей конуса достаточно мал.
% ММО

\item
Существуют ли такие 2012 треугольников, ни один из которых нельзя покрыть
2011-ю остальными?

%\item
%Докажите, что любой неравнобедренный треугольник можно разбить на три
%треугольника равного диаметра.
%% Колмогоровский турнир 2007г., высшая юниорская лига, 4ый тур, номер 1

\item\emph{(«Багаж в Московском метрополитене»)}
Будем называть «размером» прямоугольного параллелепипеда сумму трех его
измерений~— длины, ширины и высоты.
Может ли случиться, что в некотором прямоугольном параллелепипеде поместился
больший по размеру прямоугольный параллелепипед?
% тургор, осенний основной тур 1998.10-11.5

\item
Дано бесконечное множество
\\
\subproblem прямоугольников,
\quad
\subproblem квадратов,
\\
сумма площадей которых не ограничена.
Обязательно ли ими можно полностью покрыть плоскость?

\item
\subproblem
Существует ли многогранник и точка внутри него, из которой не видно ни одной
его вершины?
Всегда ли существует разбиение многогранника на тетраэдры, вершины которых есть
вершины исходного многогранника?
\\
\subproblem
Тот же вопрос для точки вне многогранника.

\end{problems}

