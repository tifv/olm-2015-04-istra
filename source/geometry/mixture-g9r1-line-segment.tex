% $date: 2015-04-06
% $timetable:
%   g9r1:
%     2015-04-06:
%       2:

\section*{Отрезки}

% $authors:
% - Фёдор Бахарев

\begin{problems}

\item
Дан остроугольный треугольник $ABC$.
Точки $B_1$, $C_1$~--- основания высот, опущенных из~вершин~$B$, $C$
соответственно.
Точка~$D$~--- основание перпендикуляра из~$B_1$ на~$AB$, $E$~--- точка
пересечения перпендикуляра, опущенного из~из~точки~$D$ на~сторону~$BC$,
с~отрезком~$B B_1$.
Докажите, что прямая~$E C_1$ параллельна~$AC$.

\item
Через центр вписанной окружности четырехугольника $ABCD$ проведена прямая.
Она пересекает сторону~$AB$ в~точке~$X$ и~сторону~$CD$ в~точке~$Y$;
углы $\angle AXY$ и~$\angle DYX$ равны.
Докажите, что $AX / BX = CY / DY$.

\item
$AE$ и~$CD$~--- высоты остроугольного треугольника $ABC$.
Биссектриса угла~$B$ пересекает отрезок~$DE$ в~точке~$F$.
На~отрезках $AE$ и~$CD$ взяли такие точки $P$ и~$Q$ соответственно,
что четырехугольники $ADFQ$ и~$CEFP$~--- вписанные.
Докажите, что $AP = CQ$.

\item
Даны непересекающиеся окружности $S_1$ и~$S_2$ и~их~общие внешние
касательные $l_1$ и~$l_2$.
На~$l_1$ между точками касания отметили точку~$A$, а~на~$l_2$~--- точки~$B$
и~$C$ так, что $AB$ и~$AC$~--- касательные к~$S_1$ и~$S_2$.
Пусть $O_1$ и~$O_2$~--- центры окружностей $S_1$ и~$S_2$, а~$K$~--- точка
касания вневписаной окружности треугольника $ABC$ со~стороной~$BC$.
Докажите, что середина отрезка~$O_1 O_2$ равноудалена от~точек $A$ и~$K$.

\item
Пусть~$O$~--- центр описанной окружности остроугольного неравнобедренного
треугольника~$ABC$, точка~$C_1$ симметрична~$C$ относительно~$O$,
$D$~--- середина стороны~$AB$, $K$~--- центр описанной окружности
треугольника $O D C_1$.
Докажите, что точка~$O$ делит пополам отрезок прямой~$OK$, лежащий внутри
угла $ACB$.

\item
Точка~$K$~--- середина чевианы~$AD$ треугольника $ABC$.
Точка~$X$ на~отрезке~$KC$ такая, что $\angle ABK = \angle XBC$.
Оказалось, что $KX \cdot BD = CX \cdot CD$.
Докажите, что $\angle BAX = \angle BCX$.

\item
На~сторонах треугольника взяли по~две точки и~соединили их~с~противоположными
вершинами.
Оказалось, что все 6 проведенных отрезков имеют равную длину.
Докажите, что все их~середины лежат на~одной окружности.

\end{problems}

