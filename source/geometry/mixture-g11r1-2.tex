% $date: 2015-04-07
% $timetable:
%   g11r1:
%     2015-04-07:
%       1:

\section*{Разнобой по планиметрии}

% $authors:
% - Фёдор Ивлев

\begin{problems}

\item
Точка $H$~— ортоцентр остроугольного треугольника $ABC$;
точка $D$~— середина стороны~$AC$.
Прямая, проходящая через~$H$ перпендикулярно отрезку~$DH$, пересекает стороны
$AB$ и~$BC$ в~точках $E$ и~$F$.
Докажите, что $HE = HF$.
% Питерский отбор 95.9.7, 95.10.6 (С. Берлов)

\item
Окружность с~центром~$O$ вписана в~четырехугольник $ABCD$ и~касается его
не~параллельных сторон $BC$ и~$AD$ в~точках $E$ и~$F$ соответственно.
Пусть прямая~$AO$ и~отрезок~$EF$ пересекаются в~точке~$K$, прямая~$DO$
и~отрезок~$EF$~— в~точке~$N$, а~прямые $BK$ и~$CN$~— в~точке~$M$.
Докажите, что точки $O$, $K$, $M$ и~$N$ лежат на~одной окружности.
% ВМО 94.4.10.3 (М. Сонкин)

\item
В~треугольнике $ABC$ вневписанные окружности касаются сторон $AB$, $BC$, $AC$
в~точках $C_1$, $A_1$, $B_1$ соответственно.
Точка~$A'$~— точка пересечения серединных перпендикуляров к~отрезкам
$B B_1$ и~$C C_1$.
Аналогичным образом определяются точки $B'$ и~$C'$.
Точки $A'$, $B'$, $C'$ лежат внутри углов $BAC$, $ABC$, $BCA$ соответственно.
Докажите, что прямые $AA'$, $BB'$, $CC'$ пересекаются в~одной точке.

\item
На~сторонах $BC$ и~$CD$ параллелограмма $ABCD$ взяты точки $M$ и~$N$
соответственно.
Диагональ~$BD$ пересекает стороны $AM$ и~$AN$ треугольника $AMN$ соответственно
в~точках $E$ и~$F$, разбивая его на~две части.
Докажите, что эти части имеют одинаковые площади тогда и~только тогда, когда
точка~$K$, определяемая условиями $EK \parallel AD$ и $FK \parallel AB$, лежит
на~отрезке~$MN$.
% ВМО 93.4.10.7 (М. Сонкин)

\item
Через центр~$O$ окружности, описанной около неравнобедренного остроугольного
треугольника $ABC$, проведены прямые, перпендикулярные сторонам $AB$ и~$AC$.
Эти прямые пересекают высоту~$AD$ треугольника $ABC$ в~точках $P$ и~$Q$.
Точка~$M$~— середина стороны~$BC$, а~$S$~— центр окружности, описанной
около треугольника $OPQ$.
Докажите, что $\angle BAS = \angle CAM$.
% ВМО 2010.5.10.3 (Д. Прокопенко)

\item
Дан треугольник $ABC$.
Окружность~$\omega$ касается описанной окружности треугольника $ABC$
в~точке~$A$, пересекает сторону~$AB$ в~точке~$K$, а~также пересекает
сторону~$BC$.
Касательная~$CL$ к~окружности~$\omega$ такова, что отрезок~$KL$ пересекает
сторону~$BC$ в~точке~$T$.
Докажите, что отрезок $BT$ равен по~длине касательной из~точки~$B$ к~$\omega$.
% ВМО 2006.5.9.4 (Д. Скробот)

\item
Биссектрисы $B B_1$ и~$C C_1$ треугольника $ABC$ пересекаются в~точке~$I$.
Прямая~$B_1 C_1$ пересекает описанную окружность треугольника $ABC$
в~точках~$M$ и~$N$.
Докажите, что радиус описанной окружности треугольника $IMN$ вдвое больше
радиуса описанной окружности треугольника $ABC$.
% ВМО 2006.5.11.4 (Л. Емельянов)

\end{problems}

