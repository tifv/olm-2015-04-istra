% $date: 2015-04-05
% $timetable:
%   g9r1:
%     2015-04-05:
%       2:

\section*{Ортоцентр}

% $authors:
% - Фёдор Бахарев

\begin{problems}

\item
Через вершину~$B$ остроугольного треугольника $ABC$ проведено две окружности,
которые касаются стороны~$AC$ в~точках $A$ и~$C$ и~пересекаются вторично
в~точке $M$.
\\
\subproblem
Докажите, что~$M$ лежит на~медиане треугольника, выходящей из~вершины~$B$.
\\
\subproblem
Докажите, что $A$, $C$, $M$ и~ортоцентр треугольника $H$ лежат на~одной
окружности.

\item
\subproblem
Докажите, что точка, симметричная ортоцентру~$H$ треугольника $ABC$
относительно середины стороны, лежит на~описанной окружности
треугольника $ABC$.
\\
\subproblem
Докажите, что $A$, $C$, $H$ и~проекция $H$ на~медиану треугольника, выходящую
из~вершины~$B$, лежат на~одной окружности.

\item
Высоты $A A_1$, $C C_1$ остроугольного треугольника $ABC$ пересекаются
в~точке~$H$.
Прямые $A_1 C_1$ и~$AC$ пересекаются в~точке~$K$.
Докажите, что прямая~$HK$ перпендикулярна медиане треугольника $ABC$
из~вершины~$B$.

\item
В~остроугольном треугольнике $ABC$ точка~$H$~— ортоцентр,
$O$~— центр описанной окружности, $A A_1$, $B B_1$ и~$C C_1$~— высоты.
Точка~$C_2$ симметрична~$C$ относительно $A_1 B_1$.
Докажите, что $H$, $O$, $C_1$ и~$C_2$ лежат на~одной окружности.

\item
Высоты $AA'$ и~$CC'$ остроугольного треугольника $ABC$ пересекаются
в~точке~$H$.
Точка $B_0$~— середина стороны~$AC$.
Докажите, что точка пересечения прямых, симметричных $B B_0$ и~$H B_0$
относительно биссектрис углов $ABC$ и~$AHC$ соответственно, лежит
на~прямой~$A'C'$.

\item
В~остроугольном неравнобедренном треугольнике $ABC$ биссектриса угла между
высотами $A A_1$ и~$C C_1$ пересекает стороны $AB$ и~$BC$ в~точках $P$ и~$Q$
соответственно.
Биссектриса угла~$B$ пересекает отрезок, соединяющий ортоцентр
треугольника $ABC$ с~серединой стороны~$AC$, в~точке~$R$.
Докажите, что точки $P$, $B$, $Q$ и~$R$ лежат на~одной окружности.

\end{problems}

