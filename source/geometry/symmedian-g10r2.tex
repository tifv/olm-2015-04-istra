% $date: 2015-04-06
% $timetable:
%   g10r2:
%     2015-04-06:
%       1:

\section*{Симедиана}

% $authors:
% - Алексей Доледенок

\definition
Пусть точка~$S$ на стороне~$BC$ треугольника $ABC$ такова, что прямая $AS$
симметрична медиане $AM$ относительно биссектрисы угла $A$.
Тогда $AS$ называется \emph{симедианой}.

\definition
Точки $B_1$ и $C_1$ лежат на лучах $AC$ и $AB$ соответственно.
Отрезок $B_1 C_1$ называется \emph{антипараллельным} отрезку $BC$, если
$\angle A B_1 C_1 = \angle ABC$.

\begin{problems}

\item
\subproblem
Пусть $S$ --- точка на стороне~$BC$.
Тогда $AS$ --- симедиана тогда и только тогда, когда $BS : SC = AB^2 : AC^2$.
\\
\subproblem
Докажите, что $AS$ делит антипараллельный отрезок пополам тогда и только тогда,
когда $AS$~--- симедиана.
\\
\subproblem
Пусть в треугольнике $ABC$ точка~$X$ такова, что $\angle ABX = \angle XAC$,
$\angle BAX = \angle XCA$.
Докажите, что $AX$ содержит симедиану треугольника $ABC$.
\\
\subproblem
Касательные к описанной окружности треугольника $ABC$ в точках $B$ и $C$
пересекаются в точке~$P$.
Докажите, что $AP$ содержит симедиану треугольника $ABC$.

\item
В остроугольном треугольнике $ABC$ на высоте~$BK$ как на диаметре построена
окружность~$\omega$, пересекающая стороны $AB$ и $BC$ в точках $E$ и $F$.
К окружности~$\omega$ провели касательные в точках $E$ и $F$.
Докажите, что их точка пересечения лежит на прямой, содержащей медиану
треугольника $ABC$, проведенную из вершины~$B$.

\item
Касательная к описанной окружности треугольника $ABC$ в точке~$A$ пересекает
прямую~$BC$ в точке~$D$.
Касательные к описанной окружности треугольника $ACD$ в точках $A$ и $C$
пересекаются в точке~$K$.
Докажите, что прямая~$DK$ делит $AB$ пополам.

\item
Высоты $A A_1$ и $C C_1$ остроугольного треугольника $ABC$ пересекаются
в точке~$H$.
Точка~$B_0$~--- середина стороны~$AC$.
Докажите, что точка пересечения прямых, симметричных $B B_0$ и $H B_0$
относительно биссектрис углов $ABC$ и $AHC$ соответственно, лежит
на прямой~$A_1 C_1$.

\item
\subproblem
Прямая, содержащая симедиану~$AS$, пересекает описанную окружность
треугольника $ABC$ в точке~$D$.
Докажите, что $DS$~--- симедиана треугольника $DBC$.
\\
\subproblem
В окружности~$\omega$ провели две параллельных хорды $AB$ и $CD$.
$E$ --- точка пересечения $\omega$ и прямой, проходящей через $C$
и середину~$AB$.
Пусть $F$~--- середина~$DE$.
Докажите, что $\angle AFE = \angle BFE$.

\item
Даны окружность, ее хорда $AB$ и точка $W$~--- середина меньшей дуги~$AB$.
На большей дуге~$AB$ выбирается произвольно точка~$C$.
Касательная к окружности из точки~$C$ пересекает касательные из точек $A$ и $B$
в точках $X$ и $Y$ соответственно.
Прямые $WX$ и $WY$ пересекают прямую~$AB$ в точках $N$ и $M$.
Докажите, что длина отрезка~$MN$ не зависит от выбора точки~$C$.

\item
В остроугольном треугольнике $ABC$ высоты $A A_1$, $B B_1$, $C C_1$
пересекаются в точке~$H$.
Из точки~$H$ провели перпендикуляры к прямым $B_1 C_1$ и $A_1 C_1$, которые
пересекли лучи $CA$ и $CB$ в точках $P$ и $Q$ соответственно.
Докажите, что перпендикуляры из точки~$C$ на $A_1 B_1$ делит $PQ$ пополам.

\end{problems}

