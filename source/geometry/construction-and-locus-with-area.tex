% $date: 2015-03-31

% $timetable:
%   g9r2:
%     2015-03-31:
%       1:
%   g9r1:
%     2015-03-31:
%       2:

% $caption: Построения и ГМТ, связанные с площадями

\section*{%
Задачи на построения и геометрические места точек, связанные с площадями}

% $authors:
% - Александр Блинков

% $build$matter[print]: [[.], [.]]

% $matter[-preamble-package-guard]:
% - preamble package: wrapfig
% - .[preamble-package-guard]

\begin{problems}

\item
Укажите геометрическое место таких точек~$M$, лежащих внутри
треугольника $ABC$, что $S_{ACM} + S_{BCM} = S_{ABM}$.

\item
Через точку, лежащую на~стороне треугольника, проведите прямую, разбивающую
данный треугольник на~две равновеликие части.

\item
Укажите геометрическое место таких точек~$M$, лежащих в~плоскости
треугольника $ABC$, что:
\\
\subproblem $S_{ACM} = S_{BCM}$;
\quad
\subproblem $S_{ACM} = S_{BCM} = S_{ABM}$.

\item
Внутри данного треугольника $ABC$ постройте точку~$M$ так, чтобы
$S_{AMB} : S_{BMC} : S_{CMA} = 3 : 2 : 1$.

\item
Внутри параллелограмма $ABCD$ дана точка~$P$.
На~границе параллелограмма постройте точку~$Q$ так, чтобы ломаная~$APQ$
разбивала его на~две равновеликие части.

\end{problems}

\begingroup
\ifx\problemfigurewidth\undefined
\newlength\problemfigurewidth
\newlength\problemtextwidth
\newlength\spacewidth
\fi
\setlength\problemfigurewidth{2.5cm}
\setlength\spacewidth{1em}
\setlength\problemtextwidth{\linewidth}
\addtolength\problemtextwidth{-\problemfigurewidth}
\addtolength\problemtextwidth{-\spacewidth}
\begin{minipage}{\problemtextwidth}
\begin{problems}
\item
\subproblem
Внутри выпуклого четырехугольника $ABCD$ укажите какую-нибудь точку~$M$ так,
чтобы ломаная $AMC$ разбивала его на~две равновеликие части.
\\
\subproblem
Через вершину выпуклого четырехугольника проведите прямую, разбивающую его
на~две равновеликие части.
\\
\subproblem
Выпуклая фигура ограничена углом $ABC$ и~дугой $AC$ (рис. справа).
Постройте какую-нибудь прямую, разбивающую ее~на~две равновеликие части.
\end{problems}
\end{minipage}\hspace{\spacewidth}%
\begin{minipage}{\problemfigurewidth}
    \jeolmfigure[width=\linewidth]{n6c}
\end{minipage}
\endgroup

\begin{problems}

\item
\subproblem
Внутри трапеции $ABCD$ с~основаниями $AD$ и~$BC$ найдите множество точек~$M$
таких, что $S_{ADM} + S_{BCM} = \frac{1}{2} S_{ABCD}$.
\\
\subproblem
Внутри выпуклого четырехугольника $ABCD$ найдите множество точек~$M$ таких, что
$S_{ABM} + S_{CDM} = S_{ADM} + S_{BCM}$.

\item
\subproblem
Докажите, что любая прямая, делящая пополам площадь и~периметр треугольника,
проходит через центр его вписанной окружности.
\\
\subproblem
Объясните, как построить такую прямую.

\item
Некоторая кривая~$\Gamma$ делит квадрат на~две части равной площади.
Всегда~ли найдутся такие две точки $A$ и~$B$ на~этой кривой, что прямая~$AB$
проходит через центр~$O$ квадрата?

\item
В~произвольном треугольнике $ABC$ точка~$D$~--- середина стороны~$AB$.
Можно~ли так расположить точки $E$ и~$F$ на~сторонах $AC$ и~$BC$
соответственно, чтобы площадь треугольника $DEF$ оказалась больше суммы
площадей треугольников $AED$ и~$BFD$?
%(Дайте ответ для каждого возможного треугольника $ABC$.)

\end{problems}

