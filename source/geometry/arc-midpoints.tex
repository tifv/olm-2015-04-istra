% $date: 2015-04-03
% $timetable:
%   g10r2:
%     2015-04-03:
%       3:

\section*{Биссектрисы и середины дуг}

% $authors:
% - Алексей Доледенок

Дан остроугольный неравнобедренный треугольник $ABC$;
$A_1$, $B_1$, $C_1$~— середины дуг $BC$, $AC$, $AB$ описанной окружности;
$A'$, $B'$, $C'$~— середины дуг $BAC$, $ABC$, $ACB$ описанной окружности;
$I$~— центр вписанной окружности;
$I_A$, $I_B$, $I_C$~— центры вневписанных окружностей, касающихся
сторон $BC$, $AC$, $AB$.

\begin{problems}

\item\claim{Лемма о трезубце}
Докажите, что
\(
    A_1 B = A_1 I = A_1 C = A_1 I_A
\).

\item
Докажите, что прямые $A_1 A'$, $B_1 B'$, $C_1 C'$ пересекаются в~одной точке.

\item
Докажите, что $A'$~— середина отрезка $I_B I_C$.

\item
Докажите, что диагонали шестиугольника, образованного пересечением
треугольников $ABC$ и~$A_1 B_1 C_1$, параллельны сторонам треугольника $ABC$
и~пересекаются в~точке $I$.

\item
Описанная окружность треугольника $BIC$ вторично пересекает прямые $AB$ и~$AC$
в~точках $E$ и~$F$.
Докажите, что $EF$ касается вписанной окружности треугольника $ABC$.

\item
Дан треугольник $ABC$.
На~продолжениях сторон $AB$ и~$CB$ за~точку~$B$ взяты точки $P$ и~$Q$
соответственно так, что $AC = QC = PA$.
Докажите, что описанные окружности треугольников $ABQ$ и~$CBP$ пересекаются
на~биссектрисе угла~$B$.

\item
Пусть $AB < AC$.
$M$~— середина стороны~$BC$.
Докажите, что $\angle {IMB} = \angle {IA'A}$.

\item
В~треугольнике $ABC$ проведена биссектриса~$BD$
(точка~$D$ лежит на~отрезке~$AC$).
Прямая~$BD$ пересекает окружность~$\Omega$, описанную около треугольника $ABC$,
в~точках $B$ и~$E$.
Окружность~$\omega$, построенная на~отрезке~$DE$ как на~диаметре, пересекает
окружность~$\Omega$ в~точках $E$ и~$F$.
Докажите, что прямая, симметричная прямой~$BF$ относительно прямой~$BD$,
содержит медиану треугольника $ABC$.

\item
Докажите, что середина стороны~$BC$, основание высоты из~вершины~$A$
и~основания перпендикуляров из~вершин $B$ и~$C$ на~внешнюю биссектрису угла~$A$
лежат на~одной окружности.

\item
Пусть $L$~— точка пересечения $BC$ и~$A A_1$, $K$ и~$M$~— основания
перпендикуляров, опущенных из~$L$ на~$AB$ и~$AC$.
Докажите, что четырехугольник $A K A_1 M$ равновелик треугольнику $ABC$
(т.~е. имеет ту~же площадь).

\item
Пусть $H$~— основание высоты из~вершины~$A$.
Докажите, что прямая~$BC$ является биссектрисой угла $I_A H I$.

\end{problems}

