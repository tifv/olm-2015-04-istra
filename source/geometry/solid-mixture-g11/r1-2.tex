% $date: 2015-04-04
% $timetable:
%   g11r1:
%     2015-04-04:
%       1:

\section*{Добавка по стереометрии}

% $authors:
% - Фёдор Ивлев

% $matter[full-version,no-header]:
% - .[-full-version]
% - ../r1-2-more[contained]

\begin{problems}

%\item
%Докажите, что всякая плоскость, проходящая через середины двух противоположных
%ребер тетраэдра, делит его объем пополам.
%% Вестник элементарной математики №344, задача №330

\item
Окружность с~центром~$I$, вписанная в~грань $ABC$ треугольной пирамиды $SABC$,
касается отрезков $AB$, $BC$, $CA$ в~точках $D$, $E$, $F$ соответственно.
На~отрезках $SA$, $SB$, $SC$ отмечены соответственно точки $A'$, $B'$, $C'$
так, что $AA' = AD$, $BB' = BE$, $CC' = CF$;
$S'$~--- точка на~описанной сфере пирамиды, диаметрально противоположная
точке~$S$.
Известно, что $SI$ является высотой пирамиды.
Докажите, что точка~$S'$ равноудалена от~точек $A'$, $B'$, $C'$.
% Ф. Бахарев, Всерос 06.5.11.6

\item
Точка~$O$ лежит в~основании $A_1 A_2 \ldots A_n$ пирамиды $S A_1 \ldots A_n$,
причем $S A_1 = S A_2 = \ldots = S A_n$
и~$\angle S A_1 O = \angle S A_2 O = \ldots = \angle S A_n O$.
При каком наименьшем~$n$ отсюда следует, что $SO$~--- высота пирамиды?

\item
Существует~ли треугольная пирамида, каждое ребро основания которой видно
из~середины противолежащего бокового ребра под прямым углом?
% ВМО 1996.3.11.2 (Н. Агаханов)

\item
Точки $A_1$, $B_1$, $C_1$, $D_1$~--- середины ребер $SA$, $SB$, $SC$, $SD$
пирамиды $SABCD$.
Известно, что отрезки $A C_1$, $B D_1$, $C A_1$, $D B_1$ проходят через одну
точку и~имеют равные длины.
Докажите, что $ABCD$~--- прямоугольник.
% ВМО 1997.3.11.6 (Н. Агаханов)

\item
Дана пирамида $SABCD$, в~основании которой лежит выпуклый четырехугольник
$ABCD$.
В~пирамиду вписана сфера, касающаяся грани $ABCD$ в~точке~$P$.
Докажите, что $\angle APB + \angle CPD = 180^\circ$.

\item
Сфера, вписанная в~тетраэдр, касается одной из~его граней в~точке пересечения
биссектрис, другой~--- в~точке пересечения высот, третьей~--- в~точке
пересечения медиан.
Докажите, что тетраэдр правильный.
% ВМО 1997.5.11.7 (Н. Агаханов)

\item
В~тетраэдре $ABCD$ определим
точку~$H_{a}$ как проекцию вершины~$A$ на~плоскость $BCD$,
точку~$H_{ac}$ как проекцию $H_{a}$ на~прямую~$AC$,
аналогично определим другие такие точки.
Докажите, что если плоские углы при вершине~$A$ равны, то~точки
$H_{ab}$, $H_{ac}$, $H_{ad}$, $H_{ba}$, $H_{ca}$ и~$H_{da}$ лежат на~одной
сфере.
% ФЮМ 2008, первый тур, номер 6, лига стратегий

\itemx{*}
Сфера с~центром в~плоскости основания $ABC$ тетраэдра $SABC$ проходит через
вершины $A$, $B$ и~$C$ и~вторично пересекает ребра $SA$, $SB$ и~$SC$ в~точках
$A_1$, $B_1$ и~$C_1$ соответственно.
Плоскости, касающиеся сферы в~точках $A_1$, $B_1$ и~$C_1$, пересекаются
в~точке~$O$.
Докажите, что $O$~--- центр сферы, описанной около тетраэдра $S A_1 B_1 C_1$.
% сложная (Л. Емельянов) Всерос 01.5.11.8

\end{problems}

