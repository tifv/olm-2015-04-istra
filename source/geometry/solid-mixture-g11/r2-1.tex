% $date: 2015-04-02
% $timetable:
%   g11r2:
%     2015-04-02:
%       1:

\section*{Стереометрия}

% $authors:
% - Фёдор Ивлев

\begin{problems}

%\item
%Докажите, что всякая плоскость, проходящая через середины двух противоположных
%ребер тетраэдра, делит его объем пополам.
%% Вестник элементарной математики №344, задача №330

\item
Высота четырехугольной пирамиды $SABCD$ проходит через точку пересечения
диагоналей ее~основания $ABCD$.
Из~вершин основания опущены перпендикуляры $A A_1$, $B B_1$, $C C_1$, $D D_1$
на~прямые $SC$, $SD$, $SA$ и~$SB$ соответственно.
Оказалось, что точки $S$, $A_1$, $B_1$, $C_1$, $D_1$ различны и~лежат на~одной
сфере.
Докажите, что прямые $A A_1$, $B B_1$, $C C_1$, $D D_1$ проходят через одну
точку.

\item
Вписанная и~вневписанная сферы треугольной пирамиды $ABCD$ касаются ее~грани
$BCD$ в~различных точках $X$ и~$Y$.
Докажите, что треугольник $AXY$ тупоугольный.
% ВМО 2013.5.11.2

\item
Окружность с~центром~$I$, вписанная в~грань $ABC$ треугольной пирамиды $SABC$,
касается отрезков $AB$, $BC$, $CA$ в~точках $D$, $E$, $F$ соответственно.
На~отрезках $SA$, $SB$, $SC$ отмечены соответственно точки $A'$, $B'$, $C'$
так, что $AA' = AD$, $BB' = BE$, $CC' = CF$;
$S'$~— точка на~описанной сфере пирамиды, диаметрально противоположная
точке~$S$.
Известно, что $SI$ является высотой пирамиды.
Докажите, что точка~$S'$ равноудалена от~точек $A'$, $B'$, $C'$.
% Ф. Бахарев, Всерос 06.5.11.6

%\item
%Точка~$O$ лежит в~основании $A_1 A_2 \ldots A_n$ пирамиды $S A_1 \ldots A_n$,
%причем $S A_1 = S A_2 = \ldots = S A_n$
%и~$\angle S A_1 O = \angle S A_2 O = \ldots = \angle S A_n O$.
%При каком наименьшем~$n$ отсюда следует, что $SO$~— высота пирамиды?

\item
В~тетраэдре $ABCD$ из~вершины~$A$ опустили перпендикуляры $AB'$, $AC'$, $AD'$
на~плоскости, делящие двугранные углы при ребрах $CD$, $DB$, $BC$ пополам.
Докажите, что плоскости $BCD$ и~$B'C'D'$ параллельны.

\item
Высоты тетраэдра пересекаются в~одной точке.
Докажите, что эта точка, основание одной из~высот, а~также точки, делящие
остальные высоты в~отношении $2 : 1$, считая от~вершин, лежат на~одной сфере.

%\item
%Могут~ли четыре центра вписанных окружностей граней тетраэдра лежать в~одной
%плоскости?
%% Всерос 10.5.11.6

%\item
%Дан правильный тетраэдр $ABCD$.
%На~ребре~$CD$ берутся всевозможные точки~$M$.
%Докажите, что ортоцентры всех треугольников $AMB$ лежат на~одной окружности.
%% ФЮМ 2008, четвертый тур, 9 задача, лига стратегий

\item
Точка~$O$~— основание высоты четырехугольной пирамиды.
Сфера с~центром~$O$ касается всех боковых граней пирамиды.
Точки $A$, $B$, $C$ и~$D$ взяты последовательно на~боковых ребрах пирамиды так,
что отрезки $AB$, $BC$ и~$CD$ проходят через три точки касания сферы с~гранями.
Докажите, что отрезок $AD$ проходит через четвертую точку касания.
% (М. Смуров), всерос 93.4.11.3

\item
В~пространстве даны прямая~$l$ и~точка~$A$, не~лежащая не~ней.
$XY$~— общий перпендикуляр к~прямой~$l$ и~произвольной прямой~$AY$
($X$ лежит на~$l$).
Найдите ГМТ $Y$ по~всем возможным прямым, проходящим через~$A$.

\item
На~боковых ребрах $SA$, $SB$ и~$SC$ правильной треугольной пирамиды $SABC$
взяты соответственно точки $A_1$, $B_1$ и~$C_1$ так, что плоскости
$A_1 B_1 C_1$ и~$ABC$ параллельны.
Пусть $O$~— центр сферы, проходящей через точки $S$, $A$, $B$ и~$C_1$.
Докажите, что прямая~$SO$ перпендикулярна плоскости $A_1 B_1 C$.
% ВМО 95.4.11.7 (Д. Терёшин)

\end{problems}

