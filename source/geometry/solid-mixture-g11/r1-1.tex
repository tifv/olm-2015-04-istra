% $date: 2015-04-02
% $timetable:
%   g11r1:
%     2015-04-02:
%       2:

\section*{Разнобой по стереометрии}

% $authors:
% - Фёдор Ивлев

\begin{problems}

\item
Дан выпуклый многогранник~$P$.
Из~каждой грани во~вне этого многогранника проведен вектор по~модулю равный
площади этой грани, а~по~направлению перпендикулярный ей.
Докажите, что сумма всех проведенных векторов равна нулю.

\item
Могут~ли четыре центра вписанных окружностей граней тетраэдра лежать в~одной
плоскости?
% Всерос 10.5.11.6

\item
В~тетраэдре провели четыре отрезка, соединяющие вершины с~центрами вписанных
окружностей противоположных граней.
Докажите, что если два таких отрезка пересекаются, то~два других тоже.

\item
Многогранник называется \emph{кубоподобным}, если у~него 8 вершин, 6 граней,
каждая из~которых является четырехугольником, и~в~каждой вершине сходится
по~три грани.
Докажите, что если 7 вершин кубоподобного многогранника лежат на~одной сфере,
то~и~восьмая тоже лежит на~этой~же сфере.

\item
Высоты тетраэдра пересекаются в~одной точке.
Докажите, что эта точка, основание одной из~высот, а~также точки, делящие
остальные высоты в~отношении $2 : 1$, считая от~вершин, лежат на~одной сфере.

\item
На~боковых ребрах $SA$, $SB$ и~$SC$ правильной треугольной пирамиды $SABC$
взяты соответственно точки $A_1$, $B_1$ и~$C_1$ так, что плоскости
$A_1 B_1 C_1$ и~$ABC$ параллельны.
Пусть $O$~--- центр сферы, проходящей через точки $S$, $A$, $B$ и~$C_1$.
Докажите, что прямая~$SO$ перпендикулярна плоскости $A_1 B_1 C$.
% ВМО 95.4.11.7 (Д. Терёшин)

%\item
%Точка~$O$~--- основание высоты четырехугольной пирамиды.
%Сфера с~центром~$O$ касается всех боковых граней пирамиды.
%Точки $A$, $B$, $C$ и~$D$ взяты последовательно на~боковых ребрах пирамиды так,
%что отрезки $AB$, $BC$ и~$CD$ проходят через три точки касания сферы с~гранями.
%Докажите, что отрезок $AD$ проходит через четвертую точку касания.
%% (М. Смуров), всерос 93.4.11.3

\item
Сфера касается всех граней многогранника.
Назовем грань многогранника \emph{большой}, если проекция сферы на~эту грань
полностью в~ней содержится.
Докажите, что больших граней не~больше 6.

\item
Дан правильный тетраэдр $ABCD$.
На~ребре~$CD$ берутся всевозможные точки~$M$.
Докажите, что ортоцентры всех треугольников $AMB$ лежат на~одной окружности.
% ФЮМ 2008, четвертый тур, 9 задача, лига стратегий

\end{problems}



\iffalse

\subsection*{Добавка по~стереометрии}

\begin{problems}

\item
В~тетраэдре $ABCD$ определим
точку~$H_{a}$ как проекцию вершины~$A$ на~плоскость $BCD$,
точку~$H_{ac}$ как проекцию $H_{a}$ на~прямую~$AC$,
аналогично определим другие такие точки.
Докажите, что если плоские углы при вершине~$A$ равны, то~точки
$H_{ab}$, $H_{ac}$, $H_{ad}$, $H_{ba}$, $H_{ca}$ и~$H_{da}$ лежат на~одной
сфере.
% ФЮМ 2008, первый тур, номер 6, лига стратегий

\item
Вписанная в~тетраэдр $SABC$ сфера~$\omega$ касается его боковых граней
$SBC$, $SCA$ и~$SAB$ в~точках $A_1$, $B_1$ и~$C_1$ соответственно.
Прямая~$A A_1$ вторично пересекает сферу~$\omega$ в~точке~$A_2$,
прямая~$S A_2$ вторично пересекает сферу~$\omega$ в~точке~$A_3$, наконец,
прямая $A_1 A_3$ пересекает ребро $AS$ в~точке~$A'$.
Точки $B'$ и~$C'$ определяются аналогично.
Докажите, что плоскость $A'B'C'$ касается сферы~$\omega$.
% ФЮМ 2009, третий тур, номер 6, лига стратегий

\item
Вписанная в~тетраэдр $ABCD$ сфера касается его граней
$ABC$, $ABD$, $ACD$ и~$BCD$ в~точках $D_1$, $C_1$, $B_1$ и~$A_1$
соответственно.
Рассмотрим плоскость, равноудаленную от~точки~$A$ и~плоскости $B_1 C_1 D_1$
и~три другие аналогично построенные плоскости.
Докажите, что тетраэдр, образованный этими четырьмя плоскостями, имеет тот~же
центр описанной сферы, что и~тетраэдр $ABCD$.
% сложная (Ф. Бахарев) Всерос 03.5.11.8

\item
Докажите, что при $n \geq 5$ сечение пирамиды, в~основании которой лежит
правильный $n$-угольник, не~может являться правильным $(n + 1)$-угольником.
% средняя - сложная (Н. Агаханов, Д. Терёшин) всерос 97.5.11.3

\item
Сфера с~центром в~плоскости основания $ABC$ тетраэдра $SABC$ проходит через
вершины $A$, $B$ и~$C$ и~вторично пересекает ребра $SA$, $SB$ и~$SC$ в~точках
$A_1$, $B_1$ и~$C_1$ соответственно.
Плоскости, касающиеся сферы в~точках $A_1$, $B_1$ и~$C_1$, пересекаются
в~точке~$O$.
Докажите, что $O$~--- центр сферы, описанной около тетраэдра $S A_1 B_1 C_1$.
% сложная (Л. Емельянов) Всерос 01.5.11.8

\item
На~ребрах $AB$, $BC$, $CD$, $DA$ тетраэдра $ABCD$ взяты точки
$K$, $L$, $M$, $N$ соответственно.
Точки $K'$, $L'$, $M'$, $N'$ симметричны точкам $K$, $L$, $M$, $N$ относительно
середин ребер $AB$, $BC$, $CD$, $DA$ соответственно.
Докажите, что объемы тетраэдров $KLMN$ и~$K'L'M'N'$ равны.
% колмогоровский турнир 2007 г., высшая лига, первый тур, номер 7

\end{problems}

\fi

