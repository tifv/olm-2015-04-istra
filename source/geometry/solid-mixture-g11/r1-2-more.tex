% $date: 2015-04-04
% $timetable:
%   g11r1:
%     2015-04-09:
%       2:

\section*{Добавка по стереометрии (ещё)}

% $authors:
% - Фёдор Ивлев

% $matter[-contained,no-header]:
% - verbatim: \setproblem{8}
% - .[contained]

\begin{problems}

\item
Вписанная в~тетраэдр $SABC$ сфера~$\omega$ касается его боковых граней
$SBC$, $SCA$ и~$SAB$ в~точках $A_1$, $B_1$ и~$C_1$ соответственно.
Прямая~$A A_1$ вторично пересекает сферу~$\omega$ в~точке~$A_2$,
прямая~$S A_2$ вторично пересекает сферу~$\omega$ в~точке~$A_3$, наконец,
прямая $A_1 A_3$ пересекает ребро $AS$ в~точке~$A'$.
Точки $B'$ и~$C'$ определяются аналогично.
Докажите, что плоскость $A'B'C'$ касается сферы~$\omega$.
% ФЮМ 2009, третий тур, номер 6, лига стратегий

\item
Докажите, что при $n \geq 5$ сечение пирамиды, в~основании которой лежит
правильный $n$-угольник, не~может являться правильным $(n + 1)$-угольником.
% средняя - сложная (Н. Агаханов, Д. Терёшин) всерос 97.5.11.3

\itemx{*}
На~ребрах $AB$, $BC$, $CD$, $DA$ тетраэдра $ABCD$ взяты точки
$K$, $L$, $M$, $N$ соответственно.
Точки $K'$, $L'$, $M'$, $N'$ симметричны точкам $K$, $L$, $M$, $N$ относительно
середин ребер $AB$, $BC$, $CD$, $DA$ соответственно.
Докажите, что объемы тетраэдров $KLMN$ и~$K'L'M'N'$ равны.
% колмогоровский турнир 2007 г., высшая лига, первый тур, номер 7

\itemx{*}
Вписанная в~тетраэдр $ABCD$ сфера касается его граней
$ABC$, $ABD$, $ACD$ и~$BCD$ в~точках $D_1$, $C_1$, $B_1$ и~$A_1$
соответственно.
Рассмотрим плоскость, равноудаленную от~точки~$A$ и~плоскости $B_1 C_1 D_1$
и~три другие аналогично построенные плоскости.
Докажите, что тетраэдр, образованный этими четырьмя плоскостями, имеет тот~же
центр описанной сферы, что и~тетраэдр $ABCD$.
% сложная (Ф. Бахарев) Всерос 03.5.11.8

\itemx{*}
Даны точки $A_1$, $A_2$, $A_3$, $A_4$ общего положения в~пространстве такие,
что
$A_1 A_2 \cdot A_3 A_4 = A_1 A_3 \cdot A_2 A_4 = A_1 A_4 \cdot A_2 A_3$.
Обозначим через $O_i$ центр описанной окружности треугольника $A_{jkl}$, где
$\{ i, j, k, l \} = \{ 1, 2, 3, 4 \}$.
Докажите, что прямые $A_i O_i$ пересекаются в~одной точке или параллельны.
% RMMO 2009.3 (of 4)

\end{problems}


