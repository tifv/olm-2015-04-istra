% $date: 2015-04-07
% $timetable:
%   g11r2:
%     2015-04-07:
%       3:

\section*{Добавка по стереометрии}

% $authors:
% - Фёдор Ивлев

\begin{problems}

\item
Сфера~$\omega$ проходит через вершину~$S$ неправильной пирамиды $SABC$
и~пересекает ребра $SA$, $SB$ и~$SC$ вторично в~точках $A_1$, $B_1$ и~$C_1$
соответственно.
Окружность пересечения сферы~$\omega$ с~описанной сферой~$\Omega$ пирамиды
$SABC$ лежит в~плоскости, параллельной плоскости $ABC$.
Точки $A_2$, $B_2$ и~$C_2$ симметричны точкам $A_1$, $B_1$ и~$C_1$ относительно
середин ребер $SA$, $SB$ и~$SC$ соответственно.
Докажите, что точки $A$, $B$, $C$, $A_2$, $B_2$ и~$C_2$ лежат на~одной сфере.

\item
В~тетраэдре провели четыре отрезка, соединяющие вершины с~центрами вписанных
окружностей противоположных граней.
Докажите, что если два таких отрезка пересекаются, то~два других тоже.

\item
Существует~ли треугольная пирамида, каждое ребро основания которой видно
из~середины противолежащего бокового ребра под прямым углом?
% ВМО 1996.3.11.2 (Н. Агаханов)

\item
Точки $A_1$, $B_1$, $C_1$, $D_1$~--- середины ребер $SA$, $SB$, $SC$, $SD$
пирамиды $SABCD$.
Известно, что отрезки $A C_1$, $B D_1$, $C A_1$, $D B_1$ проходят через одну
точку и~имеют равные длины.
Докажите, что $ABCD$~--- прямоугольник.
% ВМО 1997.3.11.6 (Н. Агаханов)

\item
Точка~$O$~--- основание высоты четырехугольной пирамиды.
Сфера с~центром~$O$ касается всех боковых граней пирамиды.
Точки $A$, $B$, $C$ и~$D$ взяты последовательно на~боковых ребрах пирамиды так,
что отрезки $AB$, $BC$ и~$CD$ проходят через три точки касания сферы с~гранями.
Докажите, что отрезок $AD$ проходит через четвертую точку касания.
% (М. Смуров), всерос 93.4.11.3

\end{problems}

