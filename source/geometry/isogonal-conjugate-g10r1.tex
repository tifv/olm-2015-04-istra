% $date: 2015-04-04
% $timetable:
%   g10r1:
%     2015-04-04:
%       3:

\section*{Изогональное сопряжение}

% $authors:
% - Фёдор Бахарев

\begin{problems}

\item
\subproblem
Докажите, что точка пересечения высот и~центр описанной окружности треугольника
изогонально сопряжены.
\\
\subproblem
Докажите, что изогонально сопряжены точка, из~которой стороны треугольника
видны под углом $120^\circ$ \emph{(точка Торричелли)} и~точка, проекции которой
на~стороны треугольника образуют равносторонний треугольник
\emph{(изодинамический центр треугольника)}.
\\
\subproblem
Касательные к~описанной окружности треугольника $ABC$ в~точках $B$ и~$C$
пересекаются в~точке~$P$.
Точка~$Q$ симметрична точке~$A$ относительно середины отрезка~$BC$.
Докажите, что $P$ и~$Q$ изогонально сопряжены относительно треугольника $ABC$.
\\
\subproblem
Докажите, что изогонально сопряжены точка пересечения медиан треугольника
и~точка, для которой сумма квадратов расстояний до~его сторон минимальна
(точка Лемуана).

\item
\subproblem
Пусть точки $P$ и~$Q$ изогонально сопряжены относительно треугольника $ABC$.
Пусть $x$, $y$ и~$z$~--- расстояния от~точки~$P$ до~прямых $BC$, $CA$ и~$AB$
соответственно, а~$x'$, $y'$ и~$z'$~--- расстояния от~точки~$Q$
до~прямых $BC$, $CA$ и~$AB$ соответственно.
Докажите, что $x \cdot x' = y \cdot y' = z \cdot z'$.
\\
\subproblem\label{geometry/isogonal-conjugate-g10r1:same-circle:problem}%
Опустим из~точки~$P$ перпендикуляры на~стороны треугольника
(или их~продолжения) и~рассмотрим окружность, проходящую через основания
перпендикуляров.
Докажите, что эта окружность совпадает с~окружностью, построенной таким~же
образом для точки~$Q$.
\\
\subproblem
Выведите из~пункта~\ref{geometry/isogonal-conjugate-g10r1:same-circle:problem},
что основания высот треугольника и~середины его сторон лежат на~одной
окружности.

\item
\subproblem
В~окружность вписан шестиугольник $ABCDEF$.
Отрезок~$AC$ пересекается с~отрезком~$BF$ в~точке~$X$,
$BE$ с~$AD$~--- в~точке~$Y$,
$CE$ и~$DF$~--- в~точке~$Z$.
Докажите, что треугольники $ABY$ и~$EDY$ подобны, причем точке, изогонально
сопряженной точке~$X$ в~треугольнике $ABY$ соответствует точка~$Z$
в~треугольнике $EDY$.
\\
\subproblem
Докажите, что точки $X$, $Y$ и~$Z$ лежат на~одной прямой.

\item
Ортоцентр~$H$ остроугольного треугольника $ABC$ отразили относительно сторон
и~получили треугольник $A_1 B_1 C_1$.
Треугольники $ABC$ и~$A_1 B_1 C_1$ в~пересечении образуют шестиугольник.
Докажите, что его диагонали пересекаются в~одной точке.

\item
В~трапеции $ABCD$ боковая сторона~$CD$ перпендикулярна основаниям,
$O$~--- точка пересечения диагоналей.
На~описанной окружности треугольника $OCD$ взята точка~$S$, диаметрально
противоположная точке~$O$.
Докажите, что углы $\angle BSC$ и~$\angle ASD$ равны.

\item
На~меньшей дуге~$AC$ описанной окружности остроугольного треугольника $ABC$
выбрана точка~$D$.
На~стороне~$AC$ нашлась такая точка~$E$, что $DE = AE$.
На~прямой, параллельной~$AB$, проходящей через~$E$, отмечена точка~$F$ такая,
что $CF = BF$.
Докажите, что точки $D$, $E$, $C$, $F$ лежат на~одной окружности.

\item
Четырехугольник $ABCD$ вписан в~окружность~$\omega$.
Окружность~$\omega_1$ касается прямых $AB$ и~$CD$ в~точках $X$ и~$Y$
и~пересекает дугу~$AD$ окружности~$\omega$ в~точках $K$ и~$L$.
Прямая~$XY$ пересекает $AC$ и~$BD$ в~точках $Z$ и~$T$.
Докажите, что $K$, $L$, $Z$ и~$T$ лежат на~одной окружности, касающейся
прямых $AC$ и~$BD$.

\item
Докажите, что при изогональном сопряжении окружность, проходящая через
вершины $B$ и~$C$, отличная от~описанной, переходит в~окружность, проходящую
через $B$ и~$C$.

\end{problems}

