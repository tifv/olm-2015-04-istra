% $date: 2015-03-30
% $timetable:
%   g11r2:
%     2015-03-30:
%       3:

\section*{Планиметрический разнобой}

% $authors:
% - Андрей Кушнир

\begin{problems}

\item
В~треугольнике $ABC$ отметили центр~$O$ описанной окружности
и~среднюю линию~$MN$, параллельную~$BC$.
Точка~$X$ такова, что $OMXN$~--- параллелограмм.
Докажите, что $AX \perp BC$.
% Интересные и~не~очень свойства ортоцентра.

\item
На~сторонах $AB$ и~$AC$ отмечены точки $C_1$ и~$B_1$ соответственно так, что
$B C_1 + C B_1 = BC$.
Докажите, что середина отрезка~$B_1 C_1$ лежит на~прямой, проходящей через
точки касания вписанной в~треугольник окружности со~сторонами $AB$ и~$AC$.
% Воробьи, линейность, теорема Симсона.

\item
Точки $O$ и~$I$~--- центры соответственно описанной и~вписанной окружностей
равнобедренного треугольника $ABC$ ($AB = AC$).
Окружности, описанные около треугольников $ABC$ и~$OIB$, пересекаются в~точках
$B$ и~$D$.
Докажите, что прямая~$AD$ касается окружности, описанной около
треугольника $OIB$.
% Найди 4 точки на~одной окружности.

\item
К~двум непересекающимся окружностям $\omega_1$ и~$\omega_2$ проведена общая
касательная, касающаяся их~в~точках $A$ и~$B$ соответственно.
Окружность, построенная на~$AB$ как на~диаметре, вторично пересекает $\omega_1$
и~$\omega_2$ в~точках $D$ и~$C$ соответственно.
Докажите, что $AC$ и~$BD$ пересекаются на~линии, соединяющей центры $\omega_1$
и~$\omega_2$.
% Гомотетия + обопри прямой уголок на~окружность.

\item
В~неравнобедренный треугольник $ABC$ вписана окружность с~центром~$I$,
касающаяся его сторон $AB$, $BC$, $CA$ в~точках $C_1$, $A_1$, $B_1$
соответственно.
Докажите, что описанные окружности треугольников
$A I A_1$, $B I B_1$, $C I C_1$ имеют общую хорду.
% Инверсия.

\item
Две окружности с~центрами $O_1$ и~$O_2$ пересекаются в~точках $P$ и~$Q$.
Касательная к~первой окружности, восстановленная в~точке~$P$, пересекает
касательную ко~второй окружности, восстановленную в~точке~$Q$, в~точке~$X$.
Докажите, что углы $PXQ$ и~$O_1 X O_2$ имеют общую биссектрису.
% Хитрое преобразование подобия.

\item
Две окружности пересекаются в~точках $P$ и~$Q$.
Прямая~$\ell$ пересекает первую из~них в~точках $A$ и~$B$, вторую~--- в~$C$
и~$D$, причем точки на~прямой в~указанном порядке и~$P$ и~$Q$ лежат в~одной
полуплоскости относительно $\ell$;
$P$ ближе к~$\ell$, чем $Q$.
Оказалось, что $AB = CD$.
Докажите, что $PB \cdot QD = QA \cdot PC$.
% Степень точки, радикальные оси, подобия.

\item
В~остроугольном треугольнике $ABC$ отметили середины $B_0$ и~$C_0$ меньших дуг
$AC$ и~$AB$ описанной окружности соответственно, $I$~--- центр вписанной
в~треугольник окружности.
Окружности $\omega_B$ и~$\omega_C$ имеют центры $B_0$ и~$C_0$ и~касаются сторон
$AC$ и~$AB$ соответственно.
Докажите, что $I$ лежит на~одной из~общих внешних касательных к~$\omega_B$
и~$\omega_C$.
% Трезубец.

\item
Окружность с~центром~$I$, вписанная в~треугольник $ABC$, касается его сторон
$AC$ и~$AB$ в~точка $B_1$ и~$C_1$ соответственно.
Точка~$K$ на~отрезке $B_1C_1$ такова, что $IK \perp BC$.
Докажите, что прямая~$AK$ делит $BC$ пополам.

% Классика жанра. Считается, радикальная ось + гомотетия, Брианшон,
% проективное преобразование.

\end{problems}

