% $date: 2015-04-01
% $timetable:
%   g9r2:
%     2015-04-01:
%       1:
%   g9r1:
%     2015-04-01:
%       2:

\section*{Теорема и неравенство Птолемея}

% $authors:
% - Александр Блинков

\begin{problems}

\item\emph{Задача Птолемея.}
В треугольнике $ABC$: $|BC| = a$, $|AC| = b$.
Найдите $|AB|$, если радиус окружности, описанной около $ABC$, равен $R$.

\item
Биссектриса угла $A$ треугольника $ABC$ пересекает описанную вокруг него
окружность в точке $W$.
\\
\subproblem
Выразите отношение $|AW| : |IW|$ через длины сторон треугольника
($I$~--- центр окружности, вписанной в треугольник $ABC$).
\\[0.5ex]
\subproblem
Докажите, что $|AW| > \cfrac{|AB| + |AC|}{2}$.

\item
На гипотенузе $AB$ прямоугольного треугольника $ABC$ во внешнюю сторону
построен квадрат, $O$~--- его центр.
Найдите $|OC|$, если $a$ и $b$~--- катеты треугольника.

% spell Помпе\'{ю}

\item\emph{(Теорема Помпе\'{ю})}
Точка~$M$ лежит на окружности, описанной около равностороннего
треугольника $ABC$.
\\
\subproblem
Докажите, что сумма расстояний от $M$ до двух вершин треугольника равна
расстоянию от $M$ до третьей вершины.
\\
\subproblem
Укажите все такие точки~$X$ плоскости, что из отрезков $XA$, $XB$ и $XC$ можно
составить треугольник.

\item
Сумма расстояний от точки~$X$, выбранной вне квадрата, до двух его ближайших
соседних вершин равна~$m$.
Найдите наибольшее значение суммы расстояний от $X$ до двух других вершин
квадрата.

\item
\subproblem
Точки $A$, $B$, $C$ и $D$~--- четыре последовательные вершины правильного
семиугольника.
Докажите, что
\[
    \frac{1}{|AB|} = \frac{1}{|AC|} + \frac{1}{|AD|}
\;.\]
\\
\subproblem
Докажите, что
\[
    \frac{1}{\sin (\pi / 7)}
=
    \frac{1}{\sin (2 \pi / 7)}
    +
    \frac{1}{\sin (3 \pi / 7)}
\;.\]

\item
В выпуклом шестиугольнике $ABCDEF$: $|AB| = |BC| = a$, $|CD| = |DE| = b$,
$|EF| = |FA| = c$.
Докажите, что
\[
    \frac{a}{|BE|} + \frac{b}{|AD|} + \frac{c}{|CF|}
\geq
    \frac{3}{2}
\;.\]

\item
Объясните, как построить четырехугольник $ABCD$ если даны его стороны
и известно, что он вписанный.

\item
В остроугольном треугольнике $ABC$ обозначим $d_1$, $d_2$ и $d_3$~---
расстояния от центра~$O$ описанной окружности до сторон.
Докажите, что $d_1 + d_2 + d_3 = R + r$, где $R$ и $r$~--- радиусы описанной
и вписанной окружностей данного треугольника.

\end{problems}

