% $date: 2015-04-06
% $timetable:
%   g10r1:
%     2015-04-06:
%       1:

\section*{Повторение (геометрия)}

% $authors:
% - Фёдор Бахарев

\begin{problems}

\item
В~треугольнике $ABC$ точка~$M$~--- середина стороны $BC$;
отрезки $A A_1$, $B B_1$, $C C_1$~--- высоты.
Прямые $AB$ и~$A_1 B_1$ пересекаются в~точке $X$, а~прямые $M C_1$ и~$AC$~---
в~точке~$Y$.
Докажите, что $XY \parallel BC$.

\item
В~остроугольном треугольнике $ABC$ точки $I_a$ и~$I_c$~--- центры вневписанных
окружностей, $H$~--- основание высоты из~вершины~$B$.
Прямая~$I_a H$ пересекает~$BC$ в~точке~$A'$, а~прямая~$I_c H$ пересекает~$AB$
в~точке~$C'$.
Докажите, что $A'C'$ проходит через центр вписанной окружности
треугольника $ABC$.

\item
Дан треугольник $ABC$ и~точка~$P$.
Прямая $\ell_A$ проходит через проекции точки $P$ на~сторону $BC$ и~высоту
из~вершины~$A$.
Аналогично определяются прямые $\ell_B$ и~$\ell_C$.
Докажите, что прямые $\ell_A$, $\ell_B$ и~$\ell_C$ пересекаются в~одной точке.

\item
Окружность касается сторон треугольника $AB$ и~$BC$ треугольника $ABC$
в~точках $D$ и~$E$, а~также внутренним образом описанной окружности
треугольника.
Докажите, что $DE$ проходит через центр вписанной окружности
треугольника $ABC$.

\item
$ABCD$~--- вписанный четырехугольник.
$M$~--- точка пересечения его диагоналей.
Некоторая прямая, проходящая через точку~$M$, пересекает окружность, описанную
около $ABCD$, в~точках $M_1$ и~$M_2$, и~окружности, описанные около
треугольников $ABM$ и~$CDM$, в~точках $N_1$ и~$N_2$.
Докажите, что $M_1 N_1 = M_2 N_2$.

\end{problems}

