% $date: 2015-03-31
% $timetable:
%   g10r1:
%     2015-03-31:
%       2:

\section*{Неравенство Йенсена}

% $authors:
% - Андрей Меньщиков

\begingroup \def\piconst{\mathrm{\pi}}

\definition
Функция $f \colon [a; b] \to \mathbb{R}$ называется \emph{выпуклой вниз}
на~отрезке $[a; b]$, если для любых $x, y \in [a; b]$ и~любых
$\alpha, \beta \geq 0$ таких, что $\alpha + \beta = 1$, выполняется
\(
    f(\alpha \cdot x + \beta \cdot y)
\leq
    \alpha \cdot f(x)+ \beta \cdot f(y)
\).
Функция~$f$ называется \emph{выпуклой вверх} на~отрезке $[a; b]$, если при
тех~же условиях выполняется аналогичное неравенство со~знаком~$\geq$.

\theorem
Если для всех $x \in (a; b)$ выполняется $f''(x) \geq 0$, то~функция~$f$
выпукла вниз на~отрезке $[a; b]$.
Если для всех $x \in (a; b)$ выполняется $f''(x) \leq 0$, то~функция~$f$
выпукла вверх на~отрезке $[a; b]$.

\begin{problems}

\item\claim{Неравенство Йенсена}
Пусть функция~$f$ выпукла вниз на~отрезке $[a;b]$,
\begin{gather*}
    x_1, x_2, \ldots, x_n \in [a; b]
,\quad
    \alpha_1, \alpha_2, \ldots, \alpha_n \geq 0
\quad\text{и}\quad
    \alpha_1 + \alpha_2 + \ldots + \alpha_n = 1
\, . \end{gather*}
Тогда справедливо неравенство
\[
    f(\alpha_1 \cdot x_1 + \ldots + \alpha_n \cdot x_n)
\leq
    \alpha_1 \cdot f(x_1) + \ldots + \alpha_n \cdot f(x_n)
\; . \]
Для выпуклой вверх функции выполнено аналогичное неравенство со~знаком~$\geq$.

\item
Пусть $x_1, x_2, \ldots, x_n \in [0; \piconst]$.
Докажите неравенство
\[
    \frac{\sin(x_1) + \ldots + \sin(x_n)}{n}
\leq
    \sin\left(
        \frac{x_1 + \ldots + x_n}{n}
    \right)
\; . \]

\item\emph{Неравенство Коши.}
Используя неравенство Йенсена для $y = \ln x$, докажите для неотрицательных
$x_1, x_2, \ldots, x_n$ неравенство:
\[
    \frac{x_1 + \ldots + x_n}{n}
\geq
    \sqrt[n]{x_1 \cdot \ldots \cdot x_n}
\; . \]

\item\emph{Неравенство о~среднем степенном.}
Используя неравенство Йенсена для $y = x^{\alpha}$, докажите для
$x_1, x_2, \ldots, x_n \geq 0$, $\alpha > 1$ неравенство:
\[
    \left(
        \frac{
            x_1^{\alpha} + x_2^{\alpha} + \ldots + x_n^{\alpha}
        }{n}
    \right)^{\frac{1}{\alpha}}
\geq
    \frac{x_1 + x_2 + \ldots + x_n}{n}
\; . \]

% spell Коши--Буняковского--Шварца

\item\emph{Неравенство Коши--Буняковского--Шварца.}
Используя неравенство Йенсена для $y = \frac{1}{x}$, докажите:
\[
    (a_1^2 + \ldots + a_n^2) \cdot (b_1^2 + \ldots + b_n^2)
\geq
    (a_1 b_1 + \ldots + a_n b_n)^2
\; . \]

\item
Для положительных чисел $a$, $b$, $c$ докажите, что
\[
    \frac{a + b + c}{3}
\leq
    \sqrt[a+b+c]{a^a b^b c^c}
\leq
    \frac{a^2 + b^2 + c^2}{a + b + c}
\; . \]

\item
Пусть $\alpha$, $\beta$, $\gamma$~--- углы некоторого треугольника
(возможно, вырожденного).
Какие значения может принимать величина
\\
\subproblem $\sin(\alpha) + \sin(\beta) + \sin(\gamma)$;
\quad
\subproblem $\cos(\alpha) + \cos(\beta) + \cos(\gamma)$?

\item
Сумма положительных чисел $x$, $y$, $z$ равна 1.
Докажите, что
\[
    \left( 1 + \frac{1}{x} \right) \cdot
    \left( 1 + \frac{1}{y} \right) \cdot
    \left( 1 + \frac{1}{z} \right)
\geq
    64
\, . \]

\item
Для положительных чисел $a$, $b$, $c$, $d$ докажите неравенство
\[
    \left(\frac{a+b}{c+d}\right)^{a+b}
\leq
    \left(\frac{a}{c}\right)^a
    \cdot
    \left(\frac{b}{d}\right)^b
\; . \]

\item
Числа $a$, $b$, $c$ являются длинами сторон некоторого треугольника.
Докажите неравенство
\[
    \sqrt{a} \cdot (a + c - b) +
    \sqrt{b} \cdot (a + b - c) +
    \sqrt{c} \cdot (b + c - a)
\leq
    \sqrt{(a^2 + b^2 + c^2) \cdot (a + b + c)}
\, . \]

\item
Докажите, что для $x_1, x_2, \ldots, x_n \geq 1$ выполняется неравенство
\[
    \frac{1}{1 + x_1} + \frac{1}{1 + x_2}
    + \ldots +
    \frac{1}{1 + x_n}
\geq
    \frac{n}{1 + \sqrt[n]{x_1 x_2 \ldots x_n}}
\; . \]

%\item
%Пусть $a, b, c, d > 0$ и $a b c d = 1$.
%Докажите, что
%\[
%    \frac{1}{(1 + a)^2} + \frac{1}{(1 + b)^2} +
%    \frac{1}{(1 + c)^2} + \frac{1}{(1 + d)^2}
%\geq
%    1
%\, . \]

\end{problems}

\endgroup % \def\piconst

