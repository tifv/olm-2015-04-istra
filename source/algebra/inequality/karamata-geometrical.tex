% $date: 2015-04-06
% $timetable:
%   g10r1:
%     2015-04-06:
%       3:

\section*{Карамата решает задачи из Прасолова}

% $authors:
% - Владимир Трушков

В~треугольнике
соответствующие стороны равны $a$, $b$, $c$,
углы (в~радианах) равны $\alpha$, $\beta$, $\gamma$,
высоты~— $h_a$, $h_b$, $h_c$,
радиусы вневписанных окружностей~— $r_a$, $r_b$, $r_c$,
а~радиус вписанной окружности~— $r$.

Докажите неравенства:

\begin{problems}

\item
\(\displaystyle
    \cos(\alpha) + \cos(\beta) + \cos(\gamma)
>
    1
\)\,.

\item
\(\displaystyle
    1
<
    \sin \left( \frac{\alpha\mathstrut}{2} \right) +
    \sin \left( \frac{\beta\mathstrut}{2}  \right) +
    \sin \left( \frac{\gamma\mathstrut}{2} \right)
\leq
    \frac{3}{2}
\)\;.

\item
\(\displaystyle
    \ctg(\alpha) + \ctg(\beta) + \ctg(\gamma)
\geq
    \sqrt{3}
\)\,.

\item
Для остроугольного треугольника:\enspace
\(\displaystyle
    \tg(\alpha) + \tg(\beta) + \tg(\gamma)
\geq
    3\sqrt{3}
\)\,.

\item
\(\displaystyle
    \cos \left( \frac{\alpha\mathstrut}{2} \right) \cdot
    \cos \left( \frac{\beta\mathstrut}{2} \right) \cdot
    \cos \left( \frac{\gamma\mathstrut}{2} \right)
\leq
    \frac{3\sqrt{3}}{8}
\)\;.

\item
\(\displaystyle
    \cos^2(\alpha) + \cos^2(\beta) + \cos^2(\gamma)
\geq
    \frac{3}{4}
\)\;.

\item
Для тупоугольного треугольника:\enspace
\(\displaystyle
    \cos^2(\alpha) + \cos^2(\beta) + \cos^2(\gamma)
>
    1
\)\;.

\item
Для остроугольного треугольника:\\[0.5ex]
\(
    \sin(2 \alpha) + \sin(2 \beta) + \sin(2 \gamma)
\leq
    \sin(\alpha + \beta) + \sin(\beta + \gamma) + \sin(\gamma + \alpha)
\)\,.

\item
\(\displaystyle
    \cos(\alpha) \cdot \cos(\beta) +
    \cos(\beta) \cdot \cos(\gamma) +
    \cos(\gamma) \cdot \cos(\alpha)
\leq
    \frac{3}{4}
\)\;.

\item
\(\displaystyle
    h_a + h_b + h_c \geq 9 r
\).

\item
\(\displaystyle
    \frac{r_a}{h_a} + \frac{r_b}{h_b} + \frac{r_c}{h_c}
\geq
    3
\)\,.

\item
\(\displaystyle
    \frac{\mathrm{\pi}}{3}
\leq
    \frac{a \cdot \alpha + b \cdot \beta + c \cdot \gamma}{a + b + c}
<
    \frac{\mathrm{\pi}}{2}
\)\;.

\end{problems}

