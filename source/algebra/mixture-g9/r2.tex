% $date: 2015-04-05
% $timetable:
%   g9r2:
%     2015-04-05:
%       1:

\section*{Разнобой по алгебре и теории чисел}

% $authors:
% - Владимир Брагин
% - Иван Митрофанов

\begin{problems}

\item
Угол, образованный лучами $y = x$ и~$y = 2 x$ при $x \geq 0$, высекает
на~параболе $y = x^2 + px + q$ две дуги.
Эти дуги спроектированы на~ось~$Ox$.
Докажите, что проекция левой дуги на~$1$ короче проекции правой.
\\\emph{(Заключительный этап-1998, 9.1)}

\item
Существуют~ли такие попарно различные натуральные числа $m$, $n$, $p$, $q$, что
$m + n = p + q$ и~$\sqrt{m} + \sqrt[3]{n} = \sqrt{p} + \sqrt[3]{q} > 1000$?
\\\emph{(Заключительный этап-2004, 9.5)}

\item
Десять попарно различных ненулевых чисел таковы, что для любых двух из~них либо
сумма этих чисел, либо их~произведение~--- рациональное число.
Докажите, что квадраты всех чисел рациональны.
\\\emph{(Заключительный этап-2005, 9.5)}

\item
Многочлен $x^3 + a x^2 + b x + c$ таков, что $-2 \leq a + b + c \leq 0$.
Известно, что у~него $3$ действительных корня.
Докажите, что хотя~бы один из~корней принадлежит отрезку $[0; 2]$.
\\\emph{(Заключительный этап-2008, 9.2)}

\item
Два многочлена $P(x) = x^4 + a x^3 + b x^2 + c x + d$ и~$Q(x) = x^2 + p x + q$
принимают отрицательные значения на~некотором интервале~$I$ длины более $2$,
а~вне интервала~--- неотрицательны.
Докажите, что найдется такое $x_0$, что $P(x_0) < Q(x_0)$.
\\\emph{(Заключительный этап-2001, 9.2)}

\item
Докажите, что найдутся такие целые числа $a$, $b$, $c$, $d$, модули которых
больше миллиона, что
\[
    \frac{1}{a} + \frac{1}{b} + \frac{1}{c} + \frac{1}{d}
=
    \frac{1}{a b c d}
\;.\]
\emph{(Заключительный этап-2006, 9.2)}

\item
Петя и~Вася придумали десять квадратных трехчленов.
После чего Вася по~очереди называл последовательные натуральные числа
(начиная с~некоторого), а~Петя каждое названное число подставлял в~один
из~трехчленов по~своему выбору и~записывал полученные значения на~доску слева
направо.
Оказалось, что числа, записанные на~доске, образуют арифметическую прогрессию
(именно в~этом порядке).
Какое максимальное количество чисел Вася мог назвать?
\\\emph{(Заключительный этап-2013, 9.6)}

\item
Обозначим за~$S(x)$ сумму цифр числа~$x$.
Найдутся~ли три таких натуральных числа $a$, $b$ и~$c$, что
$S(a + b) < 5$, $S(a + c) < 5$ и~$S(b + c) < 5$, но~$S(a + b + c) > 50$?
\\\emph{(Заключительный этап-1998, 9.3)}

\item
Сумма чисел $a_1$, $a_2$, $a_3$, каждое из~которых больше единицы, равна $S$,
причем $\frac{a_i^2}{a_i - 1} > S$ для каждого $i = 1, 2, 3$.
Докажите, что
\[
    \frac{1}{a_1 + a_2} + \frac{1}{a_2 + a_3} + \frac{1}{a_1 + a_3}
>
    1
\,.\]
\emph{(Заключительный этап-2005, 9.3)}

\end{problems}

