% $date: 2015-04-02
% $timetable:
%   g9r1:
%     2015-04-02:
%       1:

\section*{Разнобой по алгебре и теории чисел}

% $authors:
% - Владимир Брагин
% - Иван Митрофанов

\begin{problems}

\item
Существуют~ли такие попарно различные натуральные числа $m$, $n$, $p$, $q$, что
$m + n = p + q$ и~$\sqrt{m} + \sqrt[3]{n} = \sqrt{p} + \sqrt[3]{q} > 1000$?
\\\emph{(Заключительный этап-2004, 9.5)}

\item
Десять попарно различных ненулевых чисел таковы, что для любых двух из~них либо
сумма этих чисел, либо их~произведение~— рациональное число.
Докажите, что квадраты всех чисел рациональны.
\\\emph{(Заключительный этап-2005, 9.5)}

\item
Докажите, что найдутся такие целые числа $a$, $b$, $c$, $d$, модули которых
больше миллиона, что
\[
    \frac{1}{a} + \frac{1}{b} + \frac{1}{c} + \frac{1}{d}
=
    \frac{1}{a b c d}
\; . \]
\emph{(Заключительный этап-2006, 9.2)}

\item
Найдите все числа Фибоначчи, являющиеся степенями двойки или тройки.

\item
Петя и~Вася придумали десять квадратных трехчленов.
После чего Вася по~очереди называл последовательные натуральные числа
(начиная с~некоторого), а~Петя каждое названное число подставлял в~один
из~трехчленов по~своему выбору и~записывал полученные значения на~доску слева
направо.
Оказалось, что числа, записанные на~доске, образуют арифметическую прогрессию
(именно в~этом порядке).
Какое максимальное количество чисел Вася мог назвать?
\\\emph{(Заключительный этап-2013, 9.6)}

\item
Обозначим за~$S(x)$ сумму цифр числа~$x$.
Найдутся~ли три таких натуральных числа $a$, $b$ и~$c$, что
$S(a + b) < 5$, $S(a + c) < 5$ и~$S(b + c) < 5$, но~$S(a + b + c) > 50$?
\\\emph{(Заключительный этап-1998, 9.3)}

\item
Сумма чисел $a_1$, $a_2$, $a_3$, каждое из~которых больше единицы, равна $S$,
причем $\frac{a_i^2}{a_i - 1} > S$ для каждого $i = 1, 2, 3$.
Докажите, что
\[
    \frac{1}{a_1 + a_2} + \frac{1}{a_2 + a_3} + \frac{1}{a_1 + a_3}
>
    1
\, . \]
\emph{(Заключительный этап-2005, 9.3)}

\end{problems}

