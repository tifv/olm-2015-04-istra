% $date: 2015-03-31
% $timetable:
%   g9r1:
%     2015-03-31:
%       3:

\section*{Прыжки по кругу}

% $authors:
% - Владимир Брагин
% - Иван Митрофанов

\begin{problems}

\item
Кузнечик прыгает по~окружности длины~$1$.
За~каждую секунду он~прыгает по~часовой стрелке на~дугу длины~$\alpha$.
\\
\sp
При каких $\alpha$ он~сможет побывать лишь в~конечном числе точек окружности?
\\
\sp
Докажите, что не~позже чем через $1000$ секунд он~окажется на~расстоянии меньше
чем $1 / 1000$ от~своего исходного положения (расстояние считается
по~окружности).
\\
\sp
При каких $\alpha$ кузнечик сможет рано или поздно посетить любую дугу
окружности?

\item
Существует~ли такое натуральное $n$, что $\tg(n) > 10000000$?

\item
Кузнечик прошел курсы повышения квалификации и~теперь он~умеет делать два
прыжка: с~длинами $\sqrt{2}$ и~$\sqrt{3}$.
Теперь кузнечик готов прыгать по~прямой.
Докажите, что он~сможет попасть в~любой отрезок на~прямой.

\item
Дано иррациональное число~$\alpha$.
\\
\sp
Докажите, что для любого натурального~$N$ существует такое ненулевое
$-N < y \leq N$ такое, что $\{ y \alpha \} < 1 / N$.
\\
\sp\claim{Теорема Дирихле}
Докажите, что существует бесконечно много пар целых $(x, y)$ таких, что
\[
    |x - y \alpha| < \frac{1}{y}
\,.\]

\end{problems}

Вновь вернемся к~целым числам.

\begin{problems}

\item
Дано простое число $p$.
\\
\sp
Обозначим $k = [\sqrt{p}]$.
Назовем вычеты $-k, -k+1, \ldots 0, 1, 2, \ldots k$ по~модулю~$p$
\emph{маленькими}.
Докажите, что для любого вычета~$a$ существует такой ненулевой маленький
вычет~$b$ такой, что вычет~$a b$ тоже маленький.
\\
\sp
Пусть $p \mid a^2 + 1$.
Докажите, что $p$ представимо в~виде суммы двух квадратов.
\\
\sp
Вспомните, что $a^2 + 1$ бывает кратно $p$ тогда и~только тогда, когда $p + 1$
не~делится на~4.
\\
\sp
\emph{Рождественская теорема Ферма или теорема Ферма-Эйлера о~двух квадратах}
Докажите, что число представимо в~виде суммы двух квадратов тогда и~только
тогда, когда каждое простое вида $4 k + 3$ входит в~его разложение в~четной
степени.

\end{problems}

\subsection*{Для самостоятельного решения}

\begin{problems}

\item
Докажите, что функция $\sin(x) + \sin(\sqrt{2} x)$ непериодична.

\item
Докажите, что существует степень двойки, которая начинается с~числа $300$.

\item
Из~начала координат выпустили луч.
Докажите, что найдется ненулевая точка с~целыми координатами, расстояние
от~которой до~луча меньше $0{,}001$.

\item
Когда число представимо в~виде $m^2 + 2 n^2$?

\end{problems}

