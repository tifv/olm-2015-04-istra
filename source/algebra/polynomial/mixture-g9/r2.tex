% $date: 2015-04-07
% $timetable:
%   g9r2:
%     2015-04-07:
%       1:

\section*{Многочлены}

% $authors:
% - Владимир Брагин
% - Иван Митрофанов

\begin{problems}

\item
Докажите, что график любого кубического четырехчлена имеет центр симметрии.

\item
При каких $a$ и~$b$ корни многочлена $x^3 + a x + b$ образуют арифметическую
прогрессию?

\item
Вася нарисовал на~листе бумаги график многочлена $3$-й степени, а~потом выделил
полосу, ограниченную двумя прямыми, параллельными $Ox$.
В~полосе оказались три куска графика.
Докажите, что проекция одного из~этих кусков на~ось $Ox$ равна сумме проекций
двух других кусков.

\item
Длины сторон треугольника являются корнями кубического уравнения
с рациональными коэффициентами.
Докажите, что длины высот треугольника являются корнями уравнения шестой
степени с рациональными коэффициентами.

\item
Кубическое и~квадратное уравнения с~рациональными коэффициентами имеют общее
решение.
Докажите, что у~кубического уравнения есть рациональный корень.

\end{problems}

\subsection*{Многочлены и теория чисел}

\begin{problems}

\item
Существует~ли такой многочлен $P(x)$ с~целыми коэффициентами, что
$P(13) = 5$ и~$P(31) = 6$?

\item
Многочлен $P(x)$ с~целыми коэффициентами таков, что
\[
    P(2000) = 2001
,\;
    P(2015) = 2015
.\]
Докажите, что $P(n)$ нечетно при целых~$n$.

\item
Дан многочлен $P(x)$ с~целыми коэффициентами.
Оказалось, что $P(1) = 2015$,  $P(2015) = 1$, а~$P(k) = k$.
Найдите $k$.

\item
Докажите, что если многочлен с~целыми коэффициентами при трех различных целых
значениях переменной принимает значение~1, то~он не~имеет ни~одного целого
корня.

\item
Докажите, что не~существует многочлена (степени больше нуля) с~целыми
коэффициентами, принимающего при каждом натуральном значении аргумента
значение, равное некоторому простому числу.

\item
\subproblem
Докажите, что наибольший общий делитель чисел $(n - 1)$
и~$(n^4 + n^3 + n^2 + n + 1)$ равен 1 или 5.
\\
\subproblem
Два многочлена $P(x)$ и~$Q(x)$ с~целыми коэффициентами взаимно просты как
многочлены.
Докажите, что существует такое натуральное~$M$, что наибольший общий делитель
чисел $P(n)$ и~$Q(n)$ делит $M$.

\end{problems}

\subsection*{Внезапный многочлен}

\begin{problems}

\item
Гриша записал на~доске $100$ чисел.
Затем он~увеличил каждое число на~$1$ и~заметил, что произведение всех
$100$ чисел не~изменилось.
Он~опять увеличил каждое число на~1, и~снова произведение всех чисел
не~изменилось, и~так далее.
Всего Гриша повторил эту процедуру $k$ раз, и~все $k$ раз произведение чисел
не~менялось.
Найдите наибольшее возможное значение $k$.

\end{problems}

