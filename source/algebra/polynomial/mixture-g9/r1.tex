% $date: 2015-04-04
% $timetable:
%   g9r1:
%     2015-04-04:
%       1:

\section*{Многочлены}

% $authors:
% - Владимир Брагин
% - Иван Митрофанов

\begin{problems}

\item
Докажите, что график любого кубичествого четырехчлена имеет центр симметрии.

\item
\subproblem
Угол, образованный лучами $y = x$ и~$y = 2 x$ при $x \geq 0$, высекает
на~параболе $y = x^2 + p x + q$ две дуги.
Эти дуги спроектированы на~ось $Ox$.
Докажите, что проекция левой дуги на~1 короче проекции правой.
\\
\subproblem
Вася нарисовал на~листе бумаги график многочлена $3$-й степени, а~потом выделил
полосу, ограниченную двумя прямыми, параллельными $Ox$.
В~полосе оказались три куска графика.
Докажите, что проекция одного из~этих кусков на~ось $Ox$ равна сумме проекций
двух других кусков.

\item
Многочлен $x^3 + a x^2 + b x + c$ таков, что $-2 \leq a + b + c \leq 0$.
Известно, что у~него три действительных корня.
Докажите, что хотя~бы один из~корней принадлежит отрезку $[0; 2]$.

\item
Два многочлена $P(x) = x^4 + a x^3 + b x^2 + c x + d$ и~$Q(x) = x^2 + p x + q$
принимают отрицательные значения на~некотором интервале $I$ длины более $2$,
а~вне интервала~— неотрицательны.
Докажите, что найдется такое $x_0$, что $P(x_0) < Q(x_0)$.

\end{problems}

\subsection*{Многочлены и теория чисел}

\begin{problems}

\item
Кубическое и~квадратное уравнения с~рациональными коэффициентами имеют общее
решение.
Докажите, что у~кубического уравнения есть рациональный корень.

\item
Существует~ли такой многочлен $P(x)$ с~целыми коэффициентами, что
$P(13) = 5$ и~$P(31) = 6$?

\item
Докажите, что если многочлен с~целыми коэффициентами при трех различных целых
значениях переменной принимает значение~1, то~он не~имеет ни~одного целого
корня.

\item
Докажите, что не~существует многочлена (степени больше нуля) с~целыми
коэффициентами, принимающего при каждом натуральном значении аргумента
значение, равное некоторому простому числу.

\item
\label{/algebra/polynomial/mixture-g9/r1:problem:gauss-lemma}%
\subproblem
Пусть $P$, $Q$, $R$~— многочлены с~целыми коэффициентами, причем
$P = Q \cdot R$.
Все коэффициенты $P$ делятся на~простое число $p$.
Докажите, что у~$Q$ или у~$S$ все коэффициенты также делятся на~$p$.
\\
\subproblem
Пусть $P = Q \cdot R$, все коэффициенты $P$ целые, а~все коэффициенты $Q$ и~$S$
рациональные.
Докажите, что найдется такое рациональное число $r$, что коэффициенты
многочленов $Q \cdot r$ и~$R / r$ целые.

\end{problems}

\subsection*{Внезапный многочлен}

\begin{problems}

\item
Гриша записал на~доске $100$ чисел.
Затем он~увеличил каждое число на~$1$ и~заметил, что произведение всех
$100$ чисел не~изменилось.
Он~опять увеличил каждое число на~1, и~снова произведение всех чисел
не~изменилось, и~так далее.
Всего Гриша повторил эту процедуру $k$ раз, и~все $k$ раз произведение чисел
не~менялось.
Найдите наибольшее возможное значение $k$.

\item
Докажите, что при $n > 6$ число $\cos(2 \mathrm{\pi} / n)$ иррационально.
\\\emph{Подсказка: воспользуйтесь комплексными числами
и~задачей~\ref{/algebra/polynomial/mixture-g9/r1:problem:gauss-lemma}.}

\end{problems}

