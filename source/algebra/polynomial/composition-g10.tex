% $date: 2015-04-01
% $timetable:
%   g10r2:
%     2015-04-05:
%       1:
%   g10r1:
%     2015-04-01:
%       1:

\section*{Суперпозиция многочленов}

% $authors:
% - Владимир Трушков

\begin{problems}

\item
Пусть $P(x)$~--- многочлен нечетной степени.
Докажите, что уравнение $P(P(x)) = 0$ имеет не~меньше различных действительных
корней, чем уравнение $P(x) = 0$.

\item
Приведенные квадратные трехчлены $f(x)$ и~$g(x)$ таковы, что уравнения
$f(g(x)) = 0$ и~$g(f(x)) = 0$ не~имеют вещественных корней.
Докажите, что хотя~бы одно из~уравнений $f(f(x)) = 0$ и~$g(g(x)) = 0$ также
не~имеет вещественных корней.

\item
Приведенный квадратный трехчлен $f(x)$ имеет 2 различных корня.
Может~ли так оказаться, что уравнение $f(f(x)) = 0$ имеет 3 различных корня,
а~уравнение $f(f(f(x))) = 0$ имеет 7 различных корней?

\item
Квадратные трехчлены $P(x) = x^2 + a x + b$ и~$Q(x) = x^2 + c x + d$ таковы,
что уравнение $P(Q(x)) = Q(P(x))$ не~имеет действительных корней.
Докажите, что $b \neq d$.

\item
Многочлен $P(x) = x^3 + a x^2 + b x + c$ имеет три разлиных действительных
корня, а~многочлен $P(Q(x))$, где $Q(x) = x^2 + x + 2001$ действительных корней
не~имеет.
Докажите, что $P(2001) > 1 / 64$.

\item
Дан квадратный трехчлен $f(x) = x^2 + a x + b$.
Уравнение $f(f(x)) = 0$ имеет четыре различных действительных корня, сумма двух
из~которых равна $-1$.
Докажите, что $b \leq -\frac{1}{4}$.

\item
Известно, что $f(x)$, $g(x)$ и~$h(x)$~--- квадратные трехчлены.
Может~ли уравнение $f(g(h(x))) = 0$ иметь корни 1, 2, 3, 4, 5, 6, 7 и~8?

\item
Пусть $f(x) = a x^2 + b x + c$.
Докажите, что $f(f(x)) = x$ не~может иметь ровно три действительных корня.

\end{problems}

