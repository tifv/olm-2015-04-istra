% $date: 2015-03-30
% $timetable:
%   g10r2:
%     2015-03-30:
%       3:

\section*{Алгебраический разнобой}

% $authors:
% - Владимир Трушков

\begin{problems}

\item
Все коэффициенты квадратного трехчлена~--- целые нечетные числа.
Может~ли он~иметь целые корни?

\item
Квадратный трехчлен $x^2 + a x + b$ имеет целые корни, по~модулю большие~2.
Докажите, что $\lvert a + b + 1 \rvert$~--- составное число.

\item
Докажите, что многочлен $x^p + x^{p-1} + \ldots + x + p = 0$, где $p$~---
простое число, не~имеет рациональных корней.

\item
Найдите свободный член многочлена $P(x)$ с~целыми коэффициентами, если
известно, что он~по~модулю меньше тысячи, и~$P(19) = P(94) = 1994$.

\item
Приведенный квадратный трехчлен с~целыми коэффициентами в~трех
последовательных целых точках принимает простые значения.
Докажите, что он~принимает простое значение по~крайней мере еще в~одной
целой точке.

\item
Докажите, что
\[
    x^{p-1} - 1
\equiv
    (x - 1) \cdot (x - 2) \cdot \ldots \cdot (x - (p - 1))
\pmod{p}
\]
(т.~е. коэффициенты при каждой из~степеней будут совпадать по~модулю $p$.)

\item
Уравнение $x^n + a_1 x^{n-1} + a_2 x^{n-2} + \ldots + a_{n-1} x + a_n = 0$
с~целыми ненулевыми коэффициентами $a_1, a_2, \ldots, a_n$ имеет $n$ различных
целых корней.
Докажите, что если любые два корня взаимно просты, то~и~числа $a_{n-1}$ и~$a_n$
взаимно просты.

\item
Существуют~ли такие приведенные квадратные трехчлены $f$ и~$g$, что для любого
целого $n$ число $f(n) \cdot g(n)$~--- целое, а~числа $f(n)$, $g(n)$
и~$f(n) + g(n)$~--- нецелые?

\item
Назовем многочлен \emph{средиземноморским}, если он~имеет только действительные
корни и~имеет вид
\begin{align*}
    P(x)
={}&
    x^{10} - 20 x^9 + 135 x^8
    +\\&{}+
    a_7 x^7 + a_6 x^6 + a_5 x^5 + a_4 x^4 + a_3 x^3 + a_2 x^2 + a_1 x + a_0
\, . \end{align*}
Коэффициенты $a_0, \ldots, a_7$~--- действительные числа.
Найдите наибольшее действительное
число, которое может быть корнем средиземноморского многочлена.

\end{problems}

