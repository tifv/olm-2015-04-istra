% $date: 2015-04-07
% $timetable:
%   g11r2:
%     2015-04-07:
%       2:

\section*{Многочлены}

% $authors:
% - Антон Гусев

\begin{problems}

\item
Найдите все $a$, при которых многочлены $x^4 + a x^2 + 1$ и~$x^3 + a x + 1$
имеют общий корень.

\item
В~выражении $(x^4 + x^3 - 3 x^2 + x + 2)^{2014}$ раскрыли скобки и~привели
подобные слагаемые.
Докажите, что при некоторой степени переменной~$x$ получился отрицательный
коэффициент.

\item
Исходно на~доске написано $(x^3 - 3 x^2 + 5)$ и~$(x^2 - 4 x)$.
Если на~доске уже написаны многочлены $f(x)$ и~$g(x)$, то~разрешается дописать
любые из~многочленов
\[
    f(x) + g(x)
,\enspace
    f(x) - g(x)
,\enspace
    f(x) \cdot g(x)
,\enspace
    f(g(x))
,\enspace
    c \cdot f(x)
\, , \]
где $c$~— произвольная (не~обязательно целая) константа.
Может~ли на~доске после нескольких таких операций появиться многочлен вида
$(x^n - 1)$?

\item
Докажите, что не~существует многочлена от~двух переменных $P(x, y)$ такого, что
$P(x, y) > 0$ тогда и~только тогда, когда $x > 0$ и~$y > 0$.

\item
Пусть для некоторых многочленов с~действительными коэффициентами
$P(x)$ и~$Q(x)$ выполнено равенство
\[
    x^{2014}
=
    (x^3 - 6 x^2 + 11 x - 6) \cdot P(x) + Q(x)
\, , \]
где степень $Q(x)$ не~превосходит $2$.
Докажите, что все коэффициенты $P(x)$ положительны.

\item
Существует~ли такой многочлен $P(x)$, что среди его коэффициентов есть
отрицательные, а~у~$(P(x))^n$ все коэффициенты положительны для любого
натурального $n > 1$?

\item
Докажите, что если многочлен $P(x)$ с~действительными коэффициентами принимает
при всех действительных~$x$ неотрицательные значения, то~он~представим в~виде
$P(x) = Q_{1}^2(x) + \ldots + Q_{m}^2(x)$, где $Q_{1}(x), \ldots, Q_{m}(x)$~—
многочлены с~действительными коэффициентами.

\item
Многочлен $P(x)$ удовлетворяет следующим свойствам:
$P(0) = 1$, $P^2(x) = 1 + x + x^{100} \cdot Q(x)$, где $Q(x)$~— некий
многочлен.
Докажите, что коэффициент при $x^{99}$ в~многочлене $(P(x) + 1)^{100}$ равен
нулю.

\end{problems}

