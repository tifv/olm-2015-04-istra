% $date: 2015-04-01
% $timetable:
%   g10r2:
%     2015-04-01:
%       2:

\section*{Разные задачи про квадратные уравнения}

% $authors:
% - Владимир Трушков

\begin{problems}

\item
Рассматриваются квадратичные функции вида $y = x^2 + p x + q$, у~которых
$p + q / 2 = 2001$.
Докажите, что их~графики проходят через одну точку.

\item
Пусть $a$ и~$b$~--- положительные числа.
Сумма минимального значения квадратного трехчлена $f(x) = a x^2 + 8 x + b$
и~минимального значения квадратного трехчлена $g(x) = b x^2 + 8 x + a$ равна
нулю.
Докажите, что эти минимальные значения оба равны нулю.

\item
Приведенные квадратные трехчлены $f(x)$ и~$g(x)$ принимают отрицательные
значения на~непересекающихся интервалах.
Докажите, что найдутся такие положительные числа $\alpha$ и~$\beta$, что для
любого действительного $x$ будет выполняться неравенство
$\alpha \cdot f(x) + \beta \cdot g(x) > 0$.

\item
Даны три квадратных трехчлена с~попарно различными старшими коэффициентами.
Графики любых двух из~них имеют ровно одну общую точку.
Докажите, что все три графика имеют ровно одну общую точку.

\item
Квадратный трехчлен $f(x)$ разрешается заменить на~один из~трехчленов
\[
    x^2 \cdot f\left(\frac{1}{x} + 1 \right)
\text{\quad или\quad}
    (x - 1)^2 \cdot f\left(\frac{1}{x - 1} \right)
\]
Можно~ли с~помощью таких операций из~трехчлена $x^2 + 4 x + 3$ получить
трехчлен $x^2 + 10 x + 9$? 

\item
Различные числа $a$, $b$, $c$ таковы, что уравнения $x^2 + a x + 1 = 0$
и~$x^2 + b x + c = 0$ имеют общий действительный корень.
Кроме того, общий действительный корень имеют уравнения $x^2 + x + a = 0$
и~$x^2 + c x + b = 0$.
Найдите $a + b + c$.

\item
Дан квадратный трехчлен $f(x) = x^2 + a x + b$.
Известно, что для любого вещественного $x$ существует вещественное $y$ такое,
что $f(y) = f(x) + y$.
Найдите наибольшее возможное значение~$a$.

\item
Квадратный трехчлен $p(x)$ таков, что $|p(x)| \leq 1$ при $0 \leq x \leq 1$.
Докажите, что $p(- 1 / 2) \leq 7$.

\end{problems}

