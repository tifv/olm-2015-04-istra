% $date: 2015-04-07
% $timetable:
%   g11r1:
%     2015-04-07:
%       3:

\section*{Многочлены}

% $authors:
% - Антон Гусев

\begin{problems}

\item
В~выражении $(x^4 + x^3 - 3 x^2 + x + 2)^{2014}$ раскрыли скобки и~привели
подобные слагаемые.
Докажите, что при некоторой степени переменной $x$ получился отрицательный
коэффициент.

\item
Существует~ли многочлен $P(x)$ степени~$n$ с~целыми коэффициентами, для
которого можно подобрать константу~$C$ такую, что многочлен
$P(x) \cdot \bigl( P(x) + C \bigr)$ имеет $2 n$ различных корней?
\\
Разберите случаи
\quad
\subproblem $n = 4$;
\quad
\subproblem $n = 3$;
\quad
\subproblem $n = 5$.
    
\item
При каких $n$ многочлен $x^n + 1$ неприводим?

\item
При каких $n$ многочлен $x^n + 64$ неприводим?

\item
Пусть для некоторых многочленов с~действительными коэффициентами
$P(x)$ и~$Q(x)$ выполнено равенство
\[
    x^{2014}
=
    (x^3 - 6 x^2 + 11 x - 6) \cdot P(x) + Q(x)
\,,\]
где степень $Q(x)$ не~превосходит $2$.
Докажите, что все коэффициенты $P(x)$ положительны.

\item
Исходно на~доске написано $(x^3 - 3 x^2 + 5)$ и~$(x^2 - 4 x)$.
Если на~доске уже написаны многочлены $f(x)$ и~$g(x)$, то~разрешается дописать
любые из~многочленов
\[
    f(x) + g(x)
,\enspace
    f(x) - g(x)
,\enspace
    f(x) \cdot g(x)
,\enspace
    f(g(x))
,\enspace
    c \cdot f(x)
\,,\]
где $c$~--- произвольная (не~обязательно целая) константа.
Может~ли на~доске после нескольких таких операций появиться многочлен вида
$(x^n - 1)$? 

\item
$P$, $Q$~--- многочлены с~целыми коэффициентами, $k \in \mathbb{N}$.
Число $\bigl( P(Q(x)) - x \bigr)$ делится на~$k$ при всех целых~$x$.
Докажите, что $\bigl( Q(P(x)) - x \bigr)$ тоже делится на~$k$ при всех
целых~$x$.

\item
$P$, $Q$~--- взаимно простые многочлены с~целыми коэффициентами.
Докажите, что существует натуральное $C$ такое, что для любого
целого $n$ НОД чисел $P(n)$ и~$Q(n)$ не~превосходит $c$.
    
\item
Для каких $n$ существует многочлен степени $n$ с~целыми коэффициентами,
в~точках $0, \ldots, n$ принимающий значения, являющиеся степенями двойки?

\item
\emph{Допустимая замена} меняет каждый из~коэффициентов многочлена не~более чем
на~2009.
Докажите, что для произвольного $n \in \mathbb{N}$ существует многочлен
с~целыми коэффициентами $n$-й степени, который неприводим и~останется
неприводим после любой допустимой замены.

\item
Найдите все пары $(m, n)$, где $m \geq 3$ и~$n \geq 3$, такие, что
многочлен $(x^m + x - 1)$ делится на~$(x^n + x^2 - 1)$.
    
\item
Пусть $P(x)$~--- неприводимый многочлен степени~$n$ с~целыми коэффициентами,
$Q(x)$, $R(x)$~--- многочлены с~целыми коэффициентами.
Известно, что $P(Q(x))$ делится на~$R(x)$.
Докажите, что $\deg R(x) \geq n$.

\end{problems}

