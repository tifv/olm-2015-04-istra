% $date: 2015-04-07
% $timetable:
%   g10r2:
%     2015-04-07:
%       2:

\section*{Производные}

% $authors:
% - Владимир Трушков

\begin{problems}

\item
Докажите, что уравнение $4 x^3 - 3 b x^2 + 2 c x - d = 0$ имеет 3
действительных корня тогда и~только тогда, когда существует действительные
числа $m$, $n$, $p$, $q$ такие, что
\[ \left\{ \begin{aligned} &
    b = m + n + p + q
, \\ &
    c = m n + m p + m q + n p + n q + p q
, \\ &
    d = m n p + m n q + m p q + n p q
.\end{aligned} \right. \]

\item
Многочлен четвертой степени $P(x)$ имеет четыре корня, попарные расстояния
между которыми не~меньше~$1$.
Докажите, что найдутся два корня $P'(x)$, находящиеся на~расстояние
не~меньше~$1$.

\item
Найдите все действительные $a$ и~$b$ такие, что уравнение
$x^3 + a x^2 + b x + c = 0$ имеет не~более двух положительных корней при всех
значениях~$c$.

\item
Сколько существует многочленов вида $x^3 + a x^2 + b x + c$ таких, что
множество их~корней есть $\{ a, b, c \}$?

\item
Докажите, что при целых значениях $c$ уравнение
$x \cdot (x^2 - 1) \cdot (x^2 - 10) = c$ не~может иметь пяти целых корней.

\item
Докажите, что если все корни многочлена с~действительными коэффициентами
$P(x) = a_0 x^n + \ldots + a_n$ действительны, то~и~все его производные имеют
лишь действительные корни.

\item
Пусть $P(x)$~--- многочлен степени $n$
и~$P(a) \geq 0$, $P'(a) \geq 0$, \ldots, $P^{(n-1)}(a) \geq 0$,
$P^{(n)}(a) > 0$.
Докажите, что действительные корни уравнения $P(x) = 0$ не~превосходят $a$.

\item
Докажите, что многочлен
\[
    P(x)
=
    1 + x + \frac{x^2}{2!} + \ldots + \frac{x^n}{n!}
\]
не~имеет кратных корней.

\item
Пусть
\[
    c_0 + \frac{c_1}{2} + \ldots + \frac{c_n}{n+1}
=
    0
\,.\]
Докажите, что многочлен $c_0 + c_1 x + c_2 x^2 + \ldots + c_n x^n$ имеет
хотя~бы один действительный корень.

\end{problems}

