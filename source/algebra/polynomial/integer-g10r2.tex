% $date: 2015-04-02
% $timetable:
%   g10r2:
%     2015-04-02:
%       2:

\section*{Многочлены с целыми коэффициентами}

% $authors:
% - Владимир Трушков

\begin{problems}

\item
Известно, что $f(-1) = f(0) = f(1) = 0$.
Докажите, что $f(n)$ кратно трем для любого целого~$n$.

\item
Многочлен $P(x)$ таков, что $P(7) = 11$, а~$P(11) = 13$.
Докажите, что хотя~бы один из~его коэффициентов~--- не~целое число.

\item
Пусть $f$~--- многочлен с~целыми коэффициентами.
Докажите, что $f(a) - f(b)$ делится на~$a - b$, где $a$ и~$b$~--- различные
целые числа.

\item
Все коэффициенты многочлена $P(x)$~--- целые числа.
Известно, что $P(1) = 1$ и~что $P(n) = 0$ при некотором натуральном~$n$.
Найдите $n$.

\item
Дан многочлен с~целыми коэффициентами.
Если в~него вместо неизвестного подставить 2 или 3, то~получаются числа,
кратные 6.
Докажите, что если вместо неизвестного в~него подставить~5, то~также получится
число, кратное 6.

\item
$P(x)$~--- многочлен с~целыми коэффициентами.
Докажите, что если уравнение $P(x) = 1$ имеет больше трех целочисленных корней,
то~уравнение $P(x) = -1$ не~имеет целочисленных корней.

\item
Многочлен $f(x)$ с~целыми коэффициентами принимает значение $5$ при пяти
различных целых значениях $x$.
Может~ли $f(x)$ иметь целые корни?
Может~ли $f(n)$ равняться $-6$ при целом~$n$?

\item
Многочлен седьмой степени с~целыми коэффициентами в~семи целых точках принимает
значения $\pm 1$.
Докажите, что этот многочлен нельзя разложить в~произведение двух многочленов
с~целыми коэффициентами.

\item
Многочлены $P(x)$ и~$Q(x)$ с~целыми коэффициентами таковы, что число $P(k)$
делится на~$Q(k)$ при любом целом $k$.
Докажите, что многочлен $P(x)$ делится на~$Q(x)$.

\item
Пусть $a_1, a_2, \ldots, a_n$~--- различные целые числа.
Докажите, что многочлен
\[
    (x - a_1) \cdot (x - a_2) \cdot \ldots \cdot (x - a_n) - 1
\]
нельзя разложить в~произведение двух многочленов с~целыми коэффициентами.

\end{problems}

