% $date: 2015-04-06
% $timetable:
%   g10r2:
%     2015-04-06:
%       2:

\section*{Опять многочлены}

% $authors:
% - Владимир Трушков

\begin{problems}

\item
Бесконечная последовательность
$P_0(x), P_1(x), P_2(x), \ldots, P_n(x), \ldots$
определена как
\[
    P_0(x) = x
,\quad
    P_n(x) = P_{n-1}(x - 1) \cdot P_{n-1}(x + 1)
,\;
    n \geq 1
.\]
Найдите наибольшее $k$ такое, что $P_{2014}(x)$ делится на~$x^k$.

\item
Пусть $\lvert a x^2 + b x + c \rvert \leq 1$ при $x \in [-1; 1]$.
Найдите максимум выражения $a^2 + b^2 + c^2$.

\item
Докажите, что при любых отличных от~нуля числах $a$, $b$, $c$ хотя~бы одно
из~квадратных уравнений
$a x^2 + 2 b x + c = 0$, $b x^2 + 2 c x + a = 0$, $c x^2 + 2 a x + b = 0$
имеет корень.

\item
Даны корни $x_0$ и~$x_1$, $x_0$ и~$x_2$, \ldots, $x_0$ и~$x_n$ квадратных
трехчленов
$y = x^2 + a_1 x + b_1$, $y = x^2 + a_2 x + b_2$, \ldots,
$y = x^2 + a_n x + b_n$.
Найдите корни квадратного трехчлена
\[
    y
=
    x^2
    +
    \frac{a_1 + a_2 + \ldots + a_n}{n} \cdot x
    +
    \frac{b_1 + b_2 + \ldots + b_n}{n}
\; . \]

\item
Пусть $P(x)$~--- квадратный трехчлен с~действительными коэффициентами такой,
что $P(x^3 + x) \geq P(x^2 + 1)$ для всех действительных $x$.
Найдите сумму корней $P(x)$.

\item
Даны два квадратных трехчлена $f(x)$ и~$g(x)$.
Известно, что каждое из~выражений $3 f(x) + g(x)$ и~$f(x) - g(x)$~---
квадратные трехчлены, имеющие ровно один корень.
Известно также, что $f(x)$ имеет два корня.
Докажите, что трехчлен $g(x)$ не~имеет корней.

\item
Пусть $f(x) = x^3 + x + 1$, $g(x)$~--- кубический многочлен, корни которого
являются квадратами корней многочлена $f(x)$.
Найдите $g(x)$.

\item
Найдите все многочлены $P(x)$, удовлетворяющие условиям
$P(x^2) = x^2 \cdot (x^2 + 1) \cdot P(x)$, $P(2) = 2$.

\end{problems}

