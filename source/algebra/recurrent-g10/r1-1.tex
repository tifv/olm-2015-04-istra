% $date: 2015-04-05
% $timetable:
%   g10r1:
%     2015-04-05:
%       2:

\section*{Рекуррентные соотношения --- 1}

% $authors:
% - Владимир Трушков

\begin{problems}

\item
Последовательность $a_n$ удовлетворяет при любом натуральном $n$ соотношению
$a_{n+2} = (a_{n+1} + 1) / a_n$.
Найдите $a_{1998}$, если $a_{19} = 19$, $a_{97} = 97$.

\item
Докажите, что если
\[
    u_0 = 0{,}001
,\quad
    u_{n+1} = u_n \cdot (1 - u_n)
        ,\; n \geq 0
\,,\]
то~$u_{1000} < 1 / 2000$.

\item
Последовательность $\{ x_n \}$ удовлетворяет соотношениям
\[
    x_1 \in (0; 1)
,\quad
    x_{n+1} = x_n + \frac{x_n^2}{n (n+1)}
        ,\; n \geq 1
\;.\]
Докажите, что эта последовательность ограничена.

\item
Последовательность $\{ a_n \}$ определена как
\[
    a_1 = a_2 = 1
,\quad
    a_{n+2} = a_{n+1} + \frac{a_n}{3^n}
        ,\; n \geq 1
\,.\]
Докажите, что $a_n < 2$ при $n \geq 1$.

\item
Последовательность натуральных чисел $\{ a_n \}$ определим как
$a_1 = 1$, $a_{n+1} = a_n - 2$, если такого числа $a_{n+1}$ не встречалось
в последовательности, $a_{n+1} = a_n + 3$ в противном случае.
Докажите, что любой квадрат натурального числа впервые появится
в последовательности увеличением на тройку.

\item
Докажите, что если
\[
    a_1 = 1
,\quad
    a_{n+1} = \frac{a_n}{2} + \frac{1}{a_n}
        ,\; n \geq 1
\,,\]
то $0 < a_{10} - \sqrt{2} < 10^{-210}$.

\item
Последовательность задана условиями
\[
    a_1 = 100
,\quad
    a_{n+1} = a_n + \frac{1}{a_n}
        ,\; n \geq 1
.\]
Найдите $[a_{2015}]$.

\item
Последовательность задана условиями
\[
    a_1 = 1
,\quad
    a_{n+1} = a_n + \frac{1}{a_n^2}
        ,\; n \geq 1
\,.\]
Докажите, что $a_{2015} > 18$.

\item
Последовательность $\{ x_n \}$ определена начальным условием $x_1 = 1 / 2$
и соотношением $x_{n+1} = 1 - x_1 \cdot x_2 \cdot \ldots \cdot x_n$.
Докажите, что 
\\
\sp $x_{100} > 0{,}99$;
\qquad
\sp $x_{100} < 0{,}991$.

\item
Последовательность $\{ x_k \}$ определена как
\[
    x_1 = 1
,\quad
    x_{2k} = -x_k
    ,\;
    x_{2k-1} = (-1)^{k+1} x_k
        \; \text{ для натуральных~$k$.}
\]
Докажите, что $x_1 + \ldots + x_n \geq 0$ для всех натуральных~$n$.

\end{problems}

