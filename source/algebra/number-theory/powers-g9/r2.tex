% $date: 2015-04-01
% $timetable:
%   g9r2:
%     2015-04-02:
%       3:

% $caption: Большие степени (теория чисел)

\section*{Большие степени}

% $authors:
% - Владимир Брагин
% - Иван Митрофанов

\begingroup
    \def\divides{\mathrel{\vert}}

\begin{problems}

\item
Пусть $n = 561$.
Докажите, что для любого натурального $a$ выполнено $a^n \equiv a \pmod{n}$.

\item
Докажите, что если натуральные числа $n$ и~$m$ взаимно просты, то~$(2^n - 1)$
и~$(2^m - 1)$ также взаимно просты.

\item
Пусть $p$~--- нечетное простое число и~$p \divides (a - 1)$.
Докажите, что $p^2 \divides (a^n - 1)$ тогда и~только тогда, когда $p \divides n$.

\item
Докажите, что существует бесконечно много таких натуральных $n$ таких, что
$n \divides 2^n + 1$.

\item
Даны натуральные числа $m$ и~$n$.
Докажите, что $(2^n - 1)$ делится на~$(2^m - 1)^2$ тогда и~только тогда, когда
$n$ делится на~$m \cdot (2^m - 1)$.

\item
Найдите все числа, взаимно простые с~числом $2^n + 3^n + 6^n - 1$ для
любого $n$.

\item
Решите уравнение $3^k + 4^l = 5^n$ в~натуральных числах.

\item
Найдите все такие пары $(x, y)$ натуральных чисел, что $x + y = a^n$,
а~$x^2 + y^2 = a^m$ для некоторых натуральных $a$, $m$, $n$.

\item
Пусть натуральные числа $x$, $y$, $p$, $n$, $k$ таковы, что $x^n + y^n = p^k$.
Докажите, что если $n > 1$ нечетное, а~$p$ нечетное простое, то~$n$ является
степенью числа~$p$.

\item
Докажите, что ни~для какого простого $p$ и~$n > 1$ уравнение $2^p + 3^p = a^n$
не~имеет натуральных решений.

\item
Решите уравнение в~натуральных числах: $3^k = x^n + 1$.

\end{problems}

\endgroup % \def\divides

