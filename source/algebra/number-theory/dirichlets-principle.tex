% $date: 2015-04-01
% $timetable:
%   g9r2:
%     2015-04-01:
%       2:

\section*{Принцип Дирихле в теории чисел}

% $authors:
% - Иван Митрофанов

\begin{problems}

\item
Дана бесконечная вправо последовательность цифр и нечетное число~$l$,
не делящееся на $5$.
Докажите, что можно выбрать нескольцо цифр подряд, образующих число, делящееся
на $l$. 

\item
Докажите, что если $p$~--- простое число, то разрешимо сравнение
\[
    1 + x^2 + y^2 \equiv 0 \pmod{p}
\,.\]

\item
Докажите, что разрешимо сравнение
\[
    a x + b y \equiv 0 \pmod{m}
\]
для некоторых $|x|, |y| \leq \sqrt{m}$, $|x| + |y| > 0$.
 
\item
Дано простое число~$p$.
\\
\sp
Обозначим $k = [\sqrt{p}]$.
Назовем остатки $- k, - k + 1, \ldots, 0, 1, 2, \ldots, k$ по модулю~$p$
\emph{маленькими}.
Докажите, что для любого остатка~$a$ сущетсвует такой ненулевой маленький
остаток~$b$, что остаток~$a b$ тоже маленький.
\\
\sp
Пусть $p \mid a^2 + 1$.
Докажите, что $p$ представимо в виде суммы двух квадратов.

%\item\emph{Китайская теорема об остатках.}
%Пусть $x_1, x_2, \ldots, x_n$~--- взаимно простые числа,
%$r_1, r_2, \ldots, r_n$~--- произвольные целые числа.
%Тогда найдется такое $N$, что для любого $i$ выполнено
%\[
%    N \equiv r_i \mod {x_i}
%\,.\]

\item
Остаток~$a$ по простому модулю~$p$ называется \emph{квадритичным вычетом} или
просто \emph{вычетом}, если разрешимо сравнение $a \equiv x^2 \pmod{p}$.
Иначе $a$ называется \emph{невычетом}.
\\
\sp
Найдите количество квадритичных вычетов по модулю~$p$ и докажите, что
произведение и частное (по модулю~$p$) двух вычетов снова явдяется вычетом.
\\
\sp
Докажите, что произведение двух невычетов является вычетом.

\item
Докажите, что найдется число, представимое в виде суммы четырех квадратов целых
чисел более, чем миллионом способов.

\end{problems}

