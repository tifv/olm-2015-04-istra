% $date: 2015-04-02
% $timetable:
%   g11r2:
%     2015-04-02:
%       2:

\section*{Теория чисел}

% $authors:
% - Антон Гусев

\begin{problems}

\item
Пусть $S(n)$~--- сумма цифр числа~$n$.
Зададим последовательность $a_{n+1} = S(a_n)$, $a_0 = 2^{1000000}$.
Найдите $a_6$.

\item
Сколько простых чисел содержится в последовательности
$101, 10101, 1010101, \ldots$?

\item
Найдите все такие $n$, для которых сумма цифр числа $5^n$ равна $2^n$.

\item
Докажите, что для любого многочлена с целыми коэффициентами $P(x)$ и любого
натурального $k$, существует такое натуральное $n$, что
$P(1) + P(2) + \ldots + P(n)$ делится на $k$.

\item
Дано натуральное число $c$ и~последовательность простых чисел
$p_1, p_2, \ldots, p_n, \ldots$ такая, что $p_i + c$ делится на~$p_{i+1}$.
Докажите, что последовательность $\{p_n\}$ ограничена.

\item
Докажите, что $\phi(2^n - 1)$ делится на~$n$.

\item
Число~$N$, не~кратное 81 представимо в~виде суммы квадратов трех чисел,
делящихся на~3.
Докажите, что оно представимо в~виде суммы квадратов трех чисел, не~делящихся
на~3.

\item
Докажите, что для любого натурального $n$ число $2^{3^n} + 1$ делится
на $3^{n+1}$, и не делится на $3^{n+2}$.

\item
\subproblem
Дано простое число $p$ и~натуральное $a$.
Докажите, что любой простой делитель числа $a^{p-1} + a^{p-2} + \ldots + a + 1$
либо равен $p$, либо имеет вид $p k + 1$.
\\
\subproblem
Докажите, что простых чисел вида $p k + 1$ бесконечно много.

\end{problems}

