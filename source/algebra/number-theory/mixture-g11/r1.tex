% $date: 2015-03-30
% $timetable:
%   g11r1:
%     2015-03-30:
%       2:

\section*{Алгебра}

% $authors:
% - Антон Гусев

\begin{problems}

\item
Найдите все такие $n$, для которых сумма цифр числа $5^n$ равна $2^n$.

\item
Число~$N$, не~кратное 81 представимо в~виде суммы квадратов трех чисел,
делящихся на~3.
Докажите, что оно представимо в~виде суммы квадратов трех чисел, не~делящихся
на~3.

\item
Дано натуральное число $c$ и~последовательность простых чисел
$p_1, p_2, \ldots, p_n, \ldots$ такая, что $p_i + c$ делится на~$p_{i+1}$.
Докажите, что последовательность $\{p_n\}$ ограничена.

\item
Докажите, что $\phi(2^n - 1)$ делится на~$n$.

\item
Даны 4 точки на~плоскости.
Все попарные расстояния между ними~--- нечетные целые числа.
Докажите, что такого быть не~может. 

\item
Пусть $p_n$~--- минимальный простой делитель числа $(n!)^n + 1$.
Докажите, что существует $N$ такое, что при всех $n > N$ выполнено
$p_n > n + 2015$.

\item
\sp
Дано простое число $p$ и~натуральное $a$.
Докажите, что любой простой делитель числа $a^{p-1} + a^{p-2} + \ldots + a + 1$
либо равен $p$, либо имеет вид $p k + 1$.
\\
\sp
Докажите, что простых чисел вида $p k + 1$ бесконечно много.

\item
Докажите, что не~существует целых $x, y$, что $x^2 + 5 = y^3$.

\item
Известно, что $2^{2k} + 2^k + 1 = p$~--- простое число.
Докажите, что $2^{2^k+1} - 1$ делится на~$p$.

\end{problems}

