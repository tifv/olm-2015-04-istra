% $date: 2015-04-03
% $timetable:
%   g11r1:
%     2015-04-03:
%       2:

\section*{Теория чисел. Добавка}

% $authors:
% - Антон Гусев

\begin{problems}

\item
Пусть $a, b, c \in \mathbb{N}$, и~$\frac{a^2 + b^2}{a b + 1} = c$.
Докажите, что $c$~— точный квадрат.

\item
Пусть $d(n)$~— число делителей натурального числа~$n$.
Найдите все возрастающие последовательности натуральных чисел
$a_1, a_2, \ldots$ такие, что $d(a_i + a_j) = d(i + j)$ при всех натуральных
$i$ и~$j$.

\item
Числовой треугольник строится следующим образом: в~первой строчке стоит одно
число, а~в~каждой следующей строчке числа располагаются под предыдущими,
и~добавляется по~одному числу с~краю.
Число в~первой строчке равно 1;
каждое из~остальных равно сумме трех чисел в~предыдущей строчке:
из~той~же вертикали и~из~двух соседних (пустые места считаются нулями).
Докажите, что в~среднем столбце нет чисел, сравнимых с~2 по~модулю 3.

\item
Найдите все пары многочленов $P(x), Q(x)$ с~целыми коэффициентами такие, что
для всех натуральных $n$ выполняется равенство
\(
    P(1) \cdot P(2) \cdot \ldots \cdot P(n)
=
    Q(n!)
\).

\item
Для натурального~$k$ определим последовательность $a_1, a_2, \ldots$ условиями
$a_1 = k + 1$ и~$a_{n+1} = a_1 a_2 \ldots a_n + k$ при всех натуральных~$n$.
При каких $k$ в~этой последовательности бесконечно много точных квадратов?

\item
Даны натуральные числа $a, b, k$.
Для последовательности натуральных чисел $\{ x_n \}$ выполнено
$x_{n+2} = a x_{n+1} + b x_{n}$.
Докажите, что существует натуральное~$N$ такое, что при $n > N$ все
числа $x_i$, делящиеся на~$k$, будут лежать через равные промежутки.

\item
Можно~ли раскрасить каждое натуральное число в~один из~трех цветов так, чтобы
не~нашлось трех различных одноцветных натуральных чисел $x$, $y$ и~$z$, для
которых $x + y = z^2$?

\end{problems}

