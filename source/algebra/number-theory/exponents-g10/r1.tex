% $date: 2015-03-30
% $timetable:
%   g10r1:
%     2015-03-30:
%       3:

\section*{Показатели}

% $authors:
% - Андрей Меньщиков

\definition
Натуральное число~$d$ называется \emph{показателем числа~$a$ по~модулю~$m$},
если $d$ является наименьшим натуральным числом таким, что
$a^d \equiv 1 \pmod{m}$.
Из~теоремы Эйлера следует, что если $(a, m) = 1$, то~такой показатель~$d$
существует.
Если~же $(a, m) > 1$, то~такого $d$ не~существует.

\begin{problems}

\item
Числа $a$, $m$, $l$~— натуральные, $(a, m) = 1$.
Докажите, что $a^l \equiv 1 \pmod{m}$ равносильно тому, что $d \divides l$.

\end{problems}

В~частности, из~этой задачи следует, что показатель числа $a$ по~модулю $m$
делит~$\mathrm{\phi}(m)$.

\begin{problems}

\item
Пусть $p = 3 k + 2$~— простое.
\\
\subproblem
Докажите, что сравнение $x^3 \equiv 1 \pmod{p}$ имеет единственное решение.
\\
\subproblem
Докажите, что для любого оcтатка~$a$ сравнение $x^3 \equiv a \pmod{p}$ имеет
единственное решение.

\item
Пусть $p$~— простое, $p > 2$.
\\
\subproblem
Докажите, что любой простой делитель числа $2^p - 1$ представим
в~виде $2 k p + 1$, где $k \in \mathbb{N}$.
\\
\subproblem
Докажите, что любой простой делитель числа $2^p + 1$, больший 3, представим
в~виде $2 k p + 1$, где $k \in \mathbb{N}$.
\\
\subproblem
Докажите, что любой простой делитель числа $2^p + 3^p$, больший 5, представим
в~виде $2 k p + 1$, где $k \in \mathbb{N}$.
\\
\subproblem
Пусть $n \in \mathbb{N}$.
Докажите, что любой делитель числа $2^{2^n} + 1$ представим
в~виде $2^{n+1} \cdot k + 1$, где $k \in \mathbb{N}$.

\item
Пусть $p$~— простое.
Докажите, что у~числа $(p^p - 1)$ есть простой делитель, дающий остаток 1 при
делении на~$p$.

\item
Докажите, что для любых натуральных $a$ и~$n$, больших единицы, справедливо
$n \divides \mathrm{\phi}(a^n - 1)$.

\item
Найдите все пары $(p, q)$ простых чисел таких, что
$p q \divides (7^{p} - 2^{p}) \cdot (7^{q} - 2^{q})$.

\item
Найдите все пары простых чисел $(p, q)$ такие, что $p q \divides 5^{p} + 5^{q}$.

\item
Найдите все натуральные $n$ такие, что $n \divides (2^n - 1)$.

\item
Найдите все натуральные $n$, для которых числа $n$ и~$2^n + 1$ имеют один
и~тот~же набор простых делителей. 	

\item
Для натурального~$a$ обозначим через $P_a$ множество всех простых чисел,
не~являющихся делителями ни~одного из~чисел вида $(2^{k \cdot 2^a} - 1)$, где
$k$~— нечетное натуральное число.
Докажите, что при любом натуральном $a$ множество $P_a$ бесконечно.

\item
Натуральное~$n$ таково, что число $4^n + 2^n + 1$~— простое.
Докажите, что $n$~— степень тройки.

\item
Найдите все натуральные $n$ такие, что $n^2 \divides 2^n + 1$.

\end{problems}

