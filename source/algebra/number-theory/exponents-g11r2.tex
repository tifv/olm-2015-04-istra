% $date: 2015-04-04
% $timetable:
%   g11r2:
%     2015-04-04:
%       3:

\section*{Показатели}

% $authors:
% - Антон Гусев

\definition
При $(a, m) = 1$ существует натуральное $\delta$ с условием
\(
    a^{\delta} \equiv 1 \pmod{m}
\).
Наименьшее из таких чисел называется \emph{показателем $a$ по модулю~$m$}.

\begin{problems}

\item
В~обозначениях из~предыдущего определения:
\\
\subproblem
Числа $1 = a^0$, $a^1$, \ldots, $a^{\delta-1}$ попарно несравнимы
по~модулю~$m$.
\\
\subproblem
$a^s \equiv a^t \pmod{m}$ ($s, t \geq 0$)
тогда и~только тогда, когда
$s \equiv t \pmod{\delta}$.
В~частности, $a^s \equiv 1 \pmod{m}$
тогда и~только, когда
$s$ делится на~$\delta$.
\\
\subproblem
Число $\delta$ является делителем $\phi(m)$.

\end{problems}

\medskip
\hrule

\begin{problems}

\item
Найдите все простые $p$ и~$q$ такие, что $(2^p - 1)$ делится на~$q$,
а~$(2^q - 1)$ делится на~$p$.

\item
Докажите, что для любого натурального $n$ простые делители числа $2^{2^n} + 1$
имеют вид $2^{n+1} x + 1$.

\item
Докажите, что для любого натурального $a > 1$ количество правильных
несократимых дробей со~знаменателем $(a^n - 1)$ кратно $n$.

\item
$p$, $q$~--- простые числа, $q > 5$.
Докажите, что если $q \mid 2^p + 3^p$, то~$q > p$.

\item
Докажите, что $(2^n - 1)$ не~делится на~$n$ при натуральном $n > 1$.

\item
Пусть $a > 1$, $p > 2$, $p$~--- простое.
\\
\subproblem
Докажите, что простые нечетные делители числа $(a^p - 1)$ или делят $(a - 1)$,
или имеют вид $2 p x + 1$.
\\
\subproblem
Докажите, что число $(a^p - 1) / (a - 1)$ имеет хотя~бы один простой
множитель, не~являющийся делителем $(a - 1)$.
\\
\subproblem
Докажите бесконечность множества простых чисел вида $2 p x + 1$.

\item
Известно, что число $2^{32} + 1$ раскладывается на~простые множители как
$641 \cdot 6700417$.
Докажите, что существует такое натуральное $k$, что для любого
натурального $n$ число $k 2^n + 1$ будет составным.

\end{problems}

