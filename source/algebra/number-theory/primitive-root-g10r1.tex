% $date: 2015-04-01
% $timetable:
%   g10r1:
%     2015-04-01:
%       2:

\section*{Первообразные корни}

% $authors:
% - Андрей Меньщиков

\definition
Если $(a, m) = 1$ и~показатель числа~$a$ по~модулю~$m$ равен $\varphi(m)$,
то~$a$ называется \emph{первообразным корнем} по~модулю~$m$.

\claim{Замечание}
Тем самым, $a = a^0, a^1, a^2, \ldots, a^{\phi(m) - 1} \pmod{m}$~---
это \emph{все} вычеты, взаимно простые с~$m$.

\claim{Упражнение}
Докажите, что точный квадрат натурального числа не~может быть первообразным
корнем ни~по~какому простому нечетному модулю.

\begin{problems}

\item
Пусть по~модулю~$m$ существует первообразный корень.
\\
\sp
Сколько существует остатков~$a$, для которых $a^d \equiv 1 \pmod m$?
\\
\sp
А~сколько существует первообразных корней по~модулю~$m$?

\end{problems}

\claim{Напоминание}
Над $\mathbb{Z}_p$ многочлен степени $d \neq 0$ имеет не~более $d$~корней.

\begin{problems}

\item
\sp
Докажите, что
\(
    n = \sum_{d \mid n} \phi(d)
\)
(для этого рассмотрите дроби $\frac{1}{n}, \frac{2}{n}, \ldots, \frac{n}{n}$
и~подсчитайте их~количество).
\\
Далее, пусть $p$~--- простое.
\\
\sp
Докажите, что для любого $d$ существует не~более $d$~оcтатков по~модулю~$p$
с~показателем, делящим~$d$.
\\
\sp
Докажите, что для $d \mid (p - 1)$ существует ровно $d$~остатков по~модулю~$p$
с~показателем, делящим~$d$.
\\
\sp
Докажите, что для $d \mid (p - 1)$ существует ровно $\phi(d)$ остатков
по~модулю~$p$ с~показателем~$d$
(в~частности, $\phi(p - 1) > 0$, поэтому первообразные корни по~простому модулю
существуют).

\item
Пусть $p = 3 k + 1$~--- простое.
\\
\sp
Докажите, что сравнение $x^3 \equiv 1 \pmod p$ имеет три решения.
\\
\sp
Докажите, что для любого остатка~$a$ сравнение $x^3 \equiv a \pmod p$ либо
не~имеет решений, либо имеет три решения.
А~для скольких остатков~$a$ такое сравнение имеет решение?

\item
\sp
Докажите, что числа $1, 2, \ldots, (p - 1)$ можно расставить по~окружности так,
что квадрат любого числа будет сравним по~модулю~$p$ с~произведением его
соседей.
\\
\sp
Для каждого натурального~$d$ найдите
\(
    \sum_{n=0}^{p-1} n^d \pmod p
\).
\\
\sp
При помощи первообразных корней докажите, что
\(
    \genfrac{(}{)}{}{}{a}{p}
\equiv
    a^{\frac{p-1}{2}}
\pmod p
\).

\item
\sp\textbf{Усиление теоремы Эйлера.}\enspace
Пусть
\(
    m
=
    p_1^{\alpha_1} \cdot p_2^{\alpha_2} \cdot \ldots \cdot p_n^{\alpha_n}
\)~--- разложение $m$ на~простые множители.
Докажите, что для любого $a$, взаимно простого с~$m$, выполняется сравнение
\[
    a^x \equiv 1 \pmod m
\text{,\quadгде\enspace}
    x
=
    \text{НОК} \bigr(
        \phi(p_1^{\alpha_1}),
        \phi(p_2^{\alpha_2}),
        \ldots,
        \phi(p_n^{\alpha_n})
    \bigl)
.\]
\sp
Докажите, что если натуральное~$m$ делится на~два различных простых нечетных
числа или~же на~простое нечетное число и~на~4, то~по~модулю $m$ первообразного
корня не~существует.
\\
\sp
Докажите, что для нечетного простого $p$ и~натуральных $u$, $r$
($u$ не~делится на~$p$) число $(1 + p u)^{p^r} - 1$ делится на~$p^{r+1}$,
но~не~делится на~$p^{r+2}$.
\\
\sp
Пусть $g$~--- первообразный корень по~модулю простого нечетного~$p$.
Докажите, что хотя~бы одно из~чисел $g$ и~$g + p$ является первообразным корнем
по~модулю~$p^{\alpha}$ для любого $\alpha$.
\\
\sp
Пусть $g$~--- первообразный корень по~модулю $p^{\alpha}$.
Докажите, что одно из~чисел $g$ и~$g + p^{\alpha}$ является первообразным
корнем по~модулю~$2 p^{\alpha}$.
\\
\sp
Докажите, что первообразные корни существуют только по~модулям
$1$, $2$, $4$, $p^{\alpha}$, $2 p^{\alpha}$
(где $p$~--- простое нечетное число).

\item
Докажите, что для каждого натурального~$n$ найдется натуральное~$m$ такое, что
$3^n \mid 2^m + 2015$.

\item
Андрей задумал $s$ натуральных чисел и~выписал на~доску все их~попарные суммы,
в~том числе суммы числа с~самим собой.
Для какого наибольшего $s$ могло оказаться так, что все выписанные числа дают
разные остатки по~модулю $(p^2 - p)$, где $p$~--- простое?

\end{problems}

