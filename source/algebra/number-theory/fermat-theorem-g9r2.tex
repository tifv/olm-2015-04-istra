% $date: 2015-03-31
% $timetable:
%   g9r2:
%     2015-03-31:
%       2:

% $caption: Малая теорема Ферма

\section*{Зацикливание остатков. Немного о малой теореме Ферма}

% $authors:
% - Владимир Брагин
% - Иван Митрофанов

\begin{problems}

\itemy{0}
Бывает~ли так, что $7^k + 1$ делится на~$32$?

\item
Известно, что $2^l + 3^k$ делится на~$5$.
Докажите, что $(k + l - 2)$ делится на~$4$.

\end{problems}

Если число является степенью простого, то~других простых делителей у~него нет.

\begin{problems}

\item
Докажите, что $(2^k - 1)$ не~может быть степенью
\\
\subproblem пятерки;
\quad
\subproblem тройки (кроме случая $(2^2 - 1) = 3^1$).

\item
Какие члены последовательности Фибоначчи являются степенями двойки или тройки?

\item
Может~ли число $(12 k + 5) \cdot (12 l + 7)$ быть степенью простого числа?

\item\claim{Геометрическое доказательство малой теоремы Ферма}
Сколькими способами можно покрасить вершины правильного $p$-угольника
в~$a$~цветов?
(Способы, получающиеся друг из~друга поворотами, считаем одинаковыми.)

\item
\subproblem
Натуральные $a$, $b$, $c$, $d$, $e$ таковы, что
\(
    13 \divides a^{12} + b^{12} + c^{12} + d^{12} + e^{12}
\).\\
Докажите, что $13^5 \divides a b c d e$.
\\
\subproblem
Натуральные $a$, $b$, $c$, $d$, $e$ таковы, что
\(
    13 \divides a^{6} + b^{6} + c^{6} + d^{6} + e^{6}
\).\\
Докажите, что $13 \divides a b c d e$.

\item
Докажите, что любой простой делитель числа $(2^p - 1)$ имеет вид $2 k p + 1$.

\end{problems}

